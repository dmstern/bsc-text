
\newglossaryentry{eclipse}{
	name=Eclipse, 
	description={Eine u.a. durch Plugins stark anpassbare Entwicklungsumgebung von der \textsl{Eclipse Foundation}},
	type = technologies
}

\newglossaryentry{ios-dev-prog}{
	name={iOS Developer Program}, 
	description={},
	type = ignored
}

\newglossaryentry{dev-profile}{
	name={Developer Profile}, 
	description={},
	type = ignored
}

\newglossaryentry{app}{
	name = App,
	description = {Kurzform für engl. \enquote{Application} (dt. \enquote{Anwendung}) im Sinne von Anwendungssoftware. Im deutschsprachigen Raum meist im Zusammenhang mit Smartphones oder Tablet-Computern},
	first = {App (Kurzform f\"ur engl. \enquote{Application})},
	firstplural = {Apps (Kurzform f\"ur engl. \enquote{Application})}
}

\newglossaryentry{plattform}{
	name = Plattform,
	description = {Basis-Software-Umgebung, in der eine andere Software (bspw. Anwendungen und Dienste) ausgef\"uhrt wird. Hier ist das Betriebssystem mobiler Ger\"ate gemeint},
	plural = Plattformen
}

\newglossaryentry{web-app}{
	name = Web-App,
	description = {},
	first = {Web-App (oder dt. \enquote{Web-Anwendung})},
	firstplural = {Web-Apps (oder dt. \enquote{Web-Anwendungen})}
}

\newglossaryentry{hybrid-app}{  %S
	name = Hybrid-App,
	description = {Anwendungssoftware, die Web-Technologien verwendet},
}

\newglossaryentry{html}{
	name = HTML,
	description = {HTML (Hyper Text Markup Language) ist eine Auszeichnungssprache für Websites. \enquote{HTML is the publishing language of the World Wide Web} \cite{w3c_html}},
	type = technologies
}

\newglossaryentry{xml}{ %S
	name = XML,
	description = {Extensible Markup Language (dt. \enquote{erweiterbare Auszeichnungssprache}). Eine Auszeichnugssprache zur Darstellung hirachischer Inhalte in Textform},	%TODO Beschreibung einfuegen!
	type = technologies
}

\newglossaryentry{html5}{
	name = HTML5,
	description = {Version 5 der \gls{html}-Spezifikation. Unter HTML5 werden im Allgemeinen Web-Technologien wie \gls{html}, \gls{css} und \gls{js} zusammengefasst, die es Entwicklern erm\"oglichen, auch komplexere Web-Anwendungen ohne die Notwendigkeit von zus\"atzlichen Technologien wie Browserplugins etc. zu entwickeln},
	type = technologies
}

\newglossaryentry{sdk}{
	name = SDK,
	description = {Software Development Kit},
	type = \acronymtype,
}

\newglossaryentry{listview}{ %S
	name = Listview,
	description = {Ein scrollbarer Container zur Anzeige von Elementen in einer Liste},
}

\newglossaryentry{errorhandler}{%S
	name = Error-Handler,
	description = {dt. \enquote{Fehlerbehandler}. Spezieller Teil im Programmcode, der aufgerufen wird, wenn ein Fehler zur Laufzeit auftritt.},
	%Quelle: http://dictionary.reference.com/browse/exception+handler
}

\newglossaryentry{flag}{%S
	name = Flag,
	description = {Statusvariable zur Kennzeichnung bestimmter Zustände},
}

\newglossaryentry{css}{
	name = CSS,
	description = {Cascading Style Sheets},
	type = technologies,
}

\newglossaryentry{dom}{
	name = DOM,
	description = {Document Object Model},
	type = \acronymtype,
}

\newglossaryentry{ide}{
	name = Entwicklungsumgebung,
	description = {Eine Entwicklungsumgebung (IDE, für engl. \enquote{Integrated Development Environment}) ist eine Software, mit der Computer-Programme entwickelt werden k\"onnen.},
}

\newglossaryentry{string}{
	name = String,
	description = {Eine Zeichenkette in der Informatik.},
}

\newglossaryentry{framework}{%S
	name = Framework,
	description = {dt. \enquote{Rahmenstruktur}. Ein Rahmenwerk zur Erstellung von Anwendungssoftware.},
	%Eine halbvollständige Application, die zur Erstellung weiter Applicationen als wiederverwendbarer Rahmen dient
	%Quelle: http://www.itwissen.info/definition/lexikon/Framework-framework.html
}

\newglossaryentry{event}{%S
	name = Event,
	description = {dt. \enquote{Ereignis}. Eine plötzliche Veränderung eines Zustands oder Parameters, auf die ein Programm reagiert.},
	%Fast wörtlich übernommen, Quellenangabe also wohl wichig
	%Quelle: http://www.itwissen.info/definition/lexikon/Ereignis-event.html
}

\newglossaryentry{handler}{%S
	name = Handler,
	description = {dt. \enquote{Behandler}},
}

\newglossaryentry{event-handler}{%S
	name = Event-Handler,
	description = {dt. \enquote{Ereignisbehandler}. Spezieller Teil im Programmcode, der aufgerufen wird, wenn ein Event ausgelöst wird.},
}

\newglossaryentry{observer-pattern}{%S
	name = Observer-Pattern,
	description = {Ein Entwurfsmuster (engl. \enquote{pattern}) in der Softwareentwicklung, bei dem Objekte zueinander in Beziehung stehen um konsisten gehalten zu werden. Ändert sich der Zustand eines Objekts, werden alle davon abhängigen Objekte (observer) benachrichtigt und entsprechend aktualisiert.},
	%Quelle: meine alte Software-Entwicklungs Vorlesung, 7. Vorlesung, 29. Folie
}

\newglossaryentry{observer}{
	name = Observer,
	first = {Observer (engl. \enquote{Beobachter})},
	firstplural = {Observers (engl. \enquote{Beobachter})},
	description = {},
}

\newglossaryentry{gui}{
	name = GUI,
	description = {Graphical User Interface (dt. \enquote{Grafische Benutzeroberfl\"ache})},
	type = \acronymtype,
}

\newglossaryentry{mvvm}{
	name = MVVM-Pattern,
	description = {Model-View-ViewModel-Pattern. Ein Entwurfsmuster (engl. \enquote{pattern}) der Informatik zur Trennung von Benutzeroberfl\"ache und \gls{ui}-Logik},
	first = {Model-View-ViewModel\,-\-Pattern (MVVM)},
	type = \acronymtype,
}

\newglossaryentry{mvc}{%S
	name = MVC,
	description = {Model-View-Controller},
	type = \acronymtype,
}

\newglossaryentry{view-model}{
	name = ViewModel,
	description = {},
}

\newglossaryentry{model}{
	name = Model,
	description = {},
}

\newglossaryentry{view}{
	name = View,
	description = {},
}

\newglossaryentry{viewCtrl}{
	name = ViewController,
	description = {},
}

\newglossaryentry{web-view}{
	name = WebView,
	description = {},
}

\newglossaryentry{cross-compiling}{%S
	name = Cross-Compiling,
	description = {Kompiliervorgang, bei dem Builts für andere Plattformen als diejenige, auf der der Compiler läuft, hergestellt werden},
}

\newglossaryentry{built}{%S
	name = Built,
	description = {Ein ausführbares Programm},
}

\newglossaryentry{frontend}{%S
	name = Frontend,
	description = {Softwareschicht, die nahe der Benutzerschnittstelle liegt.},
}

\newglossaryentry{mono}{%S
	name = Mono,
	description = {Open source Implementierung von Microsofts .NET Framework.},
	type = technologies
	%Quelle: http://www.mono-project.com/Main_Page
}

\newglossaryentry{bytecode}{%S
	name = Bytecode,
	description = {Kompilierter Code, der auf einer virutellen Maschine ausgeführt werden kann.},
	type = technologies
}

\newglossaryentry{nekovm}{%S
	name = NekoVM,
	description = {Programmiersprache},
	type = technologies
}

\newglossaryentry{lime}{
	name = Lime,
	description = {Haxe Bibliothek},
	type = technologies
}

\newglossaryentry{osgi}{%S
	name = OSGi,
	description = {Open Services Gateway initiative. Eine auf der JVM aufbauende hardwareunabhängige Softwareplattform},
	type = technologies
}

\newglossaryentry{titanium}{%S
	name = Titanium,
	description = {Entwicklungsumgebung der Firma Appcelerator},
	type = technologies
}

\newglossaryentry{appcelerator}{%S
	name = Appcelerator,
	description = {Amerikanische Softwarefirma},
	type = technologies
}

\newglossaryentry{alloy}{%S
	name = Alloy,
	description = {Framework zu Erstellung von Benutzeroberflächen},
	type = technologies
}

\newglossaryentry{openfl}{%S
	name = OpenFL,
	description = {Haxe Bibliothek},
	type = technologies
}

\newglossaryentry{flash-editing-environment}{%S
	name = Flash-Editing-Environment,
	description = {Entwicklungsumgebung für Flash-Software},
	type = technologies
}

%war ein Rechtschreibfehler drin: nicht cocao
\newglossaryentry{cocoa-touch-widgets}{%S
	name = Cocoa Touch Widgets,
	description = {Benutzeroberflächenframework der Firma Apple},
	type = technologies
}

\newglossaryentry{xmlvm}{%S
	name = XMLVM,
	description = {Projektname eines Cross-Compilers},
	type = technologies
}

\newglossaryentry{monotouch}{%S
	name = MonoTouch,
	description = {Framework zu Erstllung von iOS Apps},
	type = technologies
}

\newglossaryentry{xamarin}{%S
	name = Xamarin,
	description = {Amerikanische Softwarefirma},
	type = technologies
}

\newglossaryentry{c-sharp}{%S
	name = C\#,
	description = {Eine von Microsoft entwickelte objektorientierte Programmiersprache},
	type = technologies
}

\newglossaryentry{cpp}{%S
	name = C++,
	description = {Eine objektorientierte Erweiterung der Programmiersprache C},
	type = technologies
}

\newglossaryentry{look-and-feel}{%S
	name = Look-And-Feel,
	description = {Standartisierte Design-Aspekte einer Software},
}

\newglossaryentry{intel}{
	name = Intel,
	description = {},
	type = ignored
}

\newglossaryentry{ui}{
	name = UI,
	description = {User Interface (dt. \enquote{Benutzerschnittstelle)}},
	type = \acronymtype,
}

\newglossaryentry{asp}{
	name = ASP,
	description = {Active Server Pages},
	type = ignored
}

\newglossaryentry{swf}{%S
	name = SWF,
	description = {Shockwave Flash. Ein Dateiformat.},
	type = \acronymtype,
}

\newglossaryentry{swt}{%S
	name = SWT,
	description = {Standard Widget Toolkit. Eine Benutzeroberflächenbibliothek für Java},
	type = \acronymtype,
}

\newglossaryentry{jface}{%S
	name = JFace,
	description = {Ein auf SWT aufbauendes Benutzeroberflächen-Toolkit},
}

\newglossaryentry{flash}{%S
	name = Flash,
	description = {Plattform zur Darstellung und Programmierung multimedlier Inhalte der Firma Adobe},
	type = ignored
}

\newglossaryentry{gps}{
	name = GPS,
	description = {Global Positioning System},
	type = \acronymtype,
}

\newglossaryentry{php}{
	name = PHP,
	description = {PHP: Hypertext Preprocessor},
	type = ignored,
}

\newglossaryentry{w3c}{
	name = W3C,
	description = {World Wide Web Consortium},
	type = \acronymtype
}

\newglossaryentry{android}{
	name = Android,
	description = {Smartphone- und Tablet-Betriebssystem von \gls{google}},
	type = technologies
}

\newglossaryentry{ios}{
	name = iOS,
	description = {Smartphone- und Tablet-Betriebssytem der Firma \gls{apple}},
	type = technologies
}

\newglossaryentry{ios-sim}{%S
	name = iOS-Simulator,
	description = {Eine Software für Apples OSX zur Simulation von iOS},
	type = technologies
}

\newglossaryentry{amazon-fireos}{
	name = Amazon Fire\-OS,
	description = {},
	type = technologies
}

\newglossaryentry{ubuntu-phone}{%S
	name = U\-bun\-tu \mbox{Phone},
	description = {Ein Smartphone-Betriebsystem der Firma Cononical},
	type = technologies
}

\newglossaryentry{tizen}{
	name = Tizen,
	description = {},
	type = technologies
}

\newglossaryentry{webos}{%S
	name = webOS,
	description = {Ein open source Smartphone- und Tablet-Betriebssystem},
	type = technologies
}


\newglossaryentry{symbian}{%S
	name = Symbian,
	description = {Handy- und Smartphone-Betriebsystem der Firma Nokia},
	type = technologies
}

\newglossaryentry{win-phone}{
	name = Windows Phone,
	first = Microsoft~Windows~Phone,
	description = {Smartphone-Betriebssytem von Microsoft},
	type = technologies
}

\newglossaryentry{win8}{
	name = Windows\,8,
	first = Microsoft~Windows~8,
	description = {Aktuelles Desktop-\,/\,Hybrid-Betriebssystem der Firma Microsoft},
	type = technologies
}

\newglossaryentry{blackberry-os}{
	name = Blackberry\,10,
	description = {Smartphone-Betriebssystem f\"ur \gls{blackberry}-Ger\"ate der Firma \gls{blackberry-inc}. Blackberry\,10 ist der Nachfolger des vorherigen Betriebssystems \emph{Blackberry\,OS}},
	type = technologies
}

\newglossaryentry{blackberry}{
	name = Blackberry,
	description = {Smartphone-Reihe der Firma \gls{blackberry-inc}},
	type = ignored
}

\newglossaryentry{blackberry-inc}{
	name = Blackberry,
	description = {Smartphone-Hersteller des gleichnamigen Smartphones \gls{blackberry}},
	type = ignored
}

\newglossaryentry{java}{
	name = Java,
	description = {Eine plattformunabh\"angige, objektorientierte Programmiersprache},
	type = ignored
}

\newglossaryentry{javaee}{%S
	name = JavaEE,
	first = {JavaEE (Java Platform Enterprise Edition)},
	description = {Java Plattform der Firma Oracle},
}

\newglossaryentry{tabris}{%S
	name = Tabris,
	description = {Framework zur Erstellung plattformunabhängiger Client-Server Mobilanwendungen mit Java.},
}

\newglossaryentry{tabris-ui}{%S
	name = Tabris UI,
	first = {Tabris User Interface Framework (Tabris UI)},
	description = {Framework zur Erstellung von Benutzeroberflächen mit Tabris.},
}

\newglossaryentry{actionbar}{
	name = ActionBar,
	description = {Eine vierteilige Menüleiste am oberen Bildschirmrand des mobilen Betriebssystems \gls{android}},
}

\newglossaryentry{xamarin-android}{%S
	name = Xamarin.Android,
	description = {Framework zur Programmierung von Android Apps in \gls{c-sharp}},
}

\newglossaryentry{xamarin-ios}{%S
	name = Xamarin.iOS,
	description = {Framework zur Programmierung von iOS Apps in \gls{c-sharp}.},
}

\newglossaryentry{netbeans}{
	name = {Net\-beans\,\-IDE},
	description = {Eine \gls{ide} auf \gls{java}-Basis von der \textsl{Oracle Corpration}},
	type = technologies
}

\newglossaryentry{intellij}{
	name = {IntelliJ\,IDEA},
	description = {Eine \gls{ide} f\"ur \gls{java} der Firma JetBrains},
	type = technologies
}

\newglossaryentry{android-studio}{
	name = {Android\,\-Studio},
	description = {Eine \gls{ide} f\"ur die Entwicklung von \glspl{app} f\"ur das Betriebssytem \gls{android}},
	type = technologies
}

\newglossaryentry{obj-c}{
	name = ObjectiveC,
	description = {Eine um objektorientierte Elemente erweiterte Variante der Programmiersprache \gls{c}},
	type = technologies
}

\newglossaryentry{c}{
	name = C,
	description = {Eine imperative Programmiersprache},
	type = technologies
}

\newglossaryentry{xcode}{
	name = Xcode,
	description = {Eine \gls{ide} f\"ur die Entwicklung von \gls{ios}- und \gls{osx}-Software},
	type = technologies
}

\newglossaryentry{osx}{
	name = Mac\,OS\,X,
	description = {Betriebsystem f\"ur Mac-Rechner der Firma \gls{apple}},
	type = technologies
}

\newglossaryentry{gmail}{
	name = Gmail,
	description = {},
	type = ignored
}

\newglossaryentry{gmx}{
	name = GMX,
	description = {},
	type = ignored
}

\newglossaryentry{web-de}{
	name = Web.de,
	description = {},
	type = ignored
}

\newglossaryentry{js}{
	name = {Java\-Script},
	description = {Eine Skriptsprache, die in einer gr\"o\"seren Umgebung (wie beispielsweise einem Browser) ausgef\"uhrt wird und in diesem Kontext z.\,B. verwendet werden kann, um \gls{html} um dynamische Elemente, wie der Ver\"anderung von Inhalten, aber auch Ausf\"uhrung von Prozeduren etc., zu erweitern \cite{w3c_js}},
	type = technologies
}

\newglossaryentry{json}{
	name = {JSON},
	description = {JavaScript Object Notation},
	type = technologies
}

\newglossaryentry{webstorage}{
	name = Web\-Storage,
	description = {WebStorage (auch: \enquote{LocalStorage} ist eine Spezifikation des \gls{w3c} zur lokalen Speicherung von gr\"o\"seren Datenmengen einer Webanwendung auf Client-Seite in Form von Schl\"ussel-Wert-Paaren.) \cite{w3c_webstorage}},
	type = technologies
}

\newglossaryentry{indexed-db}{
	name = Indexed\-DB,
	description = {Eine Datenbank...}, %TODO ERKL\"ARUNG EINF\"UGEN!
	type = technologies
}

\newglossaryentry{websql}{
	name = Web\-SQL,
	description = {}, %TODO ERKL\"ARUNG EINF\"UGEN!
	type = technologies
}

\newglossaryentry{phonegap}{
	name = Phone\-Gap,
	description = {}, %TODO ERKL\"ARUNG EINF\"UGEN!
	type = technologies
}

\newglossaryentry{cordova}{
	name = Cordova,
	description = {}, %TODO ERKL\"ARUNG EINF\"UGEN!
	type = technologies
}

\newglossaryentry{app-framework}{
	name = App Framework,
	description = {}, %TODO ERKL\"ARUNG EINF\"UGEN!
	type = technologies
}

\newglossaryentry{plugin-registry}{
	name = Plugin Registry,
	description = {}, %TODO ERKL\"ARUNG EINF\"UGEN!
	first = Apache Cordova Plugin Registry,
	type = technologies
}

\newglossaryentry{contacts-api}{
	name = Contacts API,
	description = {}, %TODO ERKL\"ARUNG EINF\"UGEN!
	first = Cordova Contacts API,
	type = technologies
}

\newglossaryentry{plugin-dev-guide}{
	name = Plugin Development Guide,
	description = {}, %TODO ERKL\"ARUNG EINF\"UGEN!
	type = technologies
}

\newglossaryentry{cli}{
	name = CLI, %TODO Evtl auch "Kommandozeilen-Werkzeug" oder nur "Kommandozeile" ? 
	first = {Kommandozeilen-Werkzeug (engl. Command-Line Interface, CLI)},
	description = {Kommandozeilen-Werkzeug (oder \enquote{Befehlszeilenschnittstelle} von engl. Command-Line Interface (CLI))}, %TODO ERKL\"ARUNG EINF\"UGEN!
	type = technologies
}

\newglossaryentry{pg-build}{
	name = {Phone\-Gap\,Build}, %TODO Hier m\"usste eigentlich ne \mbox drum, steht dann aber st\"andig \"uber.
	description = {Online-Portal von \gls{adobe}, das den Build-Prozess von \gls{phonegap}-\glspl{app} in die Cloud auslagert und damit die Anbindung von \glspl{hybrid-app} an plattformspezifische Komponenten verschiedener Mobilger\"ate erleichtern soll},
	type = technologies
}

\newglossaryentry{github}{
	name = {Git\-Hub},
	description = {Online-Portal, in dem \gls{git}-Repositories erstellt und verwaltet werden k\"onnen mit dem Fokus auf \gls{opensource}-Software},
	type = technologies
}

\newglossaryentry{git}{%S
	name = {Git},
	description = {Eine Versionsverwaltungssoftware},
	type = technologies
}

\newglossaryentry{qr}{%S
	name = {QR-Code},
	description = {Eine zweidimensionale Darstellung von Zeichenketten},
}

\newglossaryentry{plugin}{%S
	name = {Plugin},
	description = {Eine Erweiterung für eine Software.},
}

\newglossaryentry{id}{
	name = {ID},
	description = {engl. \enquote{Identifier}. Eindeutiger Kenner (bspw. eines Objekts).},
	type = \acronymtype
}

\newglossaryentry{safari}{
	name = {Safari},
	description = {Standard-Webbrowser von \gls{apple} \gls{osx}},
	type = ignored
}

\newglossaryentry{firefox}{
	name = {Firefox},
	description = {Webbrowser von der \gls{moz}},
	first = {Mozilla Firefox},
	type = ignored
}

\newglossaryentry{ff-os}{%S
	name = {Firefox\,OS},
	description = {Ein Smartphone- und Tablet-Betriebsystem der Mozilla Corporation},
	type = technologies
}

\newglossaryentry{ie}{
	name = {Internet Explorer},
	description = {Webbrowser von \gls{ms}},
	first = {\gls{ms} Internet Explorer},
	type = ignored
}

\newglossaryentry{chrome}{
	name = {Chrome},
	description = {Webbrowser von \gls{google}},
	first = {Google Chrome},
	type = ignored
}

\newglossaryentry{ms}{
	name = {Microsoft},
	description = {},
	type = ignored
}

\newglossaryentry{apple}{
	name = {Apple},
	description = {Software- und Computer-Hersteller},
	type = ignored
}

\newglossaryentry{google}{
	name = {Google},
	description = {},
	type = ignored
}

\newglossaryentry{moz}{
	name = {Mozilla},
	first = {Mozilla Foundation},
	description = {},
	type = ignored
}

\newglossaryentry{adobe}{
	name = {Adobe},
	first = {Adobe Systems},
	description = {},
	type = ignored
}

\newglossaryentry{nitobi}{
	name = {Nitobi},
	description = {},
	type = ignored
}

\newglossaryentry{apache}{
	name = {Apache},
	first = {Apache Software Foundation},
	description = {},
	type = ignored
}

\newglossaryentry{jqm}{
	name = {jQuery\,\-Mobile},
	description = {},
	type = technologies
}

\newglossaryentry{jq}{
	name = {\mbox{jQuery}},
	description = {},
	type = technologies
}

\newglossaryentry{ko}{
	name = {\mbox{Knockout}},
	description = {},
	type = technologies
}

\newglossaryentry{tag}{
	name = {Tag},
	description = {Ein Element von Auszeichnungssprachen. In \gls{xml}-Dialekten wie bspw. \gls{html} erkennbar an der Tyntax mit spitzen Klammern. (Bsp.: \texttt{<head>})},
}

\newglossaryentry{widget}{
	name = {Widget},
	description = {},
}

\newglossaryentry{laf}{
	name = {Look-And-Feel},
	description = {Aussehen und Verhalten einer Benutzeroberfl\"ache},
}

\newglossaryentry{jq-foundation}{
	name = {\gls{jq}\,Foundation},
	description = {Aussehen und Verhalten einer Benutzeroberfl\"ache},
	type = ignored
}

\newglossaryentry{data-binding}{
	name = {Data-Binding},
	description = {Verbindung von \gls{ui}-Komponenten mit Datenfeldern auf Programmebene},
}

\newglossaryentry{obs}{%S
	name = {Observable},
	description = {Ein zu beobachtendes Objekt (engl. \emph{to observe} = \emph{beobachten, überwachen}), dessen Zustandsänderungen an eine Menge von \glspl{observer} übermittelt werden können.},
}

\newglossaryentry{asynchron}{%S
	name = {asynchron},
	description = {In diesem Kontext: ein Prinzip der Datenübertragung, bei dem eine Antwort auf eine Datenanfrage nicht \emph{synchron}, also unmittelbar nach stellen der Anfrage (bspw. als Rückgabewert) eintrifft, sondern als Ergebnis zurückgesendet wird, sobald dieses vorliegt (\seeref{sec:contacts})}, 	%TODO Beschreibung hinzufügen!
%	type = ignored
}

\newglossaryentry{array}{%S
	name = {Array},
	description = {Eine Datenstruktur},
	type = ignored
}

\newglossaryentry{opensource}{
	name = {Open-Source},
	description = {},
	type = ignored
}

\newglossaryentry{api}{
	name = {API},
	description = {Application Programming Interface (dt. \enquote{Programmierschnittstelle})},
	first = API (Programmierschnittstelle),
	firstplural = APIs (Programmierschnittstellen),
	type = \acronymtype
}

\newglossaryentry{url}{
	name = {URL},
	description = {Uniform Resource Locator (dt. \enquote{einheitlicher Quellenanzeiger}) steht für die Angabe eines Ortes in Netzwerken},
	type = \acronymtype
}

\newglossaryentry{bugtracker}{
	name = {Bugtracker},
	description = {Software zur Fehler- und Problemverwaltung einer Software},
}

