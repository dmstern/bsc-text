
\newglossaryentry{eclipse}{
	name=Eclipse, 
	description={Eine u.a. durch Plugins stark anpassbare Entwicklungsumgebung von der \textsl{Eclipse Foundation}},
	type = technologies
}

\newglossaryentry{app}{
	name = App,
	description = {Kurzform f\"ur engl. \enquote{Application} (dt. \enquote{Anwendung}) im Sinne von Anwendungssoftware. Im deutschsprachigen Raum meist im Zusammenhang mit Smartphones oder Tablet-Computern},
	first = {App (Kurzform f\"ur engl. \enquote{Application})},
	firstplural = {Apps (Kurzform f\"ur engl. \enquote{Application})}
}

\newglossaryentry{web-app}{
	name = Web-App,
	description = {},
	first = {Web-App (oder dt. \enquote{Web-Anwendung})},
	firstplural = {Web-Apps (oder dt. \enquote{Web-Anwendungen})}
}

\newglossaryentry{hybrid-app}{
	name = Hybrid-App,
	description = {},
}

\newglossaryentry{html}{
	name = HTML,
	description = {HTML (Hyper Text Markup Language) ist eine Auszeichnungssprache f\"ur Websites. \enquote{HTML is the publishing language of the World Wide Web} \cite{w3c_html}},
	type = technologies
}

\newglossaryentry{html5}{
	name = HTML5,
	description = {Version 5 der \gls{html}-Spezifikation. Unter HTML5 werden im Allgemeinen Web-Technologien wie \gls{html}, \gls{css} und \gls{js} zusammengefasst, die es Entwicklern erm\"oglichen, Web-Anwendungen ohne die Notwendigkeit von zus\"atzlichen Technologien wie Browserplugins etc. zu entwickeln},
	type = technologies
}

\newglossaryentry{sdk}{
	name = SDK,
	description = {Software Development Kit},
	type = \acronymtype,
}

\newglossaryentry{css}{
	name = CSS,
	description = {Cascading Style Sheets},
	type = technologies,
}

\newglossaryentry{dom}{
	name = DOM,
	description = {Document Object Model},
	type = \acronymtype,
}

\newglossaryentry{ide}{
	name = Entwicklungsumgebung,
	description = {Eine Entwicklungsumgebung (IDE, f\"ur engl. \enquote{Integrated Development Environment}) ist eine Software, mit der Computer-Programme entwickelt werden k\"onnen.},
}

\newglossaryentry{gui}{
	name = GUI,
	description = {Graphical User Interface},
	type = \acronymtype,
}

\newglossaryentry{asp}{
	name = ASP,
	description = {Active Server Pages},
	type = ignored
}

\newglossaryentry{gps}{
	name = GPS,
	description = {Global Positioning System},
	type = \acronymtype,
}

\newglossaryentry{php}{
	name = PHP,
	description = {PHP: Hypertext Preprocessor},
	type = ignored,
}

\newglossaryentry{w3c}{
	name = W3C,
	description = {World Wide Web Consortium},
	type = \acronymtype
}

\newglossaryentry{android}{
	name = Android,
	description = {Smartphone- und Tablet-Betriebssystem von \gls{google}},
	type = technologies
}

\newglossaryentry{ios}{
	name = iOS,
	description = {Smartphone- und Tablet-Betriebssytem von \gls{apple}},
	type = technologies
}

\newglossaryentry{win-phone}{
	name = Windows Phone,
	first = Microsoft~Windows~Phone,
	description = {Smartphone-Betriebssytem von Microsoft},
	type = technologies
}

\newglossaryentry{blackberry-os}{
	name = Blackberry OS,
	description = {Smartphone-Betriebssystem f\"ur \gls{blackberry}-Ger\"ate der Firma \gls{blackberry-inc}},
	type = technologies
}

\newglossaryentry{blackberry}{
	name = Blackberry,
	description = {Smartphone-Reihe der Firma \gls{blackberry-inc}},
	type = ignored
}

\newglossaryentry{blackberry-inc}{
	name = Blackberry,
	description = {Smartphone-Hersteller des gleichnamigen Smartphones \gls{blackberry}},
	type = ignored
}

\newglossaryentry{java}{
	name = Java,
	description = {Eine plattformunabh\"angige, objektorientierte Programmiersprache},
	type = ignored
}

\newglossaryentry{netbeans}{
	name = Netbeans IDE,
	description = {Eine \gls{ide} auf \gls{java}-Basis von der \textsl{Oracle Corpration}},
	type = technologies
}

\newglossaryentry{intellij}{
	name = IntelliJ IDEA,
	description = {Eine \gls{ide} f\"ur \gls{java} der Firma JetBrains},
	type = technologies
}

\newglossaryentry{android-studio}{
	name = {Android Studio},
	description = {Eine \gls{ide} f\"ur die Entwicklung von \glspl{app} f\"ur das Betriebssytem \gls{android}},
	type = technologies
}

\newglossaryentry{obj-c}{
	name = ObjectiveC,
	description = {Eine um objektorientierte Elemente erweiterte Variante der Programmiersprache \gls{c}},
	type = technologies
}

\newglossaryentry{c}{
	name = C,
	description = {Eine imperative Programmiersprache},
	type = technologies
}

\newglossaryentry{xcode}{
	name = Xcode,
	description = {Eine \gls{ide} f\"ur die Entwicklung von \gls{ios}- und \gls{osx}-Software},
	type = technologies
}

\newglossaryentry{osx}{
	name = Mac\,OS\,X,
	description = {Betriebsystem f\"ur Mac-Rechner der Firma \gls{apple}},
	type = technologies
}

\newglossaryentry{gmail}{
	name = Gmail,
	description = {},
	type = ignored
}

\newglossaryentry{gmx}{
	name = GMX,
	description = {},
	type = ignored
}

\newglossaryentry{web-de}{
	name = Web.de,
	description = {},
	type = ignored
}

\newglossaryentry{js}{
	name = {\mbox{JavaScript}},
	description = {Eine Skriptsprache, die in einer gr\"o\"seren Umgebung (wie beispielsweise einem Browser) ausgef\"uhrt wird und in diesem Kontext z.\,B. verwendet werden kann, um \gls{html} um dynamische Elemente, wie der Ver\"anderung von Inhalten, aber auch Ausf\"uhrung von Prozeduren etc., zu erweitern \cite{w3c_js}},
	type = technologies
}

\newglossaryentry{webstorage}{
	name = WebStorage,
	description = {WebStorage (auch: \enquote{LocalStorage} ist eine Spezifikation des \gls{w3c} zur lokalen Speicherung von gr\"o\"seren Datenmengen einer Webanwendung auf Client-Seite in Form von Schl\"ussel-Wert-Paaren.) \cite{w3c_webstorage}},
	type = technologies
}

\newglossaryentry{indexed-db}{
	name = IndexedDB,
	description = {Eine Datenbank...}, %TODO ERKL\"ARUNG EINF\"UGEN!
	type = technologies
}

\newglossaryentry{websql}{
	name = WebSQL,
	description = {}, %TODO ERKL\"ARUNG EINF\"UGEN!
	type = technologies
}

\newglossaryentry{phonegap}{
	name = PhoneGap,
	description = {}, %TODO ERKL\"ARUNG EINF\"UGEN!
	type = technologies
}

\newglossaryentry{cordova}{
	name = Cordova,
	description = {}, %TODO ERKL\"ARUNG EINF\"UGEN!
	type = technologies
}

\newglossaryentry{pg-build}{
	name = {PhoneGap Build},
	description = {PhoneGap~Build ist ein}, %TODO ERKL\"ARUNG EINF\"UGEN!
	type = technologies
}

\newglossaryentry{safari}{
	name = {Safari},
	description = {Standard-Webbrowser von \gls{apple} \gls{osx}},
	type = ignored
}

\newglossaryentry{firefox}{
	name = {Firefox},
	description = {Webbrowser von der \gls{moz}},
	first = {Mozilla Firefox},
	type = ignored
}

\newglossaryentry{ie}{
	name = {Internet Explorer},
	description = {Webbrowser von \gls{ms}},
	first = {\gls{ms} Internet Explorer},
	type = ignored
}

\newglossaryentry{chrome}{
	name = {Chrome},
	description = {Webbrowser von \gls{google}},
	first = {Google Chrome},
	type = ignored
}

\newglossaryentry{ms}{
	name = {Microsoft},
	description = {},
	type = ignored
}

\newglossaryentry{apple}{
	name = {Apple},
	description = {Software- und Computer-Hersteller},
	type = ignored
}

\newglossaryentry{google}{
	name = {Google},
	description = {},
	type = ignored
}

\newglossaryentry{moz}{
	name = {Mozilla Foundation},
	description = {},
	type = ignored
}

\newglossaryentry{adobe}{
	name = {Adobe},
	description = {},
	type = ignored
}

\newglossaryentry{jqm}{
	name = {\mbox{JQueryMobile}},
	description = {},
	type = technologies
}

\newglossaryentry{jq}{
	name = {\mbox{JQuery}},
	description = {},
	type = technologies
}

\newglossaryentry{ko}{
	name = {Knockout},
	description = {},
	type = technologies
}

\newglossaryentry{tag}{
	name = {Tag},
	description = {Ein Element von Auszeichnungssprachen wie bspw. \gls{html}. In XML-Dialekten erkennbar an der Tyntax mit spitzen Klammern: \texttt{<head>}},
}



