
\chapter{Einleitung} \label{sec:einleitung}

\section{Motivation und Aufgabenstellung}

%TODO Ausformulieren:
%TODO Besseren Einleitungssatz finden!
Im Laufe der letzten Jahre haben sich auf dem Markt für mobile Endgeräte wie Tablet-Computer oder Smartphones durch viele konkurrierende Gerätehersteller eine Vielzahl von unterschiedlichen Gerätetypen und diversen mobilen Betriebssystemen herausgebildet.
Durch diesem wachsenden Markt wird hier für viele Anbieter mobiler Anwendungen immer wichtiger, einen ihre Apps für verschiedene Betriebssysteme anzubieten, um möglichst viele Nutzer unterschiedlicher Gerätesoftware zu erreichen. %TODO Quelle??

Viele Webseiten-Betreiber bieten ihre Online-Dienste vermehrt auch als mobile App an und müssen, um einen breiten Nutzer- und Kundenstamm zu gewinnen, diese gleich für verschiedene mobile Betriebssysteme anbieten.
Aufgrund der verschiedenen Technologien und Programmiersprachen der einzelnen Plattformen bedeutet dies jedoch im Normalfall einen hohen Mehraufwand, sodass viele Apps zum großen Teil lediglich Versionen für die beiden größten mobilen Plattformen \gls{ios} oder \gls{android} angeboten werden und dadurch ein Teil der Nutzer von anderen Systemen wie \gls{win-phone}, \gls{blackberry} oder kleinere wie \gls{tizen} oder \gls{ubuntu-phone} nicht erreicht werden können. 
Für viele Firmen und Anwendungsentwickler bleibt da oft allenfalls die Erstellung einer mobilen Website, mit der zwar grundsätzlich alle mobilen Plattformen internetfähiger Mobilgeräte erreicht werden können, der Mehrwert einer installierten App wie die Nutzung verschiedener Geräte-Features oder die leichtere Kundenbindung durch die feste Platzierung von Icons in der App-Übersicht der Nutzers, die mit der Installation einer mobilen App einhergeht, jedoch ausbleibt.

Da viele Frameworks und Lösungen für die Entwicklung von plattformunabhängigen Apps standardisierte Webtechnologien wie HTML5 in Kombination mit Javascript etc. verwenden, kann eine solche Web-App mit entsprechenden Hilfsmitteln unter Umständen relativ leicht als mobile App portiert werden, sodass gerade für Anwendungsentwickler, die schon gewisse Vorkenntnisse im Bereich der Web-Entwicklung mitbringen, mit relativ geringem personellen wie finanziellen Aufwand ihre Software oder ihren Online-Dienst auf mehrere mobile Plattformen bringen können.
Die breite Vielfalt an entsprechenden Technologien deutet auf einen regen Markt für die Cross-Plattform-Entwicklung hin und zeugt von der Aktualität des Themas.

In dieser Arbeit soll ein kurzer Überblick über verschiedene Ansätze gegeben und anhand von theoretischen Erläuterungen sowie mit der Implementierung eines Praxis-Beispiels mithilfe einer ausgewählte Technologie Grenzen und Möglichkeiten für die plattformunabhängige App-Entwicklung aufgezeigt werden.


%TODO Ausformulieren:
% Literatur und Quellenlage
%	Viele Quellen (Webseiten) verschiedener Lösungen und Produkte zur plattformunabhängige App-Entwicklung, teilweise ähnliche Ansätze, die ein mehr oder weniger detailliertes Bild der Ansätze und Vorteile für Entwickler zeichnen. 
	Als primäre Quelle wurden hier die einschlägigen Online-Dokumentationen der aufgezeigten und verwendeten Technologien sowie weitere Online-Quellen verwendet. 
	Zwar lässt sich auch vereinzelt Printliteratur finden, auf diese wird hier jedoch aufgrund des praktischen Charakters der Arbeit nicht näher eingegangen, da gerade in diesen modernen und schnelllebigen Technologien die wichtigsten und aktuellsten Informationen aus den aktuellen Online-Quellen wie Hersteller-Seiten und Dokumentationen zu beziehen sind.
	
\section{Ziel und Aufbau der Arbeit}

%TODO Ausformulieren:

Der Hauptteil der vorliegenden Arbeit gliedert sich grundlegend in zwei Teile.

Im ersten Teil \namerefH{sec:grundlagen} soll ein Überblick über verschiedene Arten von mobilen Apps und deren prinzipielle Entwicklung gegeben werden. 
		Dazu wird zunächst auf den üblichen Weg, mobile Apps für das jeweilige Betriebssystem zu entwickeln, eingegangen sowie Schwierigkeiten, die bei besonderen Anforderungen, insbesondere der Bedienung mehrerer Zielplattformen, auftreten können, aufgezeigt.
		Mit Web- und Hybrid-Apps werden zwei weitere Mechanismen für die mobile App-Entwicklung beschrieben, deren grundlegende Herangehensweise und Technologien dargelegt sowie Vor- und Nachteile erörtert werden.
	Im \namerefH{Überblick über Möglichkeiten der plattformunabhängigen App-Entwicklung} soll ein Überblick über verschiedene konkrete Lösungen und Produkte wie Frameworks, Code-Generatoren und deren Funktions- und Herangehensweise gegeben werden.
	Mit Cordova, Knockout, etc. Erläuterung der Grundlagen der für die Implementierung erwähnten und im Praxis-Teil erwähnten Technologien
	Im \fullref{sec:cordova}, der hier den hier elementaren Teil darstellt, da diese Technologie die Grundlage für die Umsetzung plattformspezifischer Anforderungen im Praxis-Teil bildet, soll ein Einblick in die Verwendung eines konkreten Ansatzes zur plattformunabhängigen App-Entwicklung mithilfe von sogenannten \glspl{hybrid-app} gegeben und grundlegende Möglichkeiten des Frameworks aufgezeigt werden werden.
	Grundlegende Funktionsweise und Verwendung des Frameworks und der Geräte-Schnittstellen.
	Sowie einige weitere Möglichkeiten, die die Software bietet.

%TODO Ausformulieren:
Zweiter Teil: Praktische Umsetzung
	Beschreibung der Implementierung einer Beispiel-Anwendung mithilfe des Cordova-Frameworks, um Möglichkeiten des Frameworks zu erproben und Besonderheiten in dessen Verwendung aufzuzeigen.
	Hier soll zunächst die grobe Konzeption und Idee der zu implementierenden Beispiel-App beschrieben, die Herangehensweise und Struktur anhand von verwendeten Technologien und verwendeten Entwurfsmustern erläutert sowie Kriterien für die Bewertung der verwendeten Cross-Platform-Technologie aufgestellt werden.
	In der Beschreibung der Implementierung der Geräteschnittstelle soll mithilfe der Verwendung eines eines Cordova-Plugins die Umsetzung eines Anwendungsfalls der vorher spezifizierten Beispiel-Anwendung, für den die Interaktion mit einer speziellen Funktionalität des Betriebssystems erforderlich ist, beschrieben werden.
	Dabei soll insbesondere auf die Möglichkeiten eingegangen werden, die die (jeweilige) API für die Interaktion zwischen Anwendung und Betriebssystem bietet, sowie Grenzen und Besonderheiten in der Verwendung der implementiert Schnittstelle aufgezeigt werden.
	Abschließend soll im Fazit unter anderem anhand der vorher festgelegten Kriterien sowie aus weiteren Informationen aus der jeweiligen Dokumentation eine Auswertung der explorierten Technologie vorgenommen werden und eine Abwägung über deren Vor- und Nachteile stattfinden.


Im letzten Teil \namerefH{sec:schluss} sollen die erarbeiteten Inhalte aus dem Überblick sowie der praktischen Umsetzung und deren hervorstechende Merkmale und Besonderheiten aufgezeigt werden sowie diskutiert werden, wo die Grenzen und Möglichkeiten von aktuellen Lösungen und Ansätzen liegen und inwieweit diese einen Mehrwert für Unternehmen und Entwickler verschiedener Motivation bietet.
Außerdem soll ein Ausblick gegeben werden, wo die Untersuchung des Themas und Erprobung verschiedener Technologien erweitert werden könnte.

