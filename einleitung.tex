%TODO Analog zur unnummerierten Part-/Kapitel-Überschrift in Resümee hier evtl. auch so?
\chapter{Einleitung}

%\section{Ziel der Arbeit}

\section{Motivation und Aufgabenstellung} % (Aktualität, Interesse, Nutzen)
\begin{comment}
Bei der Entwicklung mobiler Anwendungen steht - besonders im Consumer-Bereich - neben vielen anderen Entscheidungen die Wahl der Ziel-Plattform an, wobei ein wichtiger Faktor sicher die Erreichbarkeit einer möglichst großen Nutzer-Anzahl darstellt.
Vermutlich um den  Entwicklungsaufwand nicht ins Bodenlose \todo{Stil!} ausufern zu lassen beschränken sich dabei viele App-Hersteller (oder -anbieter) auf die größten, meist genutzten Plattformen wie Android oder iOS. \todo{darf man so eine Behauptung jetzt einfach machen? oder wie weißt man sowas nach?}

Doch nicht nur, dass dadurch doppelter Wartungs- und Anpassungsaufwand für die Entwicklung entsteht, auch werden dadurch viele andere Betriebssysteme vernachlässigt, was letztlich zu Lasten der Nutzer geht, die auf einem vielfältigen aber auch diffundierten Markt - je nach Hersteller und Plattform - ein zum Teil eingeschränktes und ungleich verteiltes Angebot an Anwendungen vorfinden. 

Auf der anderen Seite hat der Bereich der Webentwicklung und -gestaltung in den letzten Jahren seit Aufkommen von Smartphones und Tablets eine neue Anforderung hinzu bekommen: Websites müssen nicht mehr nur für die unterschiedlichsten Browser auf dem Desktop angepasst werden, sondern sollen sich auch auf Geräten, die für Touch-Bedienung ausgelegt sind unterschiedlicher Bildschirmgrößen gleich gut anfühlen und bedienen lassen. Da der Trend für viele Firmen in Sachen Öffentlichkeitsarbeit, Kundenbindung und -freundlichkeit neben der Firmen-Website und der eigenen Facebook-Seite auch eine eigene App zu fordern scheint, liegt vor dem zuvor genannten Hintergrund der stark variierenden App-Formate der Ansatz nahe, auch den Webbrowser als eine weitere Plattform im bunten Gefüge aus Deployment-Anforderungen zu sehen, die nach Verminderung und Zusammenführung des Entwicklungsaufwands im mobilen Bereich verlangt. 

Letzterer Ansatz ist allerdings nur einer, den es zu Untersuchen gilt; Zentraler Forschungsgegenstand soll die Exploration der Möglichkeiten und Grenzen der plattformunabhängigen App-Entwicklung anhand eines beispielhaft implementierten Anwendungsfalls sein.
\end{comment}

\section{Ziel und Aufbau der Arbeit} % → NICHT Chronologisches Vorgehen!}
