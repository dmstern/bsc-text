\documentclass[bachelor,german]{hgbthesis}

% PACKAGES =================================================

% ToDo-Notizen:
\usepackage[
	textwidth=3.7cm,		% Breite der Notizen
	disable					% Nicht drucken 
	]	{todonotes}	

% Übersetzungen ändern:
\usepackage[
	ngerman					% Neue deutsche Rechtschreibung
	]	{translator}

% Ein Glossar verwenden
\usepackage[
	nonumberlist,			% keine Seitenzahlen anzeigen
%	acronym,				% ein Abkürzungsverzeichnis erstellen
	toc,					% Einträge im Inhaltsverzeichnis
	section					% im Inhaltsverzeichnis auf section-Ebene erscheinen
	]	{glossaries}

% Abstände einstellen:
\usepackage[
	onehalfspacing			% Zeilen abstand ca. 1.25%
	]	{setspace}

% Fußnotenformatierung:
\usepackage[
	bottom,
	hang
	]	{footmisc}

% Zitat-Formatierungen:
\usepackage[
	font={
		itshape,			% kursiv
		singlespacing,		% Abstand: Einzeilig
		raggedright			% linksbündig (rechtsflatternd)
	}]	{quoting}

% METADATA =================================================

\title{Plattformunabhängige App-Entwicklung für mobile Geräte \\--\\ Grenzen und Möglichkeiten}
\author{Daniel Morgenstern}

\studiengang{Medieninformatik}
\hochschule{Beuth Hochschule für Technik Berlin}
\fachbereich{Informatik und Medien}
\studienort{Berlin}
\abgabedatum{2014}{04}{22}
\betreuerin{Prof. Dr. Simone Strippgen}
\semester{Sommersemester 2014}

% DEFINITIONS ==============================================

% Eigenes Symbolverzeichnis erstellen
\newglossary[techlg]{technologies}{tchi}{tcho}{Technologien}
\newglossary[ignlg]{ignored}{igni}{igno}{Ausgeblendetes Glossar}

% Ressourcen ----------------------------------------------
% COMMANDS =================================================

\newcommand{\usecase}[1]{\subsection*{\large#1}}
\newcommand{\actors}[1]{\item[Akteure] \hfill \\ #1}
\newcommand{\initcondition}[1]{\item[Anfangsbedingung] \hfill \\ #1}
\newenvironment{verlauf}
	{
		\item[Beschreibung] \hfill
		\begin{enumerate}
	}{
		\end{enumerate}
	}
	
\newenvironment{altVerlauf}
	{
		\item[Alternative Verläufe] \hfill
		\begin{enumerate}
	}{
		\end{enumerate}
	}

\newcommand{\namerefH}[1]{\textit{\nameref{#1}}}
\newcommand{\extend}[1]{extend \namerefH{#1}}
	
% Unnnumerierte Part-Überschrift
\newcommand{\unnumberedPart}[1]{
	\phantomsection\part*{#1}
	\addcontentsline{toc}{part}{#1}
}

% Unnnumerierte Kapitel-Überschrift
\newcommand{\unnumberedChapter}[1]{
	\phantomsection\chapter*{#1}
	\addcontentsline{toc}{section}{#1}
}

% Querverweise:
\newcommand{\linename}{Zeile~}
\newcommand{\linenamepl}{Zeilen~}
%TODO Als Option für neues Kommando "seelst" oder so. -> Siehe Seiten-Option bei Zitationen.

\newcommand{\seeref}[1]{\seename\ \autoref{#1}}
\newcommand{\seepageref}[1]{\seename\ \autopageref{#1}}
\newcommand{\seeautopageref}[1]{\seename\ \autoref{#1}, \autopageref{#1}}

\newcommand*{\fullpageref}[1]{\hyperref[{#1}]{\autoref*{#1} auf \autopageref{#1}}}
\newcommand*{\fullref}[1]{\hyperref[{#1}]{\autoref*{#1} \textit{\nameref*{#1}}}}
\newcommand{\seefullref}[1]{\seename\ \fullref{#1}}
\newcommand{\seefullpageref}[1]{\seefullref{#1},\ \autopageref{#1}}

% Style für Dateinamen:
\newcommand{\filename}[1]{\textsf{#1}}

% Variable für Pfad zu Code-Listings:
\newcommand{\pathtolistings}{codelistings}
\newcommand{\listingpath}[1]{
	\renewcommand{\pathtolistings}{#1}
}

% Variable für Pfad zu Bsp-App-Code:
\newcommand{\pathtoappcode}{app}
\newcommand{\apppath}[1]{
	\renewcommand{\pathtoappcode}{#1}
}

% Seitenränder ---------------------------------------------

% Default Margin: siehe Master-Datei.
\newcommand{\bigcodemargin}{3cm}
\newcommand{\textbodymargin}{4cm}

% Image-Layout ---------------------------------------------

\newcommand{\fullimagesize}{1\linewidth}
\newcommand{\halfimagesize}{.5\linewidth}
\newcommand{\screenshotRes}{140}

% Bildunterschrift für Quellenangabe:
\newcommand{\imagesourcespace}{-1.5em}
\newcommand{\imagesourcefont}{\captionsetup{font={footnotesize,it}}}
\newcommand{\imagesourcename}{Quelle}
\newcommand{\imagesourcelabel}{\imagesourcename:\ }

\newcommand{\ownScreenshot}{Eigener Screenshot.}
\newcommand{\ownGraphic}{Eigene Grafik.}

% Neuer Befehl für Bilder:
\newcommand{\image}[6][.]{
	\begin{figure}[!h]
	\centering
	\includegraphics[
		#3,								% 	#3:	Größe: entweder 
										% 			'width=1\textwidth,' = gesamte Breite 
										% 			'resolution=\screenshotRes' = Orig.-gr.
		natwidth=1,						% 		Parameter für latexmk
		natheight=1						% 		Parameter für latexmk
		]{#1/#2}							%	#1/#2: Unterordner / Label und Dateiname
		\caption
			[#4]						%	#4: Kurzbeschriftung (im Abbildungsverzeichnis)
			{#5}						%	#5: Beschriftung (am Bild)
		\label{fig:#2}					% 		Label: 'fig:' + 'DATEINAME'
		\vspace{\imagesourcespace}		% 		Abstand zur Quellenangabe
		\imagesourcefont{}				% 		Schrift f. Quellenangabe
		\caption*{\imagesourcelabel #6}	%	#6: Quellenangabe
			%TODO Quelle: Das müsste doch mit citefield[organization] oder so gehen.
	\end{figure}
}

\newcommand{\doubleimage}[9][.]{

	\begin{figure}[!h]
	\centering
	\begin{subfigure}{.45\textwidth}
	  \centering
	  \includegraphics[width=1\linewidth]{#1/#3}
	  \caption{#5}
	  \label{fig:#3}
	\end{subfigure}%
	\begin{subfigure}{.45\textwidth}
	  \centering
	  \includegraphics[width=1\linewidth]{#1/#4}
	  \caption{#6}
	  \label{fig:#4}
	\end{subfigure}
		\caption
			[#7]						%	#7: Kurzbeschriftung (im Abbildungsverzeichnis)
			{#8}						%	#8: Beschriftung (am Bild)
		\label{fig:#2}					% 		Label: 'fig:' + 'DATEINAME'
		\vspace{\imagesourcespace}		% 		Abstand zur Quellenangabe
		\imagesourcefont{}				% 		Schrift f. Quellenangabe
		\caption*{\imagesourcelabel #9}	%	#9: Quellenangabe
	\end{figure}
}

% Layout ===================================================

% Layout für Code-Listings ---------------------------------
% (Wird weiter unten referenziert und weiter verfeinert)
\newenvironment{code}{
	\begin{singlespace}							% Einfacher Zeilenabstand
	}{
	\end{singlespace}
}

% Layout für kleine Code-Listings auf einer Seite ----------
% (Wird weiter unten referenziert und weiter verfeinert)
\newenvironment{minicode}{
	\par\noindent\begin{minipage}{\linewidth}	% Code auf einer Seite zusammenhalten
	\begin{code}
	}{
	\end{code}
	\end{minipage}\par\addvspace{\topskip}
}

% Neuer Befehl für include-Code-Listings:
% (Wird weiter unten referenziert und weiter verfeinert)
\newcommand{\includecode}[3]{
	\begin{code}
		\lstinputlisting[#1,#2]{\pathtolistings#3} % #1: Weitere Optionen / #2: Code-Datei
	\end{code}
}

% Neuer Befehl für include-Code-Listings:
% (Wird weiter unten referenziert und weiter verfeinert)
\newcommand{\includeminicode}[3]{
	\begin{minicode}
		\lstinputlisting[#1,#2]{\pathtolistings#3} % #1: Weitere Optionen / #2: Code-Datei
	\end{minicode}
}

% Neuer Befehl für HTML-Inludes:
\newcommand{\includehtml}[2]{
	\includeminicode{style=htmlcssjs}{#2}{#1}
}

% Neuer Befehl für HTML-Inludes:
\newcommand{\includebightml}[2]{
	\includecode{style=htmlcssjs}{#2}{#1}
}

% Neuer Befehl für Bash-Inlcudes:
\newcommand{\includebash}[2]{
	\includeminicode{
		style=shellscript,
		label=lst:#1
	}{#2}{#1}
}

%-----------------------------------------------------------

% Hurenkinder- und Schusterjungen-Regel:
\widowpenalties 3 10000 10000 100
\clubpenalties 3 10000 10000 100
\displaywidowpenalty=10000

% Fußnoten auf einer Seite behalten:
\interfootnotelinepenalty10000

% Abstand von Ziffer zu Fußnotentext:
\setlength{\footnotemargin}{1.2em}

%-----------------------------------------------------------

% Style für Glossar-Begriffe:
\defglsdisplayfirst[\glsdefaulttype]{\textit{#1}}
\defglsdisplayfirst[technologies]{\textit{#1}}
\defglsdisplayfirst[ignored]{\textit{#1}}
%\defglsdisplay[ignored]{{#1}}

%TODO Zusammengesetzte Begriffe wie GUI-Toolkit oder UI-Logik etc. NICHT Kursiv!

% COLORS ===================================================

% Farben für Hyperlinks ------------------------------------
\definecolor{@darkblue}{RGB}{0,0,0}
\definecolor{@darkgreen}{RGB}{0,0,0}

% Code-Farben ----------------------------------------------
% EIGENE
\definecolor{mygreen}{HTML}{93A83C}
\definecolor{xdarkgray}{HTML}{202020}
\definecolor{lila}{HTML}{891A44}
\definecolor{xlightgray}{HTML}{FCFCFC}
\definecolor{mycyan}{HTML}{056A70}
\definecolor{darkcyan}{HTML}{1B353B}
\definecolor{cyan-black}{HTML}{0D2025}

% AUS VORLAGE
\definecolor{lightgray}{rgb}{0.9, 0.9, 0.9}
\definecolor{darkgray}{rgb}{0.4, 0.4, 0.4}
%\definecolor{purple}{rgb}{0.65, 0.12, 0.82}
\definecolor{editorGray}{rgb}{0.95, 0.95, 0.95}
\definecolor{editorOcher}{rgb}{1, 0.5, 0} % #FF7F00 -> rgb(239, 169, 0)
\definecolor{editorGreen}{HTML}{619A0E} % #007C00 -> rgb(0, 124, 0)
\definecolor{orange}{rgb}{1,0.45,0.13}		
\definecolor{olive}{rgb}{0.17,0.59,0.20}
\definecolor{brown}{rgb}{0.69,0.31,0.31}
\definecolor{purple}{rgb}{0.38,0.18,0.81}
\definecolor{lightblue}{rgb}{0.1,0.57,0.7}
\definecolor{lightred}{rgb}{1,0.4,0.5}

\lstdefinestyle{htmlcssjs} {%
	% Code
	language=HTML5,
	alsolanguage=JavaScript,
	alsodigit={.:;},	
	tabsize=2,
	showtabs=false,
	showspaces=false,
	showstringspaces=false,
	extendedchars=true,
	breaklines=true,
	%
	% Code design
	identifierstyle=\color{xdarkgray},
	keywordstyle=\color{lila}\bfseries,
	ndkeywordstyle=\color{editorGreen}\bfseries,
	stringstyle=\color{mycyan}\ttfamily,
	commentstyle=\color{mygreen}\ttfamily,
	%
	% German umlauts
	literate=%
	{Ö}{{\"O}}1
	{Ä}{{\"A}}1
	{Ü}{{\"U}}1
	{ß}{{\ss}}1
	{ü}{{\"u}}1
	{ä}{{\"a}}1
	{ö}{{\"o}}1
}

\lstdefinestyle{shellscript} {%
	% Code
	language=bash,
%	alsolanguage=,
	alsodigit={.:;},	
%	otherkeywords={"cordova ", create, add, remove, ls, list, search, rm, run, build},
%	ndkeywords={platform, platforms, plugin, plugins},
	tabsize=2,
	showtabs=false,
	showspaces=false,
	showstringspaces=false,
	extendedchars=true,
	breaklines=true,
	%
	% Code design
	identifierstyle=\color{xdarkgray},
	keywordstyle=\color{lila}\bfseries,
	ndkeywordstyle=\color{editorGreen}\bfseries,
	stringstyle=\color{mycyan}\ttfamily,
	commentstyle=\color{mygreen}\ttfamily,
	%
	% German umlauts
	literate=%
	{Ö}{{\"O}}1
	{Ä}{{\"A}}1
	{Ü}{{\"U}}1
	{ß}{{\ss}}1
	{ü}{{\"u}}1
	{ä}{{\"a}}1
	{ö}{{\"o}}1
}

% Listing Settings =========================================
%
\lstset{
 	caption=\lstname,
	escapeinside={\%*}{*\%},
%	captionpos=b,
	%
	% General Listings-Design
	backgroundcolor=\color{xlightgray},
	basicstyle={\footnotesize\ttfamily},   
	numberstyle=\tiny\color{darkgray},
	frame=single,
	rulecolor=\color{lightgray},
	%
	% Line Numbers
	xleftmargin={0.75cm},
	numbers=left,
	stepnumber=1,
	firstnumber=1,
	numberfirstline=true,	
}


% Quelle: https://www.writelatex.com/examples/listings-code-style-for-html5-css-html-javascript/htstpdbpnpmt#.U0lFQlRJWR9

%TODO TITLE etc. nur dann keyword, wenn HTML-Tag (<>).
%TODO Spitze Klammer zu wird teilweise nicht gehighlightet.
%TODO Evtl. mit includekeyword, morekeywords, excludekeyword oder so?

% CSS
\lstdefinelanguage{CSS}{
  keywords={color,background-image:,margin,padding,font,weight,display,position,top,left,right,bottom,list,style,border,size,white,space,min,width, transition:, transform:, transition-property, transition-duration, transition-timing-function},	
  sensitive=true,
  morecomment=[l]{//},
  morecomment=[s]{/*}{*/},
  morestring=[b]',
  morestring=[b]",
  alsoletter={:},
  alsodigit={-}
}

% JavaScript
\lstdefinelanguage{JavaScript}{
  morekeywords={typeof, new, true, false, catch, function, return, null, catch, switch, var, if, in, while, do, else, case, break},
  morecomment=[s]{/*}{*/},
  morecomment=[l]//,
  morestring=[b]",
  morestring=[b]'
}

\lstdefinelanguage{HTML5}{
  language=html,
  sensitive=true,	
  alsoletter={<>=-},	
  morecomment=[s]{<!-}{-->},
  tag=[s],
  otherkeywords={
  % General
  >,
  % Standard tags
	<!DOCTYPE,
  </html, <html, <head, <title, </title, <style, </style, <link, </head, <meta, />,
	% body
	</body, <body,
	% Divs
	</div, <div, </div>, 
	% Paragraphs
	</p, <p, </p>,
	% scripts
	</script, <script,
  % More tags...
  <canvas, /canvas>, <svg, <rect, <animateTransform, </rect>, </svg>, <video, <source, <iframe, </iframe>, </video>, <image, </image>, <header, </header, <article, </article, <h1, </h1, <h2, </h2, <h3, </h3, <h4, </h4, <h5, </h5, <h6, </h6, <button, </button
  },
  ndkeywords={
  % General
  =,
  % HTML attributes
  charset=, src=, id=, width=, height=, style=, type=, rel=, href=,
  % SVG attributes
  fill=, attributeName=, begin=, dur=, from=, to=, poster=, controls=, x=, y=, repeatCount=, xlink:href=,
  % properties
  margin:, padding:, background-image:, border:, top:, left:, position:, width:, height:, margin-top:, margin-bottom:, font-size:, line-height:,
	% CSS3 properties
  transform:, -moz-transform:, -webkit-transform:,
  animation:, -webkit-animation:,
  transition:,  transition-duration:, transition-property:, transition-timing-function:,
  }
}

\graphicspath{{images/}}
\listingpath{code/}
\bibliography{literatur}
\loadglsentries{glossar}
%-----------------------------------------------------------

\makeglossaries

%===========================================================

\begin{document}

%-----------------------------------------------------------
\frontmatter		% Anfangsteil (Römische Seitenzahlen)
\maketitle
\tableofcontents
%-----------------------------------------------------------

%-----------------------------------------------------------
\mainmatter			% Hauptteil (ab hier arab. Seitenzahlen)
%-----------------------------------------------------------
   
\chapter{Einleitung}

\section{Ziel der Arbeit}

\section{Motivation} % (Aktualität, Interesse, Nutzen)
\begin{comment}
Bei der Entwicklung mobiler Anwendungen steht - besonders im Consumer-Bereich - neben vielen anderen Entscheidungen die Wahl der Ziel-Plattform an, wobei ein wichtiger Faktor sicher die Erreichbarkeit einer möglichst großen Nutzer-Anzahl darstellt.
Vermutlich um den  Entwicklungsaufwand nicht ins Bodenlose \todo{Stil!} ausufern zu lassen beschränken sich dabei viele App-Hersteller (oder -anbieter) auf die größten, meist genutzten Plattformen wie Android oder iOS. \todo{darf man so eine Behauptung jetzt einfach machen? oder wie weißt man sowas nach?}

Doch nicht nur, dass dadurch doppelter Wartungs- und Anpassungsaufwand für die Entwicklung entsteht, auch werden dadurch viele andere Betriebssysteme vernachlässigt, was letztlich zu Lasten der Nutzer geht, die auf einem vielfältigen aber auch diffundierten Markt - je nach Hersteller und Plattform - ein zum Teil eingeschränktes und ungleich verteiltes Angebot an Anwendungen vorfinden. 

Auf der anderen Seite hat der Bereich der Webentwicklung und -gestaltung in den letzten Jahren seit Aufkommen von Smartphones und Tablets eine neue Anforderung hinzu bekommen: Websites müssen nicht mehr nur für die unterschiedlichsten Browser auf dem Desktop angepasst werden, sondern sollen sich auch auf Geräten, die für Touch-Bedienung ausgelegt sind unterschiedlicher Bildschirmgrößen gleich gut anfühlen und bedienen lassen. Da der Trend für viele Firmen in Sachen Öffentlichkeitsarbeit, Kundenbindung und -freundlichkeit neben der Firmen-Website und der eigenen Facebook-Seite auch eine eigene App zu fordern scheint, liegt vor dem zuvor genannten Hintergrund der stark variierenden App-Formate der Ansatz nahe, auch den Webbrowser als eine weitere Plattform im bunten Gefüge aus Deployment-Anforderungen zu sehen, die nach Verminderung und Zusammenführung des Entwicklungsaufwands im mobilen Bereich verlangt. 

Letzterer Ansatz ist allerdings nur einer, den es zu Untersuchen gilt; Zentraler Forschungsgegenstand soll die Exploration der Möglichkeiten und Grenzen der plattformunabhängigen App-Entwicklung anhand eines beispielhaft implementierten Anwendungsfalls sein.
\end{comment}

\section{Aufbau der Arbeit} % → NICHT Chronologisches Vorgehen!}

\part{Theoretische Grundlagen}
%1. Theoretische Grundlagen → als Grundlage für Implementierung
% Was muss der Leser wissen, um die Realisierung zu verstehen? 
%1. Konkreter, technologischer: Worum geht es bei Plattformunabhängigkeit, wie ist zu erreichen? → verschiedene grundsätzliche Konzepte
%2. Übersicht / Eigenschaften & Einordnung der vorgestellten L\"osungen
%3. Verwendete Technologien → wie macht das phonegap?
%1. Knockout, jquerymobile, etc.

\chapter{Apps für mobile Geräte}

\section{Mobile (native) Apps} \label{sec:native}
Unter mobilen \glspl{app} versteht man im Allgemeinen Anwendungssoftware für Tablet-Computer oder Smartphones. 
Im Laufe der letzten Jahre haben sich auf dem Markt für Mobilgeräte durch viele konkurrierende Gerätehersteller eine Vielzahl von Smartphone- und Tablet-Betriebssystemen herausgebildet.
Im Entwicklungsbereich wird in dem Zusammenhang auch von \glspl{plattform} gesprochen.

Zu den \glspl{plattform} mit dem höchsten Marktanteil zählen \glspl{google} Betriebssystem \gls{android}, \gls{ios} von \gls{apple}, \gls{win-phone} und \gls{blackberry-os} des gleichnamigen Smartphone-Herstellers \gls{blackberry-inc} \cite{platforms-marketshare}.
% Nativ-Entwicklung für die jeweiligen Plattformen: Android / iOS
Die \gls{app}-Entwicklung für diese mobilen Betriebssysteme erfolgt mehr oder weniger ähnlich und soll im Folgenden, um auf die beiden größten Vertreter einzugehen, anhand von \gls{android} beziehungsweise \gls{ios} näher beschrieben werden.

%	-> \gls{sdk}s verwalten
Grundsätzlich müssen auf der \gls{ide} die entsprechenden \glspl{sdk} der \gls{plattform}, für die entwickelt wird, installiert sein. 
Diese enthalten Softwarekomponenten, die zur Entwicklung der \gls{app} notwendig sind, beispielsweise Klassen, die es einem erlauben, auf native Funktionalitäten des Betriebssystems wie zum Beispiel das Adressbuch, den Benachrichtigungsmechanismus oder auch auf Hardwarekomponenten wie die Kamera, den Bewegungssensor oder das \gls{gps}-Modul zuzugreifen sowie die entsprechenden plattformspezifischen Oberflächenkomponenten des jeweiligen \gls{gui}-Toolkits zu nutzen.

%	-> Code schreiben für die jeweilige Plattform
% Allgemein / Android
Als Programmiersprache für die \gls{android}-\gls{app}-Entwicklung wird \gls{java} verwendet. Das heißt, als Voraussetzung für die Entwicklung von \gls{android}-\glspl{app} ist lediglich eine geeignete \gls{ide} wie \gls{eclipse}, \gls{netbeans} oder \gls{intellij} sowie eine Installation des \gls{java}- und des \gls{android}-\gls{sdk} nötig. 
\todo{wie sieht es mit dem deployment, also der Auslieferung in appstores etc. aus?}
Seit 2013 bietet Google darüber hinaus die auf \gls{intellij} basierende und eigens für die \gls{android}-Entwicklung angepasste \gls{ide} \gls{android-studio} an \cite{android-studio}, die bereits alle notwendigen Toolkits enthält. 
Nachdem der Code geschrieben ist, kann er kompiliert und zu einem lauffähigen Programm \emph{gebaut} (engl. \enquote{build}) %TODO ins Glossar, allerdings eigentlich mit Beugung und so
 werden (\seeref{fig:hybrid-apps-schaubild}). Anschließend kann die \gls{app} in dem für die Zielplattform vorgesehenen Dateiformat ausgeliefert und auf dem Zielgerät installiert werden.

% iOS: 
Auch der Software- und Computer-Hersteller Apple bietet mit \gls{xcode} eine firmeneigene \gls{ide} zur \gls{app}-Entwicklung für sein mobiles Betriebssystem \gls{ios} an. 
Anders als Google geht der iPhone-Hersteller hier allerdings etwas restriktiver vor. 
So läuft die \gls{ide} \gls{xcode}, die man für die native \gls{ios}-Entwicklung benötigt, nur unter dem hauseigenen Betriebssystem \gls{osx} und das wiederum nur auf den firmeneigenen Mac-Rechnern \cite{Cordova_Documentation_iOS_Platform_Guide}.
Darüber hinaus ist für \gls{ios}-Entwickler die Teilnahme am \gls{ios-dev-prog} erforderlich, um \glspl{app} für \gls{apple}-Geräte auszuliefern und auf Geräten installieren zu können, wofür der Konzern einen jährlichen Mitgliedsbeitrag von 99\,\$ im Jahr verlangt \cite{iOS_Developer_Program}.
So sichert sich Apple, nicht nur durch die kostenpflichtige Mitgliedschaft im \gls{ios-dev-prog}, sondern allein schon durch deren exklusive Platformunterstützung ihres eigenen Mobilbetriebssystems, auch mit jedem Entwickler einen neuen Kunden.\footnote{Diese Restriktion fällt allerdings auch in der unten beschriebenen plattformunabhängigen App-Entwicklung nicht unbedingt weg (\seeref{sec:hybrid-dev}).} \todo{darf man so was hier anmerken, oder lieber weg?}

Ansonsten verläuft der Entwicklungsprozess bei der \gls{ios}-Entwicklung im Prinzip ähnlich zur \gls{android}-Entwicklung (\seeref{fig:hybrid-apps-schaubild}).
Als Programmiersprache wird \gls{obj-c} verwendet, einer um objektorientierte Elemente erweiterte Variante der Programmiersprache \gls{c}.

Möchte ein Auftraggeber einer Software also statt seinen Kunden nur eine \gls{app} für ein Betriebssystem anzubieten, einen größeren Nutzerkreis erschließen, muss die zu entwickelnde \gls{app} für jede Zielplattform neu programmiert, getestet und gebaut werden, da jede mobile \gls{plattform} ihre eigenen Toolkits, Bibliotheken und Programmiersprachen verwendet, was die native \gls{app}-Entwicklung für potenzielle Auftraggeber zu einem sehr kostenaufwändigen Projekt werden lassen kann.
Andererseits bietet die native \gls{app}-Entwicklung vollständige Unterstützung der betriebssystemeigenen Funktionalitäten wie den Zugriff auf Kamera, Adressbuch, Bewegungssensoren etc. der jeweiligen \gls{plattform}, sodass ein Softwareprojekt mit solchen besonders hardware- oder betriebssystemnahen Anforderungen die Entwicklung einer nativen (plattformspezifischen) \gls{app} notwendig erscheinen lassen kann.\footnote{Mehr dazu in \autoref{sec:hybrid-dev}}

\section{Web-Anwendungen}\label{sec:web-app}

% Zuerst gab es Websites, dann dynamische Websites
Eine \gls{web-app} ist eine Anwendungssoftware, die auf einem Web-Server läuft und auf die der Nutzer mittels eines Browsers zugreifen kann; also eine dynamische Website, wie man sie auch schon vor dem Aufkommen von Smartphones und modernen Tablets kannte. 

Die Grundlage für die Entwicklung von Internetseiten bildet der langjährige Standard \gls{html}, mit dem deren Aussehen, Inhalt und Struktur textuell beschrieben werden kann. 
In Kombination mit \gls{css} für die modulare Gestaltung einer Website sowie \gls{js}, einer Skriptsprache zur \gls{dom}-Manipulation, bietet die \gls{html}-Spezifikation in ihrer neusten Version \textit{(\gls{html5})} im Grunde alles, was für die Entwicklung einer modernen grafischen Benutzerschnittstelle notwendig ist. 
Die Fachlogik liegt, neben den Oberflächen-Komponenten in Form von \mbox{\gls{html}-,} \gls{css}- und Javascript-Dokumenten, auf einem Webserver und verarbeitet und reagiert auf Anfragen des Clients.\footnote{Die Rolle des Clients übernimmt hier also der Browser.}
Als Server-Technologie ist ein breites Spektrum an Programmiersprachen und Umgebungen einsetzbar.\footnote{Einige sind beispielsweise \gls*{php}, \gls*{java}, \gls*{asp} u.\,v.\,a.\,m.}

Somit bietet die Entwicklung einer \gls{web-app} (abgesehen von einigen Browser-spezifischen Eigenheiten) bereits eine gewisse Plattformunabhängigkeit, da jedes moderne Betriebssystem über einen Webbrowser verfügt. 
Zwar müssen Entwickler in bestimmten Details bei der Erstellung des Codes auf die teilweise unterschiedliche Unterstützung (beispielsweise von \gls{html}-Elementen) \todo{genauer?} durch die verschiedenen Browser achten, aber darüber hinaus wird der Entwicklungsaufwand für eine \gls{web-app} nicht von der Anzahl der Zielplattformen bestimmt, da von Client-Seite aus verschiedene Browser durch die Verbreitung und Beachtung von Web-Standards weitgehend einheitliche \gls{html}-Dokumente lesen und interpretieren können und das Back\-end nicht auf Clients mit unterschiedlichen \glspl{plattform}, sondern auf Webservern liegt, deren \gls{plattform} bei der Entwicklung entweder schon bekannt oder nicht relevant ist.\footnote{Beispielsweise weil auch die Fachlogik plattformunabhängig mit \gls*{php} oder \gls*{java} realisiert wurde.}

% Dann für sämtliche Internet-Dienste auch noch eine \gls{app}
Obwohl es, durch damals eher im Business-Bereich verortete Internet-Handys und Palmtops, auch vor den heute üblichen mobilen Touch-Geräten bereits mobile Internetseiten gab, die speziell für die Darstellung auf kleinen Displays ausgerichtet waren, boten mit der massenhaften Verbreitung von mobilen, internetfähigen Geräten und deren (im Folgenden erläuterten) stark anwendungsorientierten Bedien-Konzepten viele herkömmliche Internet-Dienste nun auch zusätzlich eine native \gls{app} für verschiedene mobile \glspl{plattform} an.
So sind beispielsweise auch E-Mail-Dienste wie \gls{gmx}, \gls{web-de} oder \gls{gmail} seit der Verbreitung von Smartphones und Tablets auch in Form einer eigenen \gls{app} für \gls{android} und \gls{ios} vertreten, sodass der Nutzer, statt, wie von der Desktop-Computer-Nutzung gewohnt, einen anbieterunabhängigen Mail-Client zu konfigurieren, über den er seine E-Mails abruft, unter Umständen gleich die jeweilige \gls{app} des E-Mail-Anbieters startet \cite{gmx, web.de, gmail}.
Das heißt, der Nutzer folgt einem geänderten Bedienungsmuster seines Mobilgeräts gegenüber der herkömmlichen Computer-Nutzung: um zu einem bestimmten Ergebnis zu gelangen (beispielsweise \emph{Nachrichten lesen}) also die Frage zu beantworten, \emph{wie} er dahin gelangt (Einen Browser öffnen und zur gewünschten Seite navigieren: www.tagesschau.de), ist es für Anwender heutiger Mobilsysteme naheliegend, gleich die passende \gls{app} zu starten (Hier \zB die Tagesschau-\gls{app}).

% Gründe für App statt Web-Anwendung
Dafür gibt es verschiedene mögliche Gründe. Zum Einen muss im Gegensatz zu einer Website bei der mobilen \gls{app} nicht die komplette Oberfläche\footnote{\gls{html}-, \gls{css}- und JavaScript-Dokumente sowie Grafiken} übertragen werden, sondern lediglich die Nutzdaten,\footnote{Also beispielsweise, um beim obigen Beispiel zu bleiben, die Nachrichten in Textform.} was dem Nutzer ein höheres Maß an Performanz einbringt.
Zum Anderen können trotz Vollbildmodus in bestimmten Fällen \gls{gui}-Elemente des Webbrowsers bei der Benutzung einer Web-Anwendung störend sein, so ist beispielsweise die Adresszeile am Rand nicht unbedingt erwünscht, wenn der Nutzer statt im Internet zu surfen dort eigentlich eine bestimmte Anwendung nutzen möchte. 
Ein anderes Beispiel für ein eventuell unerwünschtes Verhalten der Benutzerschnittstelle ist das der \emph{Menü}-Taste bei \gls{android}-Geräten, die im Falle der Nutzung einer Web-Anwendung über den Browser nicht den Kontext der eigentlich benutzten Anwendung anzeigt,\footnote{hier also der Website} sondern lediglich den des Browsers.

In bestimmten Fällen kann eine nützliche Funktion einer \gls{app} die Offline-Nutzung sein, wenn beispielsweise durch die abgedeckten Anwendungsfälle keine Verbindung oder Synchronisation mit einem Server nötig ist. Beispiele hierfür könnten, um nur einige zu nennen, ein Taschenrechner, kleine Spiele, oder eine Bildverarbeitungs-\gls{app} sein. 
Für diese Offline-Nutzung einer \gls{app} zeichnet die Web-Anwendung ein geteiltes Bild: Zwar wurden in den letzten Jahren mehrere Methoden entwickelt, eine Web-Anwendung auch offline nutzen zu können, doch durch ihre Ausrichtung auf die Nutzung via Internet stellt die Implementierung dieser Funktionalität für Entwickler einen Zusatzaufwand dar. 

Einige Möglichkeiten, eine Web-Anwendung ohne Internetverbindung nutzbar zu machen, sind beispielsweise die aus der \gls{html5}-Spezifikation hervorgehenden Technologien \gls{webstorage}, ein Mechanismus zum lokalen Speichern von größeren Datenmengen in Form von Schlüssel-Wert-Paaren \cite{w3c_webstorage} sowie \gls{websql} bzw. \gls{indexed-db}, beides auf Web-Anwendungen optimierte Datenbanken-Spezifikationen, die vom \gls{w3c} herausgegeben werden \cite{w3c_websql, w3c_indexedDB}.
Allerdings bestehen auch bei diesen Mechanismen teilweise Einschränkungen durch die Browservielfalt beziehungsweise deren Versionen. So wird \gls{indexed-db} beispielsweise nicht von \gls{safari} oder \gls{ios} unterstützt, \gls{chrome} muss für die Nutzung mindestens in Version 23 oder höher vorliegen, \gls{firefox} in 10 oder höher. 
Auch bei \gls{websql} zeichnet sich ein ähnlich diffuses Bild ab: Während \gls{chrome} die Technologie ab der Version 4 und \gls{ios} ab Version 3.2 unterstützt, ist für Nutzer der Browser \gls{firefox} und \gls{ie} die Technik gar nicht verfügbar.
Lediglich \gls{webstorage} wird weitgehend von allen gängigen Browsern unterstützt \cite{html5-rocks_offline}.
Weiterhin wird die Offline-Funktionalität gegenüber der nativen \gls{app} dadurch eingeschränkt, dass der Nutzer diese ohne weiteres Zutun des Entwicklers nur dann nutzen kann, wenn die entsprechende Internet-Seite im Offline-Zustand des Geräts bereits im Browser geöffnet ist, da diese nicht lokal auf dem Gerät, sondern auf einem Webserver gespeichert ist.
Für vollständigen Offline-Zugriff müsste der Entwickler die komplette Website so paketieren, dass der Nutzer sie -- wie eine native App -- von seinem Gerät aus starten und nutzen kann.\footnote{\seeref{sec:hybrid-app} und \ref{sec:hybrid-dev}.}

Allgemein kann man sagen, dass der Zugriff auf native Funktionalitäten des Geräts respektive des Betriebssystems nicht oder nur gering unterstützt wird, sodass der geringere Entwicklungsaufwand einer solchen \gls{web-app} (\seename~ \autoref{fig:hybrid-apps-schaubild}) unter Umständen zu Lasten des Funktionsumfangs und der Usability der Anwendung geht.
\todo{recherchieren: wird was unterstützt? gibt es Möglichkeiten, per Javascript etc.? Was wird \enquote{gering} unterstützt? sollte man das noch weiter ausführen? Touch-Gesten etc.?}

\section{Hybride Apps} \label{sec:hybrid-app}

Die \gls{hybrid-app} verbindet Eigenschaften der Nutzung einer nativen \gls{app} mit den Vorteilen der Web-Entwicklung mithilfe von Web-Technologien und entsprechenden \glspl{framework} und löst damit beispielsweise das Problem der mangelnden Offline-Fähigkeit einer \gls{web-app} sowie deren geringe Unterstützung von plattformspezifischen oder hardware-nahen Funktionalitäten. 
Da jedes moderne mobile Betriebssystem für Entwickler auch die Möglichkeit bietet, eine Web-View in die zu entwickelnde App einzubinden, also eine \gls{gui}-Komponente, in die \gls{html}-Inhalte hinein geladen werden können, liegt der Ansatz für hybride Apps auf der Hand: Auf Entwicklungsebene wird die Anwendung als Web-App entwickelt, gleichzeitig mithilfe von entsprechenden \glspl{api} und \glspl{framework} zur Anbindung an die native Ebene der Zielplattform in eine App für die jeweiligen \glspl{plattform} integriert, sodass auf Benutzerseite die Nutzung einer Web-Anwendung, die in puncto Funktionsumfang, Usability und Look-And-Feel einer nativen mobilen App sehr nahe kommt, möglich wird.

Dieses Vorgehen bietet unter anderem für Web-Entwickler den Vorteil, ihre bisherigen Programmierkenntnisse im Web-Bereich im Wesentlichen auch für die Entwicklung von hybriden \glspl{app} nutzen zu können. 
So wird in der Regel der grundlegende Teil des Codes für das Frontend, wie bei der Web-Entwicklung, mit \gls{html} in Kombination mit \gls{css} und \gls{js} geschrieben und getestet. 
Da allerdings auf dem mobilen Gerät für das Backend, also die Verarbeitungsinstanzen, nicht, wie bei einer herkömmlichen Web-Anwendung, ein Server mit einer entsprechenden Server-Technologie wie \gls{php} oder \gls{asp} läuft, wird auch dieser Teil der App bei der hybriden Entwicklung meist mit \gls{js} bewerkstelligt. 
\todo{Recherchieren! Kann man Perl oder Python oder so auf einem Android laufen lassen?}
Anschließend muss die Anwendung für die verschiedenen Zielplattformen gebaut werden, um in das jeweilige Container-Format für Apps der verschiedenen \glspl{plattform} eingebunden werden zu können und den Zugriff auf die plattformspezifischen Toolkits durch die Cross-Platform-\glspl{api} zu ermöglichen (\autoref{fig:hybrid-apps-schaubild}).
Hierfür kann es erforderlich sein, dass auf der Entwicklungsplattform die jeweiligen \glspl{sdk} installiert sind, was gegenüber der Web-Entwicklung einen administrativen Mehraufwand darstellt.
Eine andere Variante ist die Auslagerung des Bauprozesses auf einen externen Build-Server, beispielsweise mithilfe eines externen Web-Service eines Drittanbieters, was den Vorteil hat, die \glspl{sdk} für die Zielplattformen nicht auf jedem Entwicklungsrechner verwalten zu müssen. 
Allerdings hat der Entwickler durch die Herausgabe des Codes an einen solchen Dienstleister nicht mehr die vollständige Kontrolle über den Code, sodass die Variante der Auslagerung des Build-Prozesses gerade für Closed-Source-Projekte unter Umständen nicht in Frage kommt. 
Des Weiteren kann der Betreiber des Build-Services unter Umständen Restriktionen bezüglich der Plattformunterstützung erteilen, wodurch eventuell eine geringere Anzahl von Zielplattformen unterstützt wird, als von Entwicklerseite gewünscht oder erfordert.\footnote{Beispielsweise unterstützt \gls{pg-build} in der neusten Version 3 nur noch die drei großen Mobilplattformen \gls{android}, \gls{ios}, und \gls{win-phone}.}

\image{hybrid-apps-schaubild}
	{width=\fullimagesize}
	{Schaubild Hybrid Apps}
		{Entwicklungsstufen der verschiedenen Arten von Apps. Während bei der nativen App der gesamte Entwicklungszyklus einmal pro \gls{plattform} durchlaufen werden muss, verringert sich der Aufwand für die \gls{web-app} erheblich. Bei der \gls{hybrid-app} muss die Anwendung zwar einmal für jede \gls{plattform} gebaut und ausgeliefert werden, um die Schnittstellen für die nativen \glspl{plattform} zu implementieren, aber der hauptsächliche Entwicklungsaufwand des Programmierens und Testens fällt aufgrund des generischen Charakters nur einmal an.}
	{Eigene Grafik.}

\chapter{Plattformunabhängige App-Entwicklung}

\section[Überblick: Lösungen und Ansätze]{Überblick: Lösungen und Ansätze für die platt\-form\-un\-ab\-hängige App-Entwicklung}
%TODO Ergebnisse aus GoogleDocs-Recherche Zusammentragen

%TODO Gibts es andere Ansätze als den hybriden?

%TODO Lösungen beschreiben und Ansätze kurz erläutern
%TODO Titanium
%TODO PhoneGap
%TODO Appcelarator
%TODO Weitere Lösungen?

\section{Entwicklung von hybriden Apps}	\label{sec:hybrid-dev}
%TODO boxed?

% Dieses Kapitel dient dazu, den Leser über die Grundlagen der im Praxis-Teil verwendeten Technologien zu setzten. Da ich im Praxis-Teil nur Cordova ausprobiere, hier der Zusatz "mit Cordova" in der Übeschrift?

Wie in \ref{sec:hybrid-app} beschrieben, bildet die hybride App-Entwicklung die Schnittmenge aus der nativen App-Entwicklung und der Web-Entwicklung mithilfe von Web-Technologien und zusätzlichen \glspl{framework} und Bibliotheken. 
Hier soll mit \gls{cordova}\,/\,\gls{phonegap} die konkrete Nutzung eines dieser \glspl{framework} und weitere verwendete Technologien wie \gls{ko} oder \gls{jqm} erläutert werden.

\subsection{Die JavaScript-Bibliothek jQuery}

Bereits für die Entwicklung von reinen Web-Anwendungen stellen neben den grundlegenden Web-Technologien \gls{html}, \gls{css} und \gls{js} Erweiterungen wie die JavaScript-Bibliothek \gls{jq} nützliche Hilfsmittel dar, die viele Funktionen gegenüber der Verwendung von \enquote{reinem} \gls{js} deutlich vereinfacht.
So ist beispielsweise der Programmieraufwand für den Zugriff auf Elemente einer \gls{html}-Seite durch \gls{jq} wesentlich geringer als ohne die Bibliothek.
Besonders deutlich wird dies an den unten aufgeführten Code-Beispielen, in denen ein Button exemplarisch die Funktionalität übernehmen soll, alle Absätze einer \gls{html}-Seite auszublenden.
Während bei herkömmlichem \gls{js} für die Selektion aller Elemente die Funktion \lstinline|getElementsByTagName()| aufgerufen werden muss (\seeref{lst:js}), ist bei \gls{jq} der Zugriff auf alle Elemente eines \glspl{tag} per Dollar(\lstinline|$|)-Notation deutlich verkürzt (\seeref{lst:jq}).
Auch die nächste Anweisung zur Ausführung einer Operation für alle ausgewählten Elemente (hier: \lstinline|hide()|, also \textit{ausblenden}) fällt bei \gls{jq} wesentlich kürzer aus, indem die Funktion noch in der selben Zeile wie der vorherigen Selektor aufgerufen werden kann (\lstinline|$("p").hide();|).
Bei der reinen \gls{js}-Variante ist nach der Selektion aller Absätze zunächst einmal ein Array ausgewählt, sodass, um auf den einzelnen Elementen Operationen ausführen zu können, durch alle Elemente des Arrays iteriert und die Funktion für jedes Element aufgerufen werden muss (\seeref{lst:js}, \linenamepl 5\,-\,7).

\includehtml{jquery.html}{
	label=lst:jq,
		caption=\gls{jq}-Beispiel-Code zum Ausblenden aller Absätze \cite{w3schools_jq_hide}.,
	}

\includehtml{without-jquery.html}{
	label=lst:js,
		caption=Die gleiche Funktionalität wie in \autoref{lst:jq} mit reinem \gls{js}.,
	}
	

\subsection{Data-Binding mit Knockout} \label{sec:ko}
Neben \gls{jq} bietet das \gls{ko}-\gls{framework} ein weiteres nützliches Hilfsmittel für die Entwicklung von Web-Anwendungen, das ebenso wie \gls{jq} und \gls{jqm} aus einer \gls{js}-Bibliothek besteht und die Verbindung der \gls{html}-Oberfläche mit der Programmlogik der Anwendung mittels \gls{data-binding}, also der dynamischen Anbindung von \gls{ui}-Komponenten zu Datenfeldern auf Programmebene, erheblich vereinfacht.
Das \gls{data-binding} wird bei \gls{ko} durch das \gls{mvvm} realisiert, das Entwicklern eine Trennung zwischen Benutzeroberfläche und \gls{ui}-Logik ermöglicht.
Diese Aufteilung dient unter anderem der Übersichtlichkeit des Codes und kann beispielsweise die Aufteilung der Entwicklung von Benutzerschnittstellen erleichtern, indem die \gls{ui}- und die Geschäftslogik von Softwareentwicklern übernommen werden kann, während Designer den Schwerpunkt auf die Gestaltung der Oberfläche legen können \cite{Model_View_ViewModel__Wikipedia}.

\image{mvvm-pattern}
	{width=1\textwidth}
	{MVVM-Pattern}
		{Schematische Darstellung des Entwurfsmusters MVVM: Die View ist über das \gls{data-binding} mit dem ViewModel verbunden, indem die UI-Logik implementiert ist und das mit der Geschäftslogik (Model) interagiert.}
	{Wikimedia Commons \cite{MVVMPattern}.}

Das \gls{mvvm} stellt eine Gliederung der Software in drei Grundlegende Komponenten dar:
Die \gls{view} repräsentiert die Präsentationsschicht, also die Benutzeroberfläche, im Falle der Web-Anwendung also die \gls{html}-Seite, deren Elemente per \gls{data-binding} an Eigenschaften des \glspl{view-model} gebunden werden können. 
Das \gls{model} steht für die Geschäftslogik und beinhaltet das Datenmodell und die Funktionen, die vom \gls{view-model} angefragt werden können, um beispielsweise Benutzereingaben zu validieren oder Daten für die Anzeige in der Oberfläche zu erhalten (\seeref{fig:mvvm-pattern}).

Ein wesentlicher Mechanismus für die Aktualisierung der Oberfläche bei einer Änderung des \glspl{view-model} von \gls{ko} ist die Verwendung von \glspl{obs}, also Objekten oder Datenfeldern, welche bei Änderungen ihres Inhalts eine Nachricht aussenden, sodass andere Objekte automatisch auf die Zustandsänderung des \glspl{obs} reagieren können, beispielsweise, um die Anzeige auf der Oberfläche zu aktualisieren.

Im Code-Beispiel unten (\autoref{lst:ko}) wird ein einfaches \gls{view-model} mit drei Eigenschaften erstellt: \lstinline|firstName|, \lstinline|lastName|, und \lstinline|fullName|, wobei die ersten beiden \emph{\glspl{obs}} darstellen und letztere aus den anderen beiden Feldern generiert wird (\linenamepl 10\,-\,13).
Durch den Aufruf der \gls{ko}-Funktion \lstinline|observable()| ist es nicht notwendig, bei einer Änderung der Daten an der Oberfläche, die Änderung der Anzeige der Daten (\linename 27) explizit anzustoßen (Beispielsweise per EventListener auf einer \gls{ui}-Komponente).
Stattdessen übernimmt das \gls{ko}-\gls{framework} die Durchreichung aller Änderungen im \gls{view-model}, sodass bei einer Benutzereingabe in eines der \lstinline|<input>|-Felder eine Änderung der Daten im \gls{view-model} registriert wird und automatisch alle damit verbundenen \glspl{view} aktualisiert werden (\seeref{fig:ko-hello-world}).\footnote{Hier das \texttt{<h2>}-Element, das mit der \gls{view-model}-Eigenschaft \texttt{fullName} verknüpft ist.}

\includehtml{ko.html}{
	label=lst:ko,
		caption=Einfaches Anwendungsbeispiel für die Verwendung der \gls{js}-Bibliothek \gls{ko}.,
	}

\image{ko-hello-world}
	{width=1\textwidth}
	{Knockout-Beispiel im Browser}
		{Knockout-Beispiel aus \autoref{lst:ko} im Browser. Links im Bild: Anzeige bei Initialisierung der Oberfläche, rechts: Benutzereingabe ins Eingabefeld: \enquote{Mars}. \\ Änderungen in der UI (Hier: im Eingabefeld) werden sofort im \gls{view-model} registriert und automatisch an alle verknüpften Anzeigen weitergereicht (Hier an das fettgedruckte Begrüßungselement).}
	{Eigener Screenshot.}

\subsection{JQuery\,Mobile: Mobile Web-Oberflächen}

%TODO Hier Evtl. auch nochmal Bezug nehmen auf Überblick: Was gibts für GUI-Libs?

Um das Erscheinungsbild und Verhalten von Webseiten an eine bessere Benutzung für mobile Geräte anzupassen, bietet sich der Einsatz eines entsprechenden \gls{gui}-Toolkits an. 
Die von der \gls{jq-foundation} entwickelte \gls{gui}-Bibliothek \gls{jqm} bietet hier Möglichkeiten für Entwickler von mobilen Webseiten, ihre Dokumente an verschiedene Eigenschaften anzupassen, die gemeinhin unter dem Begriff \gls{laf} zusammengefasst werden, wie dem äußeren Erscheinungsbild, der Fähigkeit, mit Touch-Gesten umzugehen, der Anpassung an die geringere Display-Größe sowie von vielen mobilen \glspl{app} gewohnten Animationen, und somit erwartungskonform zu gestalten.

\Gls{jqm} besteht mit einer \gls{js}-Bibliothek und einem zusätzlichen Stylesheet in \gls{css}, aus zwei Dokumenten, deren Einbindung in die \gls{html}-Seite analog zu der von \gls{jq} per Verlinkung als \lstinline|<script>|- beziehungsweise \lstinline|<link>|-\gls{tag} funktioniert.
Da \gls{jqm} auf \gls{jq} aufbaut, muss auch der Link zum \gls{jq}-Script gesetzt sein, um auf die benötigten Funktionen zugreifen zu können (\seeref{lst:jqm}, \linenamepl 3\,-\,5).

\includehtml{jquerymobile.html}{
	label=lst:jqm,
		caption=Einbindung von \gls{jqm} in eine \gls{html}-Seite \cite{w3schools_jqm_start}.,
	}
	
Die Definition von \gls{gui}-Elementen geschieht hierbei über das \gls{html}-Attribut \lstinline|data-role|, dem vordefinierte Werte wie \lstinline|page|, \lstinline|header|, \lstinline|footer|, \lstinline|button| u.\,v.\,a.\,m. zugeteilt werden können, anhand derer das \gls{jqm}-\gls{framework} den \gls{html}-Elementen die jeweilige Style-Definition aus dem Stylesheet zuweisen kann (\seeref{lst:jqm}, \linenamepl 8, 9, 12 und 15).
Somit wird dem Entwickler ermöglicht, ohne zusätzlichen Entwicklungsaufwand für die Programmierung von \gls{gui}-Komponenten oder Erstellung von Style-Definitionen Webseiten mit zeitgemäßem und adäquaten \gls{laf} für mobile Geräte anzupassen (\seeref{fig:jqm} und \ref{fig:without-jqm}).

Neben der Zuweisung von Rollen über das \lstinline|data-role|-Attribut, durch die das \gls{framework} den \gls{gui}-Komponenten automatisch entsprechende Style-Definitionen zuweist, können weiterhin über das \lstinline|class|-Attribut direkt Style-Klassen aus dem \gls{jq}-Stylesheet verwendet werden.
Beispielsweise sorgt im obigen Beispiel der Zusatz \lstinline|class="ui-content"| (\autoref{lst:jqm}, \linename 12) für ein besseres Layout des \lstinline|<div>|-Inhalts, wie der Vergleich ohne das \lstinline|class|-Attribut in \autoref{fig:without-ui-content-class} zeigt.


\image{jqm}
	{width=1\textwidth}
	{jQuery\,\-Mobile-Beispiel}
	{jQuery\,\-Mobile-Beispiel aus \autoref{lst:jqm} im Browser.}
	{Eigener Screenshot.}

\image{without-ui-content-class}
	{width=1\textwidth}
	{Beispiel ohne \lstinline|class|-Attribut}{Beispiel-Oberfläche wie in \autoref{fig:jqm}, ohne das class-Attribut in \autoref{lst:ko}, \linename 12.}
	{Eigener Screenshot.}

\image{without-jqm}
	{width=1\textwidth}
	{Beispiel ohne jQuery\,\-Mobile}{Beispiel-Oberfläche wie in \autoref{fig:jqm}, aber ohne die \gls{jqm}-Bibliotheken.}
	{Eigener Screenshot.}

\subsection{Phonegap\,/\,Cordova} \label{sec:cordova}

% Vorstellung ----------------------------------------------

\subsubsection{Grundlegendes} \label{sec:cordova-grundlegendes}

\gls{phonegap} ist ein \gls{opensource}-\gls{framework} von \gls{adobe} zur Erstellung von hybriden \glspl{app} und bildet damit die Grundlage für den hier explorierten Ansatz zur plattformunabhängigen \gls{app}-Entwicklung.
Die Software wurde ursprünglich unter dem Namen \gls{phonegap} von der Firma \gls{nitobi} entwickelt, die 2011 von \gls{adobe} aufgekauft wurde \cite{Adobe_Announces_Agreement_to_Acquire_Nitobi_Creator_of_PhoneGap}. 
Später wurde die Code-Basis der \gls{apache} übergeben und dort in \gls{cordova} umbenannt, wodurch das \gls{framework} in den Quellen stellenweise unter beiden Namen erscheint.
\Gls{adobe} \gls{phonegap} baut also als dessen Distribution auf \gls{cordova} auf, wobei derzeit der einzige wesentliche Unterschied im Namen des Pakets besteht (Stand: \citedate{PhoneGap_Cordova_and_whats_in_a_name}), nach eigenen Angaben aber durchaus weitere Tools mit Bezug auf andere \gls{adobe}-Dienste in die \gls{phonegap}-Distribution einfließen können \cite{PhoneGap_Cordova_and_whats_in_a_name}.

Da das \gls{apache}-\gls{framework} als \gls{opensource}-Basis auch die allgemeine Grundlage für weitere \gls{cordova}-Distributionen bildet und so auch die Community-Anlaufstelle zur Mitwirkung am \gls{cordova}-Projekt darstellt~\cite{PhoneGap_Cordova_and_whats_in_a_name}, wird im Folgenden weitgehend der Begriff \enquote{\gls{cordova}} verwendet, die meisten Inhalte treffen aber ebenso auf die \gls{adobe}-Version \gls{phonegap} zu. \todo{Oder ist das nicht eigentlich eh klar, da Cordova du \emph{Grundlage} bildet?}
%TODO Nach praktischem Teil nochmal checken!

% Beschreibung / Übersicht: Komponenten --------------------

Wie in \autoref{sec:hybrid-app} beschrieben, wird bei der hybriden App-Entwicklung eine Web-App programmiert, die dann in eine native WebView \enquote{verpackt} werden und somit auf dem jeweiligen Mobilgerät als mobile App ausgeführt werden kann. 
Das \gls{cordova}-\gls{framework} besteht darüber hinaus im Wesentlichen aus einer \gls{api} in Form einer \gls{js}-Bibliothek, die es dem Entwickler ermöglicht, auf native Funktionalitäten des mobilen Betriebssystems zuzugreifen sowie einem mitgelieferten \gls{cli}, das für die Erstellung, Erweiterung und Anpassung der Anwendung, zur Bewerkstelligung des Build-Prozesses für die verschiedenen \glspl{plattform} sowie auch der Ausführung in einem Emulator oder auf einem Mobilgerät zum Testen der Anwendung dient \cite{Cordova-Docs_Overview}.

% Entwicklungsworkflows ------------------------------------

In der Cordova-Dokumentation werden zwei grundlegende Entwicklungsszenarien beschrieben, für die das \gls{framework} verwendet werden kann. 
Mit der Version 3.0 wurde das \gls{cli} Teil des Software-Paktes, das viele Arbeitsschritte automatisiert ausführt und damit den Cordova-Entwicklungsprozess vereinfacht. 
Der Hauptfokus dieses Werkzeugs liegt im für die hybride App-Entwicklung grundlegenden Entwicklungsworkflows, dem \emph{Web-Ent\-wick\-lungs\-an\-satz}.
Dieser Ansatz bietet die breiteste Plattformunterstützung bei möglichst geringem Mehraufwand für unterschiedliche \glspl{plattform} und bildet damit die Grundlage für den hier hauptsächlich fokussierten Ansatz \cite{Cordova-Docs_CLI}.

Soll nur eine bestimmte Zielplattform bedient werden, kann auch nach dem \emph{Nativen-Plattformansatz} entwickelt werden.
Dabei wird das \gls{cli} in erster Linie für die Erstellung des Grundgerüsts der App verwendet, dessen Web-Oberfläche und nativer Kern dann mithilfe einer entsprechenden \gls{ide} in Kombination mit einem SDK der jeweiligen \gls{plattform} weiter verarbeitet und kompiliert werden können. 
Dieser Ansatz kann beispielsweise sinnvoll sein, wenn Entwickler mobiler Apps ihre Kenntnisse im Web-Bereich für die mobile \gls{app}-Entwicklung nutzen möchten und sehr plattformspezifische Eigenschaften angepasst werden sollen, ist aber aufgrund des Mangels an entsprechenden Tools nicht oder nur gering für die Entwicklung plattformunabhängiger \glspl{app} geeignet \cite{Cordova-Docs_CLI}. 

% Plattform- / Feature-Unterstützung -----------------------

\image{cordova-platform-support}
	{width=\fullimagesize}
	{Cordova Plattform- und Feature-Unterstützung}
		{Plattform- und Feature-Unterstützung des \gls{cordova}-\glspl{framework}: \\ 
		Bis auf wenige (\enquote{kleinere}) mobile Betriebssysteme bietet \gls{cordova} auch über die am weitesten verbreiteten \glspl{plattform} wie \gls{android}, \gls{ios} und \gls{win-phone} hinaus die volle Feature-Unterstützung für \gls{amazon-fireos}, \gls{ubuntu-phone} und eine fast vollständige für \gls{blackberry-os}.\\
		Für Entwickler von hybriden \glspl{app} dürfte hier auch vor allem die erste Zeile \emph{cordova CLI} interessant sein, da durch die Kompatibilität der jeweiligen Plattform-\glspl{sdk} nur bestimmte Kombinationen von Entwicklungs- und Zielplattform möglich sind.}
	{Screenshot aus der \gls{cordova}-Dokumentation \cite{Cordova-Docs_Platform-Support}.}

% Voraussetzungen ------------------------------------------

\gls{cordova} stellt für die Verwendung von nativen Funktionalitäten des mobilen Betriebssystems mit seiner \gls{js}-Plattform-\gls{api} eine Verbindung von der Web-Anwendungsebene zu den jeweiligen \glspl{sdk} der Zielplattformen her.
Somit müssen, um auf plattformspezifische Features zugreifen zu können, auf der Entwicklungsplattform alle \glspl{sdk} der gewünschten Zielplattformen installiert sein.
Da die \glspl{sdk} jedoch teilweise nur von bestimmten Desktop-Betriebssystemen unterstützt werden, muss das \gls{cli} unter Umständen auf mehreren Rechnern ausgeführt werden, die jeweils nur bestimmte mobile \glspl{plattform} bedienen (\seeref{fig:cordova-platform-support}).

% Funktionsweise -------------------------------------------
\subsubsection{Funktionsweise}

Nachdem das \gls{cordova}-\gls{framework} auf dem Entwicklungsrechner installiert ist, kann mit dem \gls{cli} ein neues App-Projekt angelegt werden. 
Dazu muss auf der Kommandozeile in das Entwicklungsverzeichnis für das zu erstellende Projekt navigiert und das \gls{cordova}-Tool mit dem \lstinline|create|-Befehl aufgerufen werden (\seeref{lst:cordova-create}).

\includebash{cordova-create}{
	caption={Befehl zum Erstellen einer Cordova-App.},
	}

Das erste Argument \lstinline|hello-beuth| spezifiziert dabei den Namen des Ordners für das zu erstellende App-Projekt, der mit dem \lstinline|create|-Befehl angelegt wird.
Die anderen beiden Parameter sind optional und geben mit der \gls{id} in der rückwärts geschriebenen Domain-Bezeichnung und dem Namen ({\mbox{\enquote{HelloBeuth}}), der später für die App angezeigt wird, weitere Informationen für die App an, die in einer \gls{xml}-Datei im neu angelegten Projekt-Ordner gespeichert werden \cite{Cordova-Docs_CLI}.

Mit der Ausführung dieses Skripts erstellt das Tool in einem neuen Ordner ein Grundgerüst für die \gls{app}, das die nötige Struktur für den weiteren Entwicklungsprozess enthält (\seeref{fig:cordova-directory}).
Auf oberster Ebene liegt die Konfigurationsdatei \filename{config.xml}, in der grundlegende Eigenschaften sowie Informationen für den Build-Prozess hinterlegt werden können (\seeref{lst:config.xml}).
Daneben werden unter anderem (zu diesem Zeitpunkt noch leere) Ordner für \glspl{plugin} und plattformspezifischen Code sowie ein \filename{www}-Ordner angelegt, der das Grundgerüst für den Web-Teil der \gls{hybrid-app} beinhaltet.
Neben der Datei \filename{index.html}, die die Beschreibung der Web-Oberfläche für die Anwendung darstellt, werden nach üblichen Konventionen in der Webentwicklung weiterhin die Unterordner \filename{js}, \filename{css} und \filename{img} angelegt, worin die zur \gls{app} gehörigen \gls{js}- und \gls{css}-Dateien bzw. Bilder gespeichert werden~\cite{Cordova-Docs_CLI}.


\image{cordova-directory}
	{resolution=\screenshotRes}
	{Cordova-Dateistruktur}
		{Dateistruktur einer mit \gls{cordova} erstellten \gls{app}. Das Grundgerüst für die \gls{hybrid-app} wird automatisch angelegt und entspricht den üblichen Web-Entwicklungskonventionen.}
	{Eigener Screenshot.}


\includehtml{hello-beuth/config.xml}{
	label=lst:config.xml,
		caption={Die Konfigurationsdatei \filename{config.xml}, in der allgemeine Informationen über die \gls{hybrid-app} gespeichert werden. In \linename 2 und 4 tauchen die in \autoref{lst:cordova-create} angegebenen optionalen  Parameter \emph{Id} und \emph{Name} wieder auf.},
	}
	

Die Initialisierung der \gls{app} erfolgt über den \filename{deviceready}-Eventhandler, der standardmäßig von \filename{www/js/index.js} referenziert wird (\seeref{lst:index.html}, \linename 21 und \autoref{lst:index.js}, \linenamepl 11\,u.\,12).

Das in \linename 18 referenzierte Script \lstinline|cordova.js| stellt die \og wesentliche \gls{api} zur nativen Betriebssystemebene dar, ist hier im \filename{www}-Ordner jedoch noch nicht vorhanden, sondern wird erst nach Ausführung des \lstinline|build|-Befehls in seiner jeweiligen plattformspezifischen Ausführung in das entsprechende Unterverzeichnis im Ordner \filename{platforms} eingefügt. 

\includehtml{hello-beuth/www/index.html}{
	firstline=20,
	lastline=43,
	label=lst:index.html,
		caption={Startseite der von \gls{cordova} erzeugten \gls{app}, die auf das \lstinline|deviceready|-Event reagiert.},
	}

\includehtml{hello-beuth/www/js/index.js}{
	firstline=19,
	lastline=49,
	label=lst:index.js,
		caption={Standardmäßig von \gls{cordova} angelegte \gls{js}-Datei \filename{index.js}},
	}

Um die Oberfläche der \gls{app} anzuzeigen, lässt sie sich, da sie im Wesentlichen aus einer \gls{html}-Seite besteht, in einem herkömmlichen Browser öffnen (\seeref{fig:cordova-app-browser}).
Da jedoch das bei der hybriden \gls{app} darunterliegende mobile Betriebssystem hier nicht zur Verfügung steht, ist hier auch keine Anbindung an dessen Funktionalitäten möglich.
Dafür muss die Anwendung entweder direkt auf einem mobilen Gerät, dessen \gls{plattform} die \gls{app} und das \gls{cordova}-\gls{api} unterstützen, oder mithilfe eines Emulators ausgeführt werden (\seeref{fig:android_emulate_install}) \cite{Cordova-Docs_CLI}.

\image{cordova-app-browser}
	{resolution=\screenshotRes}
	{Cordova-Beispiel-App im Browser}
		{Die Startseite der Beispiel-App aus \autoref{lst:index.html} lässt sich auch im Desktop-Browser öffnen und anzeigen, allerdings kann hier kein \lstinline|deviceready|-Event empfangen werden.}
	{Eigener Screenshot.}

\image{android_emulate_install}
	{width=1\textwidth}
	{Cordova-Beispiel-App im Android-Emulator}
		{Anzeige in der App-Übersicht (links) und Ausführung der Cordova-Beispiel-App in einem Android-Emulator (rechts). Das grüne Label zeigt den Empfang des \lstinline|deviceready|-Events an.}
	{\gls{cordova}-Dokumentation \cite{android_emulate_install.png}}

Bevor die \gls{hybrid-app} mit \gls{cordova} zum Laufen gebracht werden kann, muss sie, wie die meisten Computerprogramme, in ein ausführbares Format überführt, also \emph{gebaut} werden.
Im Falle der \gls{hybrid-app} bedeutet das, dass diejenigen plattformspezifischen Komponenten der \gls{app} hinzugefügt werden, die nötig sind, um die Verbindung zwischen der plattformunabhängigen Web-Schicht und der nativen Schicht des jeweiligen Zielbetriebssystems herzustellen.

Damit das \gls{cordova}-\gls{framework} die entsprechenden Schnittstellen in die Anwendung einfügen kann, muss vor dem Build-Prozess ein Satz an Zielplattformen angegeben werden. 
Voraussetzung hierfür ist, dass die \glspl{sdk} der jeweiligen Zielplattformen zu der verwendeten Entwicklungsplattform kompatibel und installiert sind \cite{Cordova-Docs_CLI}.

Über den \gls{cordova}-Befehl \lstinline|platform| lassen sich mit den Optionen \lstinline|add| und \lstinline|remove| \glspl{plattform} zur Projektkonfiguration der Anwendung hinzufügen bzw. entfernen (\seeref{lst:cordova-platform-add-remove}).
Die Ausführung dieser Befehle wirkt sich auf den Inhalt des \filename{platforms}-Ordners innerhalb der Projektstruktur aus \cite{Cordova-Docs_CLI}.

\includebash{cordova-platform-add-remove}{
	caption={\gls{cordova}-Skript um \glspl{plattform} zum Projekt hinzuzufügen bzw. zu entfernen. \linename 1 fügt die \gls{plattform} \gls{android} hinzu, \linename 2 entfernt diese wieder.},
	}

Der \lstinline|list|-Befehl dient dazu, eine Liste aller \glspl{plattform} des Projekts auszugeben. Wie bei den meisten anderen Befehlen auch, kann synonym auch eine Kurzschreibweise (\lstinline|ls|) verwendet werden (\seeref{lst:cordova-platform-list}).

\includebash{cordova-platform-list}{
	caption={\gls{cordova}-Skript zum Auflisten aller \glspl{plattform} des Projekts.},
	}

Anschließend kann mit dem \lstinline|build|-Befehl die \gls{app} gebaut werden:
\includebash{cordova-build}{
	caption={Bauen der \gls{cordova}-\gls{app} für \gls{android}.},
	}


Der \lstinline|build|-Befehl stellt dabei eine Kurzform für die beiden Befehle \lstinline|prepare| und  \lstinline|compile| dar (\seeref{lst:cordova-prepare-compile}).
Diese Aufteilung kann beispielsweise sinnvoll sein, um erst den \lstinline|prepare|- Befehl auszuführen und den plattformspezifischen Code, den Cordova in \filename{platforms/android} generiert, anschließend mit einer entsprechenden \gls{ide} (wie beispielsweise \gls{android-studio}) und deren nativem SDK anzupassen und zu kompilieren.\footnote{\vgl Nativ-Entwicklungsansatz, \seeref{sec:cordova-grundlegendes}.}


\includebash{cordova-prepare-compile}{
	caption={Ausführlichere Schreibweise für den \lstinline|build|-Befehl aus \autoref{lst:cordova-build}.},
	}


Ist der Build-Prozess abgeschlossen, kann die fertige \gls{app} auch mit dem \gls{cordova}-\gls{cli} getestet werden. 
Die Anwendung kann dafür entweder an einen Emulator, der in vielen Fällen mit dem jeweiligen \gls{sdk} mitgeliefert wird, übergeben (\seeref{lst:cordova-emulate}) oder direkt auf einem an den Entwicklungsrechner angeschlossenen Mobilgerät ausgeführt werden (\seeref{lst:cordova-run}).
Für letzteres müssen unter Umständen noch entsprechende Einstellungen auf dem Gerät vorgenommen werden \cite{Cordova-Docs_CLI}.

\includebash{cordova-emulate}{
	caption={Übergibt die \gls{app} an den Emulator des \gls{android}-\glspl{sdk}.},
	}

\includebash{cordova-run}{
	caption={Führt die \gls{app} auf einem angeschlossenen Mobilgerät aus.},
	}

\subsubsection{Schnittstelle zur mobilen Plattform}

Um den eigentlichen Mehrwert des \gls{cordova}-\glspl{framework} zu nutzen, also auf native Features der Zielplattformen zuzugreifen, sind verschiedene \glspl{plugin} nötig, die ebenfalls mit dem \gls{cli} verwaltet werden können.
\glspl{plugin} bestehen aus einem zusätzlichen Stück Code, der die Kommunikation zu den nativen Komponenten herstellt.
\gls{cordova} bietet von Haus aus einen Basissatz von ca. 17 Kern-\glspl{plugin} an,\footnote{In den Quellen finden sich teilweise unterschiedliche Informationen über die Anzahl der Basis-\glspl{plugin}. In der offiziellen \gls{cordova}-Dokumentation werden folgende aufgelistet: \emph{Battery Status, Camera, Contacts, Device, Device Motion (Accelerometer), Device Orientation (Compass), Dialogs, FileSystem, File Transfer, Geolocation, Globalization, InAppBrowser, Media, Media Capture, Network Information (Connection), Splashscreen} und \emph{Vibration} \cite{Cordova_Docs_Plugin_APIs}. Auf der \gls{cordova}-Homepage finden sich zudem noch die beiden Einträge \emph{Console} und \emph{Statusbar} \cite{Apache_Cordova_Contribute}.} es können aber auch eigene entwickelt werden.\footnote{So bietet \zB \gls{adobe} mit seiner \gls{phonegap}-Variante noch einige weitere \glspl{plugin} wie beispielsweise den Zugriff auf einen angeschlossenen Barcode-Scanner an, die über die Grundausstattung von \gls{cordova} hinausgehen \cite{Cordova-Docs_CLI}.}
Für die Erstellung und Veröffentlichung eigener \glspl{plugin} stellt \gls{apache} in seiner \gls{cordova}-Dokumentation eine Entwicklungsanleitung bereit \cite{Cordova_Plugin_Development_Guide}.

\paragraph{Einrichten von Plugins:}

Für die Installation und Dokumentation von \glspl{plugin} bietet \gls{apache} mit der \gls{plugin-registry} ein Online-Portal an, in der sich zur Zeit 229 \glspl{plugin} für das \gls{cordova}-\gls{framework} finden, darunter auch die \og Kern-\glspl{plugin} von \gls{apache} \cite{Cordova_Docs_Plugin_APIs}. Der überwiegende übrige Teil stammt von verschiedenen Entwicklern und Drittanbietern, die ihre nach dem \gls{plugin-dev-guide} entwickelten \glspl{plugin} diesem Portal hinzufügen können \cite{Cordova_Plugin_Development_Guide}, und bieten häufig spezielle Unterstützung für bestimmte mobile Plattformen an \cite{Cordova_Plugin_Registry_viewAll}.

Das \gls{cordova}-\gls{cli} bietet auch die Möglichkeit, alle verfügbaren \glspl{plugin} zu durchsuchen, hinzuzufügen, aufzulisten und zu entfernen (\seeref{lst:cordova-plugin-search}, \ref{lst:cordova-plugin-add}, \ref{lst:cordova-plugin-ls} und \ref{lst:cordova-plugin-rm}).


\includebash{cordova-plugin-search}{
	caption={Suchen von verfügbaren \glspl{plugin}.},
	}

\includebash{cordova-plugin-add}{
	caption={\gls{plugin} zur App hinzufügen.},
	}

\includebash{cordova-plugin-ls}{
	caption={Analog zum \lstinline|list|-Befehl für \glspl{plattform} (\seeref{lst:cordova-platform-list}) können auch \glspl{plugin} des aktuellen Projekts aufgelistet werden.},
	}

\includebash{cordova-plugin-rm}{
	caption={\gls{plugin} entfernen.},
	}

\paragraph{Verwendung von Plugins}	\label{sec:cordova-plugins-verwendung}

\gls{cordova}-\glspl{plugin} können einen Satz von Methoden und Objekttypen mitbringen, mit deren Hilfe Entwickler die Funktionen der jeweiligen \glspl{plugin} nutzen können.
So beinhaltet beispielsweise das \emph{contacts}-\gls{plugin} für den Zugriff auf die native Adressverwaltung des Geräts neben dem zentralen \lstinline|Contact|-Objekt, das eine Instanz eines Eintrags im Adressbuch des Geräts repräsentiert, die Typen \mbox{\lstinline|ContactName|,} \mbox{\lstinline|ContactField|,} \mbox{\lstinline|ContactAddress|,} \mbox{\lstinline|ContactOrganization|,} die einige der Attribute des \lstinline|Contact|-Objekts darstellen sowie das Konfigurationsobjekt \mbox{\lstinline|ContactFindOptions|,} das bestimmte Optionen für den Zugriff auf das Adressbuch kapselt und einen \mbox{\lstinline|ContactError|,} welcher bei Auftreten eines Fehlers, beispielsweise bei der Suche nach Kontakten, behandelt werden kann \cite{Cordova_Plugin_Registry_Contacts}.

Darüber hinaus können hier die beiden Methoden \lstinline|navigator.contacts.create| und \lstinline|navigator.contacts.find| verwendet werden, um neue Kontaktobjekte zu erstellen bzw. zu suchen (\seeref{sec:contacts}).
Das Präfix \lstinline|navigator| identifiziert ein elementares Objekt des \gls{js}-Kerns, das in der Webentwicklung Informationen über den verwendeten Webbrowser bereitstellt, wie beispielsweise den Namen oder dessen Version sowie Informationen über aktivierte Browserplugins \cite{selfhtml_navigator}.
Das \gls{cordova}-\gls{framework} ersetzt mittels seiner zentralen, plattformspezifischen \gls{js}-Datei \filename{cordova.js} das \gls{js}-Objekt \lstinline|navigator| durch einen eigenen \lstinline|CordovaNavigator| \cite{cordova.js_replaceNavigator}, dem dann durch die verschieden \glspl{plugin} neue Eigenschaften und Methoden zugewiesen werden können über die der Zugriff auf verschiedene \glspl{plugin} bewerkstelligt werden kann, wie in diesem Beispiel das \lstinline|contacts|-Objekt, welches die beiden \gls{plugin}-Methoden \lstinline|create| und \lstinline|find| beinhaltet.
Der \og \gls{js}-Objekttyp \lstinline|Contact| ist in der \gls{contacts-api} definiert und beinhaltet die Felder \mbox{\lstinline|id|,} \mbox{\lstinline|displayName|,} \mbox{\lstinline|name|,} \mbox{\lstinline|nickname|,} \mbox{\lstinline|phoneNumbers|,} \mbox{\lstinline|code|,} \mbox{\lstinline|addresses|,} \mbox{\lstinline|ims|,} \mbox{\lstinline|organizations|,} \mbox{\lstinline|birthday|,} \mbox{\lstinline|note|,} \mbox{\lstinline|photos|,} \lstinline|categories| und \lstinline|urls| sowie die Methoden \lstinline|clone| zum duplizieren von Kontaktinstanzen, \mbox{\lstinline|remove|,} um Kontakte aus dem Adressbuch zu entfernen und \mbox{\lstinline|save|,} mit der sich neu erstellte Kontakte in der Adressverwaltung des Geräts abspeichern lassen \cite{cordova-plugin-contacts}.
%TODO was sollte wohin? das sind doch schon wieder grenzen und möglichkeiten und sollte daher vielleicht eher umsetzung?

Je nach Aufgabengebiet unterscheiden sich die verschiedenen \glspl{plugin} in ihrer Verwendung.
Während bei einigen, wie \zB dem \gls{cordova}-\gls{plugin}, Schnittstellenmethoden über das \lstinline|navigator|-Objekt bereitgestellt werden (\seeref{lst:cordova-plugin-contacts}), bieten andere wie bspw. das \emph{Battery-Status}-\gls{plugin} die Behandlung verschiedener Events an,\footnote{Im Beispiel des \emph{Battery-Status}-\glspl{plugin} geben diese Auskunft über eine Statusänderung des Ladezustands oder den Anschluss an ans Stromnetz \cite{Cordova_Plugin_Registry_battery-status}.} die über das Hinzufügen eines \glspl{event-handler} zum \lstinline|window|-Objekt\footnote{Ähnlich wie das \lstinline|navigator|-Objekt ist auch das \lstinline|window|-Objekt Bestandteil der \gls{js}-Spezifikation, das Eigenschaften und Methoden zur Information und Steuerung des Browser-Fensters  bietet \cite{selfhtml_window}.} verarbeitet werden können (\seeref{lst:cordova-plugin-battery-status}) und wieder andere, wie \zB das \emph{Device}-\gls{plugin} stellen ein bestimmtes Objekt bereit, über das in diesem Fall Informationen abgefragt werden können \cite{Cordova_Plugin_Registry_Contacts, Cordova_Plugin_Registry_battery-status, Cordova_Plugin_Registry_device}.\footnote{In diesem Beispiel über das verwendete Gerät (\seeref{lst:cordova-plugin-device}).}

\includehtml{cordova-plugin-contacts.js}{label=lst:cordova-plugin-contacts, caption={Beispiel für die Verwendung des \emph{Contacts}-\glspl{plugin} für die Erstellung eines neuen \lstinline|Contact|-Objekts \cite{Cordova_Plugin_Registry_Contacts}.},}

\includehtml{cordova-plugin-battery-status.js}{label=lst:cordova-plugin-battery-status, caption={Beispiel für die Verwendung des \emph{Battery-Status}-\glspl{plugin} \cite{Cordova_Plugin_Registry_battery-status}.},}

\includehtml{cordova-plugin-device.js}{label=lst:cordova-plugin-device, caption={Beispiel für die Verwendung des \emph{Device}-\glspl{plugin}. Bei Empfang des \emph{DeviceReady}-\glspl{event} wird die Modell-Bezeichnung des Geräts in der Konsole ausgegeben \cite{Cordova_Plugin_Registry_device}.},}

\subsubsection{PhoneGap\,Build}	%TODO Evtl. umbenennen in "Remote Build Services" od. "Der Build-Prozess" oder so.
%TODO Gliederung: Eher als Alternative zum lokalen Build?

%TODO Noch grundsätzlicher einleiten: Es gibt auch Cloud-Build-Services wie zB.....

Mit seinem Online-Portal \gls{pg-build} bietet \gls{adobe} einen webbasierten Build-Service für \gls{phonegap}-Apps an, der den Bau-Prozess, im Falle der hybriden \glspl{app} also die eigentliche Anbindung an plattformspezifische Toolkits auf nativer Ebene sowie das Einbetten der \gls{web-app} in eine native WebView, auslagert und damit deutlich vereinfacht.

Wie in \autoref{fig:hybrid-apps-schaubild} dargestellt, muss dieser Entwicklungsschritt im Gegensatz zu den vorherigen\footnote{Also der Entwicklung des Codes und der Oberfläche sowie Tests.} trotz (oder gerade wegen) des plattformunabhängigen Charakters mehrfach (also einmal für jede Zielplattform) ausgeführt werden, um die vorher entwickelte Web-App in die Umgebung einer nativen App einzubetten (\seeref{sec:hybrid-app}).
Dabei kann unter anderem besonderer Mehraufwand auf der administrativen Ebene entstehen, da, wie im \autoref{sec:cordova} beschrieben, nicht jede Entwicklungsplattform zu den gewünschten Zielplattformen kompatibel ist, und somit für die Verwaltung der verschiedenen \glspl{sdk} und das Bauen der jeweiligen hybriden \glspl{app} der Einsatz mehrerer Entwicklungsplattformen nötig sein kann, sofern ein breites Spektrum an Zielplattformen angestrebt wird.

Um also nicht auf gar mehreren Entwicklungsrechnern sämtliche Toolkits aller erforderlichen Zielplattformen verwalten zu müssen, können \gls{phonegap}-Entwickler hier ihren Quellcode hochladen und die App für die verschiedenen Zielplattformen im jeweiligen Format zusammenbauen bauen lassen.
Allerdings bietet auch \gls{pg-build} nicht für sämtliche mobilen \glspl{plattform}, die von \gls{cordova}\,/\,\gls{phonegap} unterstützt werden, seinen Service an.
Während bis Version 2.9 noch \glspl{app} für \gls{ios}, \gls{android}, \gls{win-phone}, \gls{blackberry-os}, \gls{webos} und \gls{symbian} in dem Portal gebaut werden konnten,
sind ab der Version 3.0 nur die drei größten Betriebssysteme \gls{ios}, \gls{android} und \gls{win-phone} verfügbar \cite{PhoneGap_Build_Documentation_Supported-Platforms}.

Um den Build-Prozess über den Cloud-Build-Service einzuleiten, erstellt der Entwickler eine \gls{web-app} in seiner gewohnten \gls{ide} in Form von \gls{html}-, \gls{js}- und \gls{css}-Dokumenten, die er anschließend auf \gls{pg-build} hochlädt.
Hierfür muss zunächst über die Web-Oberfläche des Portals eine neue \gls{app} angelegt werden.

Abhängig vom jeweiligen Bezahlpaket haben \gls{phonegap}-Entwickler die Möglichkeit, ihre Anwendung als private oder öffentliche \gls{app} hochzuladen. 
Während private \glspl{app} entweder als Zip-Paket hochgeladen oder als Link zu einem \gls{git}-Repository\footnote{Beispielsweise wie die öffentlichen Apps auf \gls{github} gehosted oder einem eigenen \gls{git}-Repository} in \gls{pg-build} eingestellt werden können (\seeref{fig:phonegap-build_create-public}), müssen öffentliche (also \gls{opensource}-) \glspl{app} in Form eines \gls{github}-Repositorys zugänglich gemacht werden.

\begin{quoting}
\enquote{We only allow open-source apps to be built from public Github repos}~\cite{PhoneGap_Build_Apps}
\end{quoting}


\image{phonegap-build_create-public}
	{width=1\textwidth}
	{Erstellung einer neuen App auf PhoneGap\,Build}
		{Dialog für die Erstellung einer neuen privaten \gls{app} auf \gls{pg-build}. Links das Eingabefeld zum Eintragen eines \gls{git}-Repository-Links und rechts der Button zum Hochladen von Zip-Archiven.}
	{Eigener Screenshot.}

Neben der Bezahlvariante gibt es auch ein kostenloses Paket, das eine private \gls{app} beinhaltet, bei der kostenpflichtigen Variante sind bis zu 25 private App enthalten.
\gls{opensource}-\glspl{app} können bei allen Paketen unbegrenzt angelegt werden (\seeref{fig:phonegap-build_plans}).

\image{phonegap-build_plans}
	{width=1\textwidth}
	{PhoneGap\,Build-Pakete}
		{Übersicht über die verschiedenen Bezahlpakete von \gls{pg-build}: Ab 9,99\,\$ im Monat können Entwickler bis zu 25 private \glspl{app} anlegen, in der kostenlosen Variante nur eine private, aber unbegrenzt öffentliche.}
	{Eigener Screenshot \cite{Adobe_PhoneGap_Build_Plans}.}

Nach dem Hochladen des Codes auf \gls{pg-build}, wird dort automatisch der Build-Prozess für die \glspl{hybrid-app} initiiert.
Dabei sorgt \gls{pg-build} dafür, dass für jede \gls{plattform} die entsprechende \gls{phonegap}-\gls{js}-Bibliothek %TODO Checken (auch im Rest vom Text): cordova.js ist glaub ich gar keine Bibliothek!
 injiziert wird, welche die \gls{api} zur nativen Betriebssystemebene enthält (\seeref{sec:cordova}) \cite{PhoneGap_Build_Documentation_getting-started}.
Ist der Build-Prozess abgeschlossen, kann die \gls{app} im jeweiligen Format für die verschiedenen \glspl{plattform} als Direkt-Link oder per \gls{qr} heruntergeladen werden (\seeref{fig:phonegap_build_workflow} und \ref{fig:phonegap-build_apps}).


\image{phonegap_build_workflow}
	{width=1\textwidth}
	{PhoneGap\,Build-Workflow}
		{Schematische Darstellung eines möglichen Entwicklungsworkflows mit \gls{pg-build}: Der Code wird als Web-Anwendung auf \gls{github} hochgeladen, bei \gls{pg-build} über das \gls{github}-Repository aktualisiert und für die jeweiligen \glspl{plattform} gebaut, sodass die plattformspezifischen \glspl{app} heruntergeladen und auf den Zielgeräten installiert werden können.}
	{Eigene Grafik.}

%TODO Als SVG einbinden! -> Funktioniert noch nicht.
%TODO Bildquellen (auch die *innerhalb* des Bilder angeben.)

\image{phonegap-build_apps}
	{width=1\textwidth}
	{PhoneGap\,Build-Oberfläche}
		{Oberfläche des \gls{pg-build}-Portals: Detailansicht für eine Beispiel-\gls{app}. Hier können verschiedene Einstellungen vorgenommen werden, sowie der Code aus einem Repository aktualisiert und die fertig gebaute \gls{app} per Download-Button oder \gls{qr} heruntergeladen werden.}
	{Eigener Screenshot.}

\image{screenshot_app_android}
	{width=1\textwidth}
	{Tablet-Screenshot mit installierter PhoneGap\,Build-App}
	{Die Beispiel-App (\enquote{HelloWorld}) aus \autoref{fig:phonegap-build_apps} aus \gls{pg-build} auf einem \gls{android}-Gerät installiert.}
	{Eigener Screenshot.}

\part{Praktische Umsetzung} \label{sec:praxis} 
% (mein Anteil)  verweis auf technische grundlagen.

\chapter{Konzeption}	
%		Idee, Anforderungen -> plattformunabhägig
%		Funktionalitätsbeschreibung
%		Design (Architektur)
%		Bezug zur Technologie (Plattform(un)abhängige Anforderungen)

\section{Vorüberlegungen}	\label{sec:kriterien}
	%	(Aspekte / Vorüberlegungen (Sicherheit, Verfügbarkeit, plattformspezifische Anforderungen, Usability, ?Entwickler-Komfort?, ?Feature-Reichtum?, ...))
	%TODO Wenn der Abschnitt keine ganze Seite gibt, dann als Absatz am Anfang des Kapitels!
	
	%TODO Irgendwo erwähnen, welche Test-Zielplattformen ich verwendet habe. --> Evlt. auch in [impl].
	
 	Um nachfolgend die Exploration der ausgewählten Cross-Plattform-Technologie zu erläutern und die Eignung für die Praxis zu erörtern, sollen hier einige Vorüberlegungen getroffen werden.
 	
 	Die Anforderungen an mobile Anwendungen, sowie deren Einsatzgebiete bilden ein breites Spektrum ab. Verschiedene SDKs adressieren teilweise gezielt bestimmte Einsatzgebiete der späteren Applikation. Das führt zu der Frage, inwieweit die verwendeten Technologien für unterschiedlich motivierte Entwickler und Softwarefirmen, sowie deren Kunden, brauchbar sind. Hat ein Unternehmen beispielsweise bereits eine bestehende Software-Infrastruktur, sollte sich eine plattformunabhängige App gut in das bestehende System integrieren lassen.
 	
 	Für Entwickler und Unternehmen wiederum, kann eine plattformunabhängige Entwicklung Ressourcen sparen und somit potentiell kleineren Unternehmen den Einstieg in den Markt für mobile Anwendungen erleichtern. 
 	Ein solches Ressourcenersparnis stellt sich allerdings nur ein, wenn der Aufwand zur Einrichtung und Benutzung des Frameworks kleiner ist, als der für die Entwicklung einer separaten App für jede Plattform. 
 	Hier fließt mit ein, wie viele Zielplattformen für die Auslieferung der App angestrebt sind, sowie die Benutzerfreundlichkeit des Frameworks aus Entwicklersicht, also die Frage, wie komfortabel sich Anwendungen entwickeln lassen.
 	
 	Ein offensichtliches Kriterium zur Beurteilung des Frameworks anhand eines gegebenen Anwendungszwecks sind die zur Verfügung stehenden Features, die auf dem Mobilgerät genutzt werden können. Stimmt dieser nicht mit denen der nativen App-Entwicklung überein, ist zu klären, welche Features fehlen und auf welche der Entwickler eventuell verzichten kann oder muss.
 	
 	Weiter ist interessant, ob sich weitere potentielle Nachteile durch die Nutzung eines Frameworks zur hybriden App-Entwicklung ergeben und welche Vorteile sich heraus kristallisieren. 
 	Es ließe sich beispielsweise die Frage stellen, ob die verwendeten Web-Technologien einer hybriden App prinzipielle Stärken und Schwächen in puncto Performance haben. 
 	Hierüber wird die Beispiel-Anwendung allerdings wenig Aufschluss geben, da eine native Implementierung der gleichen Funktionen zur Vergleichbarkeit fehlt. Die Performance einer Hybrid-App ließe sich als einen Punkt der Nutzerfreundlichkeit bzw. Vor- und Nachteilen für den Endanwender sehen, was gerade für Consumer-Apps von großer Bedeutung ist.
 	
 	Weitere Vor- und Nachteile hybrider Apps ließen sich an den Fragen der Stabilität einer solchen App, also deren Fehleranfälligkeit und damit verbundenen Abstürzen der Software, sowie der Sicherheit der Software erörtern. Dies schließt die Punkte mit ein, bspw. durch Manipulation des bestehenden Codes, so dass Schadcode in externe Systeme oder auch Client-Geräte eingeschleust werden kann, sowie der Verschlüsselung von Datenübertragungen. 
 	Wichtig ist unter anderem die Verschlüsselung von Nutzerdaten auf dem Gerät, sowie die sichere Kommunikation der App mit externen Diensten und Geräten.
 	
 	Gerade für Client-Server-Anwendungen ist eine verschlüsselte Datenübertragung oft ein zentraler Punkt. Es lässt sich aber auch allgemeiner Fragen, wie gut sich solche Anwendungen durch hybride Apps umsetzen lassen. Gerade für eingangs erwähnte Firmen mit bestehender Infrastruktur kann dies eine wichtige Anforderung sein.

\section{Beispiel-Anwendung}	\label{sec:bsp-app} 
%	Zu Explorationszwecken zu implementierender Anwendungsfall (App-Prototyp)

%\subsection{Anforderungen}
% Eine einfache app, bei der plattformunabhängigkeit sinn macht (grenzen / möglichkeiten)

% [ Minimal, weil nicht eigentliches Thema der Arbeit.: ]
% (Motivation, Idee)

% Funktionalitätsbeschreibung 
% 		Anwendungsfalldiagramm
%		Anwendungsfall beschreibung

\doubleimage[contacts]{errands-home}
	{errands-home-ios}{errands-list-ios}
	{Startbildschirm \\ der Beispielanwendung}{Listenansicht \\der Beispielanwendung}
	{Beispiel-Anwendung im iOS-Simulator}
	{Startbildschirm und Listenansicht der Beispiel-Anwendung im \gls{ios-sim}.}
	{\ownScreenshot}

Für die praktische Erprobung und Untersuchung einer Technologie zur plattformunabhängigen \gls{app}-Entwicklung sollte hier mit \gls{cordova} ansatzweise eine Beispiel-Anwendung in Form einer plattformunabhängigen mobilen \gls{app} entwickelt werden.
Für die Auswahl eines geeigneten Anwendungsgebiets war ein Kriterium die Maßgabe, möglichst solche Anforderungen der Anwendung zu identifizieren, die eine gewisse plattformspezifische Anforderung haben, also (\enquote{plattformkritische}) Features benötigen, die sich beispielsweise nicht ohne weiteres mit einer reinen Web-Anwendung umsetzen lassen, um den eigentlichen Mehrwert eines entsprechenden Cross-Plattform-Frameworks nutzen zu können und aufzuzeigen.

Dabei soll der Fokus nicht speziell auf der Umsetzung eines möglichst breiten Funktionsumfangs der Anwendung selbst oder deren Usability liegen, sondern vielmehr anhand einer beispielhaften Implementierung weniger grundlegender plattformkritischer Features die Verwendung der Cross-Plattform-Technologie erläutert werden, um Möglichkeiten und Grenzen aus Sicht des Entwicklers aufzuzeigen.

Die Beispiel-App trägt den Titel \enquote{Besorgungen}, soll also für den Anwender ein Hilfswerkzeug für die Erledigung alltäglicher Aufgaben darstellen.
Letztendlich handelt es sich um eine etwas komplexere ToDo-Liste, in der sogenannte \enquote{Tasks} (also Aufgaben, Besorgungen) in Listen kategorisiert und verwaltet werden können.
Diese Listen können vom Nutzer selbst angelegt, gelöscht und verändert werden.
Um von der Schnittstelle zum Geräte-Adressbuch gebraucht zu machen, können Listen mit im Adressbuch vorhandenen Kontakten geteilt werden, d.\,h. es sollen Kontakte aus dem Adressbuch geladen und einer Liste als neues Listenmitglied hinzugefügt werden.

Weiterhin soll ein grundlegendes Prinzip darin bestehen, dass Tasks vielerlei Eigenschaften zugewiesen bekommen können.
So zum Beispiel Fotos, die von der Kamera des Geräts aufgenommen werden sollen, Ortsangaben, um GPS-Daten zu empfangen und zu verarbeiten sowie Datumsangaben für die Interaktion mit dem nativen Kalender.
Außerdem soll die Anwendung Benachrichtigungen über das Erreichen eines angegebenen Datums oder Ortes per nativem Benachrichtigungsmechanismus des jeweiligen Betriebssystems aussenden können.\footnote{Eine detailliertere Spezifikation der Anwendungsfälle für die geplante Beispiel-App findet sich in \fullpageref{sec:af}.}
%TODO nochmal überprüfen, ob das alles so sinn macht.
Der Entwicklungsstand des hier beschriebenen Beispiel-Codes beschränkt sich jedoch auf die Umsetzung der \og Funktionalität, Listen zu teilen, also neue Listenmitglieder anhand der Daten aus der Kontaktverwaltung des Geräts zu Listenobjekten hinzuzufügen.

\section{Ausgewählte Technologie und Architektur für die Implementierung}

Da \gls{cordova} eins der meist genutzten \glspl{framework} zur plattformunabhängigen \gls{app}-Entwicklung darstellt, das kostenfrei, \gls{opensource} und gut dokumentiert ist und zudem den Vorteil bietet, vorhandene Kenntnisse der Web-Entwicklung für die Erstellung von mobilen \glspl{app} nutzbar zu machen, wurde dieses hier für die Implementierung der in \autoref{sec:bsp-app} beschriebenen Beispiel-Anwendung und zur näheren Beleuchtung ausgewählt.
Weiterhin fand die auf \gls{jq} aufbauende Oberflächen-Bibliothek für mobile Web-Oberflächen \gls{jqm} sowie die \gls{data-binding}-Bibliothek \gls{ko} für die Umsetzung der hybriden \gls{app} hier Verwendung (\fullref{sec:hybrid-dev}).

Wie in \autoref{sec:ko} beschrieben, stellt \gls{ko} ein recht mächtiges und hilfreiches Mittel für die einfache und gut strukturierte Programmierung unter \gls{js} dar.
Der dadurch empfohlene und erleichterte Einsatz des \glspl{mvvm} und die damit einhergehende klare und leicht wartbare Struktur bildet die Grundlage für die Architektur der hier beispielhaft implementierten Anwendung.
	
Grundlegend besteht die Anwendung aus den drei in \autoref{fig:mvvm-pattern} abgebildeten Teilen \emph{\gls{view}} in Form einer \gls{html}-Seite, dem \gls{js}-Objekt \emph{\gls{model}}, das den Zugriff auf das Gerät bewerkstelligt und dessen Daten bereitstellt und dem \emph{\gls{view-model}}, welches die Eigenschaften und anzuzeigenden Daten der \gls{view} beinhaltet.
Das \gls{js}-\gls{framework} \gls{ko} hält dabei durch das \gls{data-binding} die Anzeige der Daten auf der \gls{view} mit den Daten im \gls{view-model} synchron (\seeref{sec:ko}).

Die Anbindung vom \gls{model} an das \gls{view-model} wurde hier über das \gls{observer-pattern} realisiert, mit dem eine lose Kopplung zwischen einer Komponente und deren \glspl{observer} erzielt werden kann.
Dabei registriert sich das \gls{view-model} bei dessen Initialisierung beim \gls{model} per Aufruf der Methode \lstinline|addEventListener(eventType, eventHandler)| (\seeref{lst:app:model.js}) und übergibt dabei als Parameter \enquote{\lstinline|eventType|} einen \gls{string}, der den Typ des Events angibt, auf dessen Aussendung sich das \gls{view-model} registriert sowie eine Funktion als \lstinline|eventHandler|, welche ausgeführt werden soll, sobald das Event empfangen wird (\seeref{lst:app:ViewModel.js}).
Dazu hat das \gls{model} einen Verweis auf eine \lstinline|ObserverMap|, welche die registrierten \gls{event-handler} der \gls{observer} in Listen speichert, die dem jeweiligen Ereignistyp, der bei der Registrierung angegeben wird, zugeordnet werden (\seeref{lst:app:ObserverMap.js}).

Statt also bei erfolgreicher Ausführung einer Operation die Daten direkt per Methodenaufruf an das \gls{view-model} zu übergeben, wird hier lediglich die Methode \lstinline|notifyObservers()| der \lstinline|ObserverMap| aufgerufen, wodurch die registrierten \gls{observer} über das Auftreten des Ereignisses benachrichtigt werden.
Das \gls{model} muss dabei keinerlei Kenntnis vom \gls{view-model} haben, lediglich im \gls{view-model} ist bekannt, welche Methoden des \glspl{model} aufgerufen werden können, bzw. welche Events dieses aussendet.

Durch diese lose Kopplung kann einerseits die Entwicklung erleichtert werden, da bei einer möglichen Aufteilung der Implementierung das Entwicklerteam, das für die Implementierung des \glspl{model} zuständig ist, keine Kenntnis über die Struktur des \glspl{view-model} oder gar der \gls{view} haben muss und darüber hinaus könnten auch weitere \gls{observer} auf Benachrichtigungen des \glspl{model} registriert werden, beispielsweise, um weitere Aktionen nach Auftreten eines Events durchzuführen.

Neben diesen architektonischen Aspekten wurde hier das \gls{observer-pattern} für die Kopplung zwischen \gls{model} und \gls{view-model} gewählt, da bei Anforderung der Daten durch das \gls{view-model} noch nicht bekannt ist, wann diese Daten genau zurückgeliefert werden, da das \gls{cordova}-\gls{framework} nach Aufruf einer \gls{plugin}-\gls{api}-Methode\footnote{Beispielweise \lstinline|navigator.contacts.find()| zum Anfordern der Kontaktdaten aus dem Adressbuch} ausgeführt wird (\seeref{lst:find-contacts}).} diese erst vom Gerät laden muss und so das Zurückliefern der Daten per Rückgabewert zu einem \lstinline|null|-Wert führen kann, da bei Aufruf der Methode \lstinline|findContacts()| die Daten unter Umständen noch nicht verfügbar sind (\seeref{lst:find-contacts}).
% Auskommentiert, da nicht sicher, ob das wirklich der Grund bei Cordova.
%Dazu dient auch die Übergabe eines \lstinline|onSuccess|-\glspl{handler}, der nach Erfolgreicher Ausführung der \og \gls{api}-Methode ausgeführt wird. 
%%TODO Theoretische erläuterungen zu cordova in Theorie-Teil verschieben!


\chapter{Implementierung der Geräte-Schnittstelle}	\label{sec:impl}	%TODO Gliederung bzw. Name des Abschnitts? Erklärung der Architekur-Impl ist ja auch Umsetzung --> Evtl. nochmal Strippgen fragen.
%TODO Titel sowieso schwierig, weil kapitel ja schon "Praktische Umsetzung" heißt.

%\section{Ausgewählte Technologie} 
%%TODO Gehört das hier wirklich noch hin? In Theorie-Teil ist doch eigentlich schon klar, welche Technologie ausgewählt wurde. % --> VERMUTLICH DOCH. GEHÖT JA ZU EIGENEM ANTEIL, TECHNOLOGIEN AUSZUWÄHLEN.
%%TODO Evtl. einfach mit in folgenden Abschnitt verbacken, weil nicht mehr als eine Seite wird.	--> Zumal sowas wie obsever-pattern bzw. mvvm an cordova bzw. knockoout hängt.



%\section{Implementierung der Geräte-Schnittstelle}
% Beschreibung der konkreten Umsetzung, Implementierung der wesentlichen Features mithilfe der Technologien (Funktionsweise (Wie genau bewerkstelligt \gls{framework} die Plattformunabhängigkeit / Was steckt dahinter?))

		
\section{Zugriff auf die Kontaktverwaltung des Geräts} \label{sec:contacts}

\subsection{Funktionsweise}

Im Anwendungsfall \namerefH{shareList} der in \autoref{sec:bsp-app} und \fullref{sec:af} beschriebenen Beispielanwendung soll die Anforderung erfüllt werden, aus der Anwendung heraus auf das native Adressbuch des Geräts zuzugreifen, um die bestehenden Kontakte zu laden und anzuzeigen, sodass diese vom Nutzer ausgewählt werden, an die Anwendung übergeben und als neues Listenmitglied in ein Listenobjekt eingetragen werden können.

Nachdem das \gls{cordova}-Plugin \emph{Contacts}, wie in \fullref{sec:cordova} beschrieben, zur Anwendung hinzugefügt wurde, lässt sich damit der Zugriff auf die native Kontaktverwaltung des jeweiligen Betriebssystems bewerkstelligen.
Dieses bietet beispielsweise die Möglichkeit, nach bestimmten Kontakten im Adressbuch zu suchen, neue Kontakte zu erstellen und dem Adressbuch hinzuzufügen, sowie bestehende Kontakte zu entfernen oder zu duplizieren \cite{Cordova_Plugin_Registry_Contacts}.

Um Kontakte im Adressbuch zu suchen, muss im Wesentlichen die Methode \lstinline|navigator.contacts.find(fields, onSuccess, onError, options)| ausgeführt werden, wobei \lstinline|fields| die zu durchsuchenden Datenfelder repräsentiert, \lstinline|onSuccess| die Funktion angibt, die bei erfolgreicher Ausführung der \lstinline|find|-Methode ausgeführt werden soll, \lstinline|onError| den \gls{errorhandler} bei Auftreten eines Fehlers beim Suchen der Kontakte und \lstinline|options| zusätzliche Optionen wie Suchfilter oder ein \lstinline|multiple|-\gls{flag}, das angibt, ob mehrere Kontakte zurückgegeben werden sollen (\seeref{lst:find-contacts}, \linename 26).
Die \lstinline|find|-Methode wird \gls{asynchron} ausgeführt, das heißt, dass der Aufrufer nicht wissen kann, wann das Ergebnis zurückgeliefert wird, also nicht als Rückgabewert zurückgegeben werden kann, sondern bei erfolgreicher Suche an den \og \lstinline|onSuccess|-\gls{handler} übergeben wird und dort weiter verarbeitet werden kann \cite{Cordova_Plugin_Registry_Contacts}.

Da in diesem Beispiel (\autoref{lst:find-contacts}) keine bestimmten, sondern \emph{alle} Kontakte angezeigt werden sollen, wird der \lstinline|find|-Methode kein spezieller \lstinline|filter| übergeben (\linename 23). 
Ebenso ist aus dem selben Grund hier keine Einschränkung der durchsuchten Felder nötig, sodass durch der \lstinline|fields|-Parameter den Wert \lstinline|"*"| hat, und so alle Felder des \lstinline|Contact|-Objekts durchsucht werden (\linename 25).
Bei erfolgreicher Ausführung der \lstinline|find|-Methode werden die zurückgegebenen Kontaktdaten als Parameter in Form eines \glspl{array}, das die entsprechenden JavaScript-Objekte vom Typ \lstinline|Contact| beinhaltet, an die \lstinline|onSuccess|-Methode übergeben und vom \gls{model} an dessen \gls{observer} (in diesem Fall das \gls{view-model}) versendet (\linename 13). 

Der Bezeichner \lstinline|events.FOUND_CONTACTS| stellt eine String-Konstante dar, die in einem globalen Konstanten-Objekt definiert wurde und den Typ des \glspl{event} identifiziert (\seeref{lst:app:consts.js}).

	\includehtml{praxis/find-contacts.js} { label=lst:find-contacts, caption={Verwendung des \emph{Contacts}-Plugins von \gls{cordova}.}}

Die in \autoref{lst:find-contacts} per \gls{observer-pattern} versendeten Daten werden in der entsprechenden \gls{handler}-Methode an die \lstinline|contacts|-Eigenschaft des \glspl{view-model} übergeben (\autoref{lst:ViewModel-Contacts}, \linenamepl 16\,-\,18), deren Elemente (also die Kontakt-Objekte) anschließend durch das Data-Binding in der Oberfläche angezeigt werden (\su).

	\includehtml{praxis/ViewModel-Contacts.js}{label=lst:ViewModel-Contacts, caption={Wesentlicher Ausschnitt des \glspl{view-model}.},}

Die Methode \lstinline|findContacts()|, die den Aufruf zum Laden der Kontakte an das \gls{model} delegiert, wird hier per Event-Binding an die Kontakt-Komponente der Oberfläche gebunden, sodass diese jedes mal aufgerufen wird, wenn die Kontaktliste auf- oder zugeklappt wird, um die Daten aus dem \gls{model} anzufordern (\seeref{lst:contacts-ui}, \linename 3).
Durch die Bindung der Eigenschaft \lstinline|contacts| in Form eines \emph{Observable Arrays} des \glspl{view-model} (\autoref{lst:ViewModel-Contacts}, \linename 6) an die \gls{listview} der Oberfläche werden in dieser Liste alle Kontakte des Adressbuchs angezeigt (\seeref{lst:contacts-ui}, \linename 7).

	\includehtml{praxis/contacts-ui.html}{label=lst:contacts-ui, caption={\gls{ui}-Komponente zur Darstellung der Kontaktliste und Auswahl einzelner Kontakte.},}
	
	\pagebreak
	
\doubleimage[contacts]{apps-overview}
	{apps-overview-android}{apps-overview-ios}
	{Die installierte App in der \gls{android}-\glspl{app}-Übersicht.}
	{Die installierte App in der \gls{ios}-\glspl{app}-Übersicht.}
	{Die installierte App in der App-Übersicht.}
	{Nach der Ausführung des \lstinline|cordova emulate|-Befehls wird die zu testende \gls{cordova}-\gls{app} in der Übersicht des Betriebssystems im jeweiligen Emulator angezeigt.}
	{Eigener Screenshot.}
	
\doubleimage[contacts]{contacts-app}
	{contacts-android}{contacts-ios}
	{\gls{android}-Adressbuch.}{\gls{ios}-Adressbuch.}
	{Natives Adressbuch mit Beispiel-Kontakten}
	{Anzeige des nativen Adressbuchs mit Beispiel-Kontakten.}
	{\ownScreenshot}
	
\doubleimage[contacts]{errands-members}
	{errands-members-expanded-android}{errands-members-expanded-ios}
	{Auswahl-Dialog unter \gls{android}.}{Auswahl-Dialog unter \gls{ios}.}
	{Kontakt-Auswahl in der Anwendung}
	{Kontakt-Auswahl-Dialog für das Hinzufügen von Kontakten aus dem Adressbuch. Die Anwendung lädt die Kontakte aus dem Adressbuch des Geräts in die Auswahl-Box der Anwendung (\seeref{fig:contacts-app}).}
	{\ownScreenshot}
	
\doubleimage[contacts]{errands-selected}
	{errands-members-selected-android}{errands-members-selected-2-ios}
	{Selektierte Kontakt-Einträge unter \gls{android}.}{Selektierte Kontakt-Einträge unter \gls{ios}.}
	{Übergabe der Kontakte an die Anwendung}
	{Übergabe der Kontakte an die Anwendung: Wird ein Kontakt ausgewählt, wird dieser dem \gls{obs}-\gls{array} \lstinline|contacts| des \glspl{view-model} (\seeref{lst:ViewModel-Contacts}) hinzugefügt und als \emph{ausgewählt} in der Oberfläche markiert.}
	{\ownScreenshot}
	
	\pagebreak
	
\subsection{Explorationsergebnisse} \label{sec:contacts-ergebnis}
Einige Unterschiede ergeben sich bei der Verwendung des \emph{Contacts}-\glspl{plugin} zwischen den verschiedenen Zielplattformen.
So können beispielsweise nicht alle Methoden der \gls{contacts-api} auf allen mobilen Plattformen gleichermaßen ausgeführt werden.
Die hier verwendete \lstinline|find()|-Methode bietet mit der Unterstützung für \gls{android}, \gls{blackberry-os}, \gls{ff-os}, \gls{ios}, \gls{win-phone} und \gls{win8} eine relativ  breite Kompatibilität für alle größeren gängigen und mobilen Betriebssysteme.
Allerdings auch werden einige Datentypen \emph{nicht} von allen Plattformen unterstützt, so zum Beispiel das \lstinline|ContactOrganization|-Objekt\footnote{\seefullpageref{sec:cordova-grundlegendes}.}, das lediglich bei der Entwicklung für \gls{android}, \gls{blackberry-os}, \gls{ff-os}, \gls{ios} und \gls{win-phone} zur Verfügung steht, damit aber immer noch alle großen Plattformen unterstützt \cite{Cordova_Plugin_Registry_Contacts}.

Doch auch bei den unterstützten Plattformen sind einige Besonderheiten zu beachten, wenn es um die Verwendung einzelner Attribute der verschiedenen Objekttypen geht.
So ist beispielsweise das Feld \lstinline|displayName| des \lstinline|Contact|-Objekts unter \gls{ios} nicht verfügbar, sodass in diesem Fall auf das \lstinline|name|- oder \lstinline|nickname|-Feld zurückgegriffen werden muss.
Ähnliches gilt auch für weitere Objekte oder deren Methoden und Felder, die unter bestimmten Plattformen nicht oder nur teilweise unter Berücksichtigung bestimmter Besonderheiten funktionieren \cite{Cordova_Plugin_Registry_Contacts}.

So wird beispielsweise \lstinline|find|-Methode grundsätzlich mit \gls{android}, \gls{blackberry-os}, \gls{ff-os}, \gls{ios}, \gls{win-phone} und \gls{win8} von den wichtigsten mobilen \glspl{plattform} unterstützt, doch bei letzterer besteht die Einschränkung, dass aus der hybriden \gls{app} heraus nicht ohne weiteres auf die Kontaktdaten zugegriffen werden kann, wie es bei den anderen \glspl{plattform} der Fall ist (\so), sondern eine Nutzerinteraktion nötig ist, um Kontakte aus dem Adressbuch auszuwählen und weiter zu verarbeiten, sodass bei Anfrage von Kontaktdaten die \gls{app} für das native \gls{win8}-Adressbuch geöffnet wird und der Nutzer die gewünschten Kontakte selbst von Hand auswählen muss.
Darüber hinaus bieten \gls{win8}-Kontakte lediglich einen Lesezugriff, sodass hier aus der Kontaktdatenbank geladene Objekte nicht verändert, gelöscht oder dupliziert werden können \cite{Cordova_Plugin_Registry_Contacts}.

Eine Kategorisierung von Kontakten über das \lstinline|cagegories|-Attribut wird bei \gls{ff-os} sowie \gls{ios} nicht unterstützt, sodass beim Versuch, diesen Wert anzufragen, \lstinline|null| zurückgegeben wird.
In Bezug auf die Unterstützung des \lstinline|Contact|-Objekts ergeben sich die meisten Einschränkungen für die Plattform \gls{win-phone}.
Hier finden sich in der \gls{contacts-api}-Dokumentation beispielsweise Informationen über ein teilweise leicht inkonsistentes Verhalten beim Erstellen bzw. Suchen von Kontakten.
So unterscheidet sich beispielsweise der Wert des \lstinline|displayName|-Attributs, das bei Erstellung eines \lstinline|Contact|-Objekts angegeben wird von dem, das bei der Suche zurückgeliefert wird.
Ebenso können bei der Erstellung eines \lstinline|Contact|-Objekts mehrere \glspl{url} angegeben werden, während bei den Objekten des Suchergebnisses nach Ausführung der \lstinline|find|-Methode lediglich \emph{eine} verfügbar ist.

Einige Listenfelder des \lstinline|Contact|-Objekts besitzen ein Attribut mit dem Namen \lstinline|pref|, dem ein boolescher Wert zugewiesen kann, der angibt, ob das jeweilige Objekt innerhalb einer Liste\footnote{So beispielsweise bei den Feldern \lstinline|emails|, \lstinline|phoneNumbers| oder \lstinline|addresses|, denen jeweils mehrere E-Mail-Adressen, Telefonnummern bzw. Adressen zugewiesen werden können.} das vom Nutzer präferierte darstellt.
Dieses \lstinline|pref|-Attribut wird ebenfalls unter einigen Plattformen nicht unterstützt, so \zB für \lstinline|phoneNumbers| und \lstinline|emails| unter \gls{win-phone} sowie beim \lstinline|addresses|-Feld unter \gls{android}\,2.X, \gls{blackberry-os}, \gls{ios} und \gls{win8}.
Die \lstinline|Contact|-Felder \mbox{\lstinline|note|,} \mbox{\lstinline|ims|,} \lstinline|birthdays| und \lstinline|categories| werden für \gls{win-phone} gar nicht unterstützt.

Das \og \lstinline|Contact|-Feld \lstinline|name| hat den Datentyp \lstinline|ContactName| und besteht aus den weiteren \gls{string}-Attributen \mbox{\lstinline|formatted|,} \mbox{\lstinline|familyName|,} \mbox{\lstinline|code|,} \mbox{\lstinline|middelName|,} \lstinline|honorificPrefix| und \lstinline|code|.
Grundlegend wird dieses Objekt zwar von allen elementaren Plattformen unterstützt, doch auch hier gibt es einige Einschränkungen.
Das \lstinline|formatted|-Attribut wird unter \gls{android}, \gls{blackberry-os}, \gls{ff-os} und \gls{ios} nur teilweise unterstützt und bietet für \gls{android}, \gls{ios} sowie \gls{ff-os} lediglich einen Lesezugriff.
Unter \gls{win8} stellt dieses das einzige Attribut des \lstinline|ContactName|-Objekts dar, alle anderen werden nicht unterstützt.

Während bei der nativen App-Entwicklung für \gls{ios} oder \gls{android} \gls{ui}-Kom\-po\-nen\-ten für die Arbeit mit Kontakten bereitstehen, liefert die \gls{contacts-api} hier lediglich Mechanismen für den Zugriff auf die Kontaktdaten, jedoch nicht die entsprechenden UI-Komponenten, da diese vom jeweiligen verwendeten \gls{gui}-Toolkit abhängen.
Die hier verwendete Oberflächen-Bibliothek \gls{jqm} beinhaltet lediglich allgemeine \glspl{widget} wie Listen, Buttons, Tabellen etc., sodass die Erstellung einer \gls{ui}-Komponente für die Auswahl von Kontakten dem Entwickler überlassen bleibt.

In Verbindung mit der Data-Binding-Bibliothek \gls{ko} kann eine solche Komponente jedoch relativ einfach erstellt werden, indem die Felder in der Oberfläche an Eigenschaften des \glspl{view-model} gebunden werden, welches durch Benachrichtigungen des Models, das den Zugriff auf die native Ebene des Betriebssystems abwickelt, die Daten der Anwendung und des Geräts anzeigen kann (\seeref{sec:ko}).

Grundsätzlich bietet die \gls{contacts-api} trotz einiger Einschränkungen (\so) ein nützliches Werkzeug mit einem relativ breiten Funktionsumfang für den grundlegenden Zugriff auf die Kontaktverwaltung der meisten großen mobilen Betriebssysteme.
Der für dieses Feature definierte Anwendungsfall konnte mit den Mitteln der \gls{cordova}-\gls{api} und weiteren Technologien der Webentwicklung wie \gls{ko} oder \gls{jqm} gut für die hier verwendeten Plattformen \gls{ios} und \gls{android} umgesetzt werden.
Trotz des grundsätzlich plattformunabhängigen Ansatzes sollte allerdings bei der Entwicklung mithilfe dieser \gls{api} die Unterstützung der verwendeten Features und Attribute und deren Besonderheiten für die angestrebten Zielplattformen in der \gls{contacts-api}-Dokumentation überprüft werden.
Somit ist eine Kenntnis über die angestrebten Zielplattformen für die Verwendung eines solchen \glspl{plugin} beinahe unabdingbar und die Entwicklung spezieller Funktionalitäten in gewisser Weise wieder ein Stück weit abhängiger von den verwendeten Plattformen. %TODO Den Satz evtl. auch ins Fazit!

%TODO Dopplungen / Gliederung checken!!
\todo{Was hiervon sollte in den nächsten Abschnitt, was ist doppelt erörtert?}

%			Explorationsergebnisse
%			Grenzen und Möglichkeiten bei der Umsetzung (Nach jeder Feature-Beschreibung)
%				Aha-Erlebnisse
%				Konkrete Quellcode-Beispiele
%				Ecken und Kanten, Spitzfindigkeiten zeigen 
%				Eigenschaften, Was fällt auf ? Am Beispiel

\chapter{Auswertung und Eignung für die Praxis} % Zusammenfassung der Erfahrung Umsetzung
%		Benutzung (Technik, Methodik, Entwicklerfreundlichkeit, Kompatibilität mit anderen Systemen / Software)
%		Generelle Bewertung des Frameworks und der Technologie sowie 
%		Vor-/Nachteile (zueinander / gegenüber nativer App-Entw.), Risiken, Chancen
%TODO In \cite{Apache_Cordova_Contribute} werden noch einige Plattformen mehr aufgelistet. (wohl je nach cordova-version) -> evlt. hier auch noch erwähnen und/oder auflisten.
























%\newcommand{\schlussname}{Resümee / Ausblick}

\unnumberedPart{Resümee / Ausblick}
%\phantomsection
%\part*{\schlussname}
%\addcontentsline{toc}{part}{\schlussname}
%Genereller Ausblick / Resümee (der ganzen Arbeit)

\unnumberedChapter{Resümee} %TODO Einen Sinnvollen Namen geben.

\unnumberedChapter{Ausblick} %TODO Einen Sinnvollen Namen geben.

%\chapter*{(Name des Resümees)}
%sadfksadf adfsadf
%
%\chapter*{Ausblick, evtl. auch mit Name?}
%dsalkjsad sadflkjdsf öljölkjölkj


% Verzeichnisse --------------------------------------------

%TODO Für alle ignored (und evtl. auch technologies)-Glossar-Begriffe: Hyperlinks entfernen.
% Glossar
\chapter*{\glossaryname} 	% Glossar-Überschrift ohne Nummerierung

% Begriffe
\addcontentsline{toc}{chapter}{\glossaryname}
\printglossary[style=long,title=Begriffe und Abkürzungen]
\pagebreak %TODO Besser irgendwie über Glossar-Einstellungen oder Style!

% Technologien
\printglossary[type=technologies,style=long]

% ----------------------------------------------------------
% Abbildungsverzeichnis
\addcontentsline{toc}{chapter}{\listfigurename}
\listoffigures

%TODO Code-Quellen evtl. unter Abschnitt "Software"
% Literaturverzeichnis
\MakeBibliography

%-----------------------------------------------------------

\end{document}

%===========================================================