\documentclass[bachelor,german]{hgbthesis}

% PACKAGES =================================================

\usepackage[
	textwidth=3.7cm,
	disable
	]	{todonotes}		% ToDo-Notizen
\usepackage[
	ngerman
	]	{translator}
\usepackage[
	nonumberlist,		%keine Seitenzahlen anzeigen
	acronym,			%ein Abkürzungsverzeichnis erstellen
	toc,					%Einträge im Inhaltsverzeichnis
%	section				%im Inhaltsverzeichnis auf section-Ebene erscheinen
	]	{glossaries}

% METADATA =================================================

\graphicspath{{images/}}
\bibliography{literatur}
\loadglsentries{glossar}
\loadglsentries{akronyme}
\makeglossaries

\title{Plattformunabhängige App-Entwicklung für mobile Geräte \\--\\ Grenzen und Möglichkeiten}
\author{Daniel Morgenstern}

\studiengang{Medieninformatik}
\hochschule{Beuth Hochschule für Technik Berlin}
\fachbereich{Informatik und Medien}
\studienort{Berlin}
\abgabedatum{2014}{04}{22}
\betreuerin{Prof. Dr. Simone Strippgen}
\semester{Wintersemester 2013/2014}

% Zeilenabstand 
\renewcommand{\baselinestretch}{1.2} 
%TODO soll nur in Textteil gelten, nicht in Bildunterschriften etc

% Customize the format of the first use.  See the manual for details if
% you want to include more information here such as the definition.
\defglsdisplayfirst[\glsdefaulttype]{\textit{#1}}

\begin{document}

% DEFINITIONS ==============================================

%TODO Funktioniert leider noch nicht.
% Hurenkinder und Schusterjungen verhindern
\clubpenalty10000
\widowpenalty10000
\displaywidowpenalty=10000
\interfootnotelinepenalty10000

%===========================================================
\frontmatter		% Anfangsteil (Römische Seitenzahlen)
\maketitle
\tableofcontents
%===========================================================

%===========================================================
\mainmatter			% Hauptteil (ab hier arab. Seitenzahlen)
%===========================================================
   
%TODO Analog zur unnummerierten Part-/Kapitel-Überschrift in Resümee hier evtl. auch so?
\chapter{Einleitung}

%\section{der Arbeit}

\section{Motivation und Aufgabenstellung} % (Aktualität, Interesse, Nutzen)
\begin{comment}
Bei der Entwicklung mobiler Anwendungen steht - besonders im Consumer-Bereich - neben vielen anderen Entscheidungen die Wahl der Ziel-Plattform an, wobei ein wichtiger Faktor sicher die Erreichbarkeit einer möglichst großen Nutzer-Anzahl darstellt.
Vermutlich um den  Entwicklungsaufwand nicht ins Bodenlose \todo{Stil!} ausufern zu lassen beschränken sich dabei viele App-Hersteller (oder -anbieter) auf die größten, meist genutzten Plattformen wie Android oder iOS. \todo{darf man so eine Behauptung jetzt einfach machen? oder wie weißt man sowas nach?}

Doch nicht nur, dass dadurch doppelter Wartungs- und Anpassungsaufwand für die Entwicklung entsteht, auch werden dadurch viele andere Betriebssysteme vernachlässigt, was letztlich zu Lasten der Nutzer geht, die auf einem vielfältigen aber auch diffundierten Markt - je nach Hersteller und Plattform - ein zum Teil eingeschränktes und ungleich verteiltes Angebot an Anwendungen vorfinden. 

Auf der anderen Seite hat der Bereich der Webentwicklung und -gestaltung in den letzten Jahren seit Aufkommen von Smartphones und Tablets eine neue Anforderung hinzu bekommen: Websites müssen nicht mehr nur für die unterschiedlichsten Browser auf dem Desktop angepasst werden, sondern sollen sich auch auf Geräten, die für Touch-Bedienung ausgelegt sind unterschiedlicher Bildschirmgrößen gleich gut anfühlen und bedienen lassen. Da der Trend für viele Firmen in Sachen Öffentlichkeitsarbeit, Kundenbindung und -freundlichkeit neben der Firmen-Website und der eigenen Facebook-Seite auch eine eigene App zu fordern scheint, liegt vor dem zuvor genannten Hintergrund der stark variierenden App-Formate der Ansatz nahe, auch den Webbrowser als eine weitere Plattform im bunten Gefüge aus Deployment-Anforderungen zu sehen, die nach Verminderung und Zusammenführung des Entwicklungsaufwands im mobilen Bereich verlangt. 

Letzterer Ansatz ist allerdings nur einer, den es zu Untersuchen gilt; Zentraler Forschungsgegenstand soll die Exploration der Möglichkeiten und Grenzen der plattformunabhängigen App-Entwicklung anhand eines beispielhaft implementierten Anwendungsfalls sein.
\end{comment}

\section{Ziel und Aufbau der Arbeit} % → NICHT Chronologisches Vorgehen!}

\chapter{Theoretische Grundlagen}
%1. Theoretische Grundlagen → als Grundlage für implementierung.
%Was muss der leser wissen, um die realisierung zu verstehen? 
%1. Konkreter, technoloischer: Worum geht es bei plattformunabhängiglkeit, wie ist zu erreichen? → veschiedene grundsätzliche Konzepte
%2. übersicht / Eigenschaften & Einordnung der vorgestellten L\"osungen
%3. Verwendete Technologien → wie macht das phonegap?
%1. Knockout, jquerymobile, etc.

\section{Apps für mobile Geräte}

Unter mobilen Apps (Kurzform für engl. \glqq Application\grqq) versteht man im Allgemeinen Anwendungssoftware für Tablet-Comuter oder Smartphones. Im Laufe der letzten Jahre haben sich auf dem Markt für Mobilgeräte durch viele konkurierende Gerätehersteller eine Vielzahl von Smartphone- und Tablet-Betriebssystemen herausgebildet. Im Entwicklungsbereich wird in dem Zusammenhang auch von einer Plattform gesprochen.

Zu den Plattformen mit dem höchsten Marktanteil zählen Googles Android, iOS von Apple, Microsoft Windows Phone und Blackberry OS des gleichnamigen Smartphone-Herstellers Blackberry. \cite{idc1}

Zumeist nutzen die verschiedenen Plattformen nicht nur unterschiedliche Dateiformate für ihre Anwendungen sondern auch unterschiedliche Programmiersprachen und Toolkits für die Programmierung. So verwendet beispielsweise Android die Programmiersprache Java in Kombination mit einem eigenen Android SDK, während Apps für Apples iOS mit ObjectiveC in der Firmeneigenen Entwicklungsumgebung Xcode (die auch nur auf dem firmeneigenen Betriebssystem Mac~OSX läuft) geschrieben werden. Somit ist die Plattformunabhängigkeit mit den herkömmlichen Entwicklungsmethoden kaum zu erreichen, da die verschiedenen Produkte und Plattformen zueinander nicht kompatibel sind.

Im Planungsprozess einer Softwarelösung ist in bestimmten Fällen eine denkbare Alternative zur nativen App eine Webanwendung, die mithilfe des Browsers abrufbar ist, also konkret eine dynamische Website, wie man sie auch von der herkömmlichen Internet-Nutzung eines Desktop-Systems kennt. 

Diese Variante hat zwar den Vorteil, dass die Anwendung plattformunabhängig ist, da jedes moderne (mobile) Betriebssystem einen Browser besitzt, der Entwicklungsaufwand also unabhängig von der Anzahl der Zielplattformen der gleiche bleibt, allerdings stößt der Entwickler schnell an seine Grenzen, wenn Hardware- oder betriebssystemnahe Anforderungen gestellt sind. So ist es beispielsweise mit Webtechnologien wie HTML oder Javascript nicht ohne weiteres möglich, auf die Kamera oder das GPS-Modul eines Gerätes oder das Adressbuch eines Betriebssystems zuzugreifen.
\todo{nachweis!}

Ebenso kommt die Webanwendung im Normalfall nicht für Offline-Anwendungen in Frage, da die meisten mobilen Betriebssysteme nicht vorsehen, dass der Nutzer aus dem Dateisystem eine HTML-Seite öffnet. \todo{nachweis!} 
Dahingehend liegt ein Ansatz zur plattformunabhängigen App-Entwicklung auf der Hand: Die Hybrid-App, also die Verwendung von Webtechnologien mithilfe einer Schnittstelle zwischen der Webanwendung und der nativen Ebene des Betriebssystems.

\section{Verwendete Technologien}
\subsection{Grundlegende Technologien}
Konkret heißt das, die Software wird als Webanwendung mit den zugehörigen Technologien (HTML, Javascript, CSS) entwickelt und für die jeweiligen benötigten Plattformen in eine native App eingebettet, welche hier allerdings hauptsächlich aus einer Web-View, also einer abgespeckten Variante eines Web-Browsers zur Anzeige der Webanwendung, besteht. Da dieser Ansatz der am häufigsten von Cross-Platform-Frameworks genutzte ist, \todo{nachweis!} stellt dieser auch den Schwerpunkt der in dieser Arbeit explorierten Technologien dar und soll hier näher erläutert werden. 

\subsection{Phonegap}
Die entscheidende Lücke auf dem Weg zur Hybrid-App zwischen Webanwendung und der nativen App schließt das Framework \emph{PhoneGap} von Adobe. 
Bei der Webanwendung taten sich in der Herleitung zwei grundlegende Problemfelder gegenüber der nativen App-Entwicklung auf. 
Zum Einen besteht eine Webanwendung aus mehreren Dokumenten, die auf einem Server liegen und nur über den Browser des Betriebssystems abgerufen werden, die Benutzung fühlt sich also für den Nutzer durch die ausbleibende Installation, die benötigte Internetverbindung und die optische Präsenz der Browser-Oberfläche nicht wie eine mobile App an. 
Zum Anderen können die Anforderungen an die Anwendung einen Zugriff auf native Features des Geräts oder seines Betriebssystems erfordern, der mittels herkömmlicher Webtechnologien nicht oder nur sehr eingeschränkt möglich sind. 

Für letzteres bietet PhoneGap eine Javascript-Bibliothek, die auf das Cross-Platform-Framework \emph{Cordova} von Apache aufbaut und den Zugriff auf native Features des jeweiligen Betriebssystems ermöglicht.\todo{wie genau, weiß ich noch nicht, noch rausfinden!}
Bei der konkreten Verwendung von nativen Features muss jedoch teilweise wieder auf die Unterstützung durch die jeweiligen Plattformen geachtet werden.\todo{oder zumindest muss deklariert werden, was verwendet werden soll. -> checken!}

Darüber hinaus übernimmt das Online-Portal \emph{PhoneGap~Build} den Bauprozess der App, also das Überführen in ein installierfähiges App-Format. 
Während der Entwickler ohne diesen Build-Service die verschiedenen Toolkits aller Zielplattformen lokal verwalten müsste, um Cordova zu verwenden, \cite{pgdoc1} reicht es hier aus, die Web-Anwendung (beispielsweise per öffentlichem Git-Repository) auf das Portal hoch zu laden, den Build-Prozess im Browser abzuwarten und die fertigen Apps auf das gewünschten Zielgerät herunterzuladen und zu installieren. 

Für Auftraggeber oder Entwickler, die ihren Code als sehr sensibel und vertraulich ansehen, könnte diese Variante allerdings ein Problem darstellen sein, da der Quellcode dann stets auch auf den Servern von Adobe liegt.\todo{gehört vllt. eher woanders hin.}

\subsection{JQueryMobile}
Die Javascript-Bibliothek JQueryMobile baut auf JQuery auf und bietet ein Oberflächen-Toolkit für mobile Webseiten. Sie besteht zum Einen aus einer Javascript-Datei, die in die HTML-Seite eingebunden wird und zum Anderen aus einem CSS, das für das an mobile Touch-Geräte angepasste aussehen verantwortlich ist. 
So können einerseits UI-Elementen explizit Style-Klassen aus dem JQueryMobile-CSS zugewiesen werden, andererseits sorgt das JQuery-Javascript ohnehin bei allen verwendeten HTML-Elementen dafür, dass die Style-Klassen dynamisch vergeben werden und die UI-Komponenten somit ihr App-typisches Aussehen erhalten.
Lediglich mit dem zusätzlichen Attribut "data-role" werden den UI-Elementen Eigenschaftentypen zugewiesen, über die die JQueryMobile-Bibliothek erkennt, wie dieses dargestellt werden soll.
\marginpar{Konkreter, Beispiel?}

Dieser Mechanismus erleichtert es dem Frontend-Entwickler erheblich, eine auf Touchscreens ausgerichtete Oberfläche zu entwerfen, da er im Grunde neben einigen zusätzlichen Attributen sämtliche herkömmlichen HTML-Elemente verwenden kann. 

\subsection{KnockoutJS}
Auch bei \emph{Knockout} handelt es sich um eine Javascript-Bibliothek, die per \mbox{\texttt{<script>}-Tag} der HTML-Seite hinzugefügt wird. Knockout übernimmt mit einem MVVM (Model-View-ViewModel) das \emph{Data-Binding}, also die Verknüpfung zwischen Daten-Feldern im Programm und UI-Elementen.
So lässt sich die UI (View) sehr klar von der Programmlogik (Model) trennen, was der Lesbarkeit, Erweiterbarkeit und Wartbarkeit der Software zugute kommt.

Statt also Javascript mit HTML-String-Schnipseln zu vermischen, indem das DOM direkt manipuliert und UI-Elemente zur Laufzeit
\todo{gibt es bei javascript überhaupt eine laufzeit?} 
programmatisch erweitert werden, definiert der Entwickler in einem separaten Javascript ein View-Model und bindet mit dem HTML-Attribut \glqq data-bind\grqq \ eine bestimmte View-Eigenschaft an ein Datenfeld aus dem View-Model.
\marginpar{Konkreter, Beispiel, Grafik?}

\chapter{Praktische Umsetzung} %  (mein Anteil)  verweis auf technische grundlagen.

	\section{Beispiel-Anwendung} %	Zu Explorationszwecken zu implementierender Anwendungsfall (App-Prototyp)
%		Idee, Anforderungen -> plattformunabhägig
%		Funktionalitätsbeschreibung
%		Design (Architektur)
%		Bezug zur Technologie (Plattform(un)abhängige Anforderungen)

	\section{Kriterien zur Bewertung}
	%	(Aspekte / Vorüberlegungen (Sicherheit, Verfügbarkeit, plattformspezifische Anforderungen, Usability, ?Entwickler-Komfort?, ?Feature-Reichtum?, ...))

\section{Umsetzung}

\subsection{Ausgewählte Technologie} % Entscheidung und Begründung für 1-2 Frameworks/Lösungen zur genaueren Evaluation
\subsection{Implementierung} % Beschreibung der konkreten Umsetzung, Implementierung der wesentlichen Features mithilfe der Technologien (Funktionsweise (Wie genau bewerkstelligt Framework die Plattformunabhängigkeit / Was steckt dahinter?))

Um auf die wie in [theorie] beschrieben, 
%TODO -> hinzufügen und verwenden von features / Bezug auf meine Implementierung / Quellcode / Weitere Möglichkeiten / Grenzen aus Plugin-Docs
		
\subsubsection{Zugriff auf die Kontaktverwaltung des Geräts}

%TODO Reihenfolge??
Im Anwendungsfall \emph{Liste Teilen} der oben beschriebenen Beispielanwendung \todo{Anwendungsfallbeschreibung einfügen.} soll die Anforderung erfüllt werden, aus der Anwendung heraus auf das native Adressbuch des Geräts zuzugreifen, um die bestehenden Kontakte zu laden und anzuzeigen, sodass diese vom Nutzer ausgewählt werden, an die Anwendung übergeben und als neues Listenmitglied in ein Listenobjekt eingetragen werden können.

Nachdem das \gls{cordova}-Plugin \emph{Contacts}, wie in \autoref{sec:cordova} beschrieben, zur Anwendung hinzugefügt wurde, lässt sich damit der Zugriff auf die native Kontaktverwaltung des jeweiligen Betriebssystems bewerkstelligen.
Dieses bietet beispielsweise die Möglichkeit, nach bestimmten Kontakten im Adressbuch zu suchen, neue Kontakte zu erstellen und dem Adressbuch hinzuzufügen, sowie bestehende Kontakte zu entfernen oder zu duplizieren \cite{Cordova_Plugin_Registry_Contacts}.

Um Kontakte im Adressbuch zu suchen, muss im Wesentlichen die Methode \lstinline|navigator.contacts.find(fields, onSuccess, onError, options)| ausgeführt werden, wobei \lstinline|fields| die anzuzeigenden Datenfelder repräsentiert, \lstinline|onSuccess| die Funktion angibt, die bei erfolgreicher Ausführung der \lstinline|find|-Methode ausgeführt werden soll, \lstinline|onError| den \gls{errorhandler} bei Auftreten eines Fehlers beim Suchen der Kontakte und \lstinline|options| zusätzliche Optionen wie Suchfilter oder ein \lstinline|multiple|-\gls{flag}, das angibt, ob mehrere Kontakte zurückgegeben werden sollen (\seeref{lst:find-contacts}, \linename 26).

Da in diesem Beispiel (\autoref{lst:find-contacts}) keine bestimmten, sondern \emph{alle} Kontakte angezeigt werden sollen, wird der \lstinline|find|-Methode kein spezieller \lstinline|filter| übergeben (\linename 23). In der Oberfläche soll der Name verfügbar sein, sodass der \lstinline|fields|-Parameter die entsprechenden Felder als Array beinhaltet (\linename 25). Bei erfolgreicher Ausführung der \lstinline|find|-Methode werden die zurückgegebenen Kontaktdaten als Parameter in Form eines Arrays, das die entsprechenden JavaScript-Objekte vom Typ \lstinline|Contact| beinhaltet, an die \lstinline|onSuccess|-Methode übergeben und vom \gls{model} an dessen \glspl{observer} (in diesem Fall das \gls{view-model}) versendet (\linename 13).
%TODO Am Ende nochmal checken, ob der Code wirklich so geblieben ist (Code + Beschreibung).

	\includehtml{praxis/find-contacts.js} { label=lst:find-contacts, caption={Verwendung des \emph{Contacts}-Plugins von \gls{cordova}.}}

Die in \autoref{lst:find-contacts} per \gls{observer-pattern} versendeten Daten werden in der entsprechenden Handler-Methode an die \lstinline|contacts|-Eigenschaft des \glspl{view-model} übergeben (\autoref{lst:ViewModel_Contacts}, \linenamepl 16\,-\,18), deren Elemente (also die Kontakt-Objekte) anschließend durch das Data-Binding in der Oberfläche angezeigt werden (\su).

\includehtml{praxis/ViewModel_Contacts.js}{label=lst:ViewModel_Contacts, caption={Wesentlicher Ausschnitt des \glspl{view-model}.},}

Die Methode \lstinline|findContacts|, die den Aufruf zum Laden der Kontakte an das \gls{model} delegiert, wird hier per Event-Binding an die Kontakt-Komponente der Oberfläche gebunden, sodass diese jedes mal aufgerufen wird, wenn die Kontaktliste auf- oder zugeklappt wird, um die Daten aus dem \gls{model} anzufordern (\seeref{lst:contacts-ui}, \linename 3).
Durch die Bindung der Eigenschaft \lstinline|contacts| in Form eines \emph{Observable Arrays} des \glspl{view-model} (\autoref{lst:ViewModel_Contacts}, \linename 6) an die \gls{listview} der Oberfläche werden in dieser Liste alle Kontakte des Adressbuchs angezeigt (\seeref{lst:contacts-ui}, \linename 7).

	\includehtml{praxis/contacts-ui.html}{label=lst:contacts-ui, caption={\gls{ui}-Komponente zur Darstellung der Kontaktliste und Auswahl einzelner Kontakte.},}
		
Einige Unterschiede ergeben sich bei der Verwendung des Contacts-Plugins zwischen den verschiedenen Zielplattformen.
So können beispielsweise nicht alle Methoden der contacts-\gls{api} auf allen mobilen Plattformen gleichermaßen ausgeführt werden.
Die hier verwendete \lstinline|find()|-Methode bietet mit der Unterstützung für \gls{android}, \gls{blackberry-os}, \gls{ff-os}, \gls{ios}, \gls{win-phone} und \gls{win8} eine relativ  breite Kompatibilität für alle größeren gängigen und mobilen Betriebssysteme.
Auch werden einige Datentypen nicht von allen Plattformen unterstützt, so zum Beispiel das \lstinline|ContactOrganization|-Objekt, das lediglich bei der Entwicklung für \gls{android}, \gls{blackberry-os}, \gls{ff-os}, \gls{ios} und \gls{win-phone} zur Verfügung steht, damit aber immer noch alle großen Plattformen unterstützt \cite{Cordova_Plugin_Registry_Contacts}.

Doch auch bei den unterstützten Plattformen sind einige Besonderheiten zu beachten, wenn es um die Verwendung einzelner Attribute der verschiedenen Objekttypen geht.
%TODO -> bspw. displayName ios
So ist beispielsweise das Feld \lstinline|displayName| des \lstinline|Contacts|-Objekts unter \gls{ios} und \gls{win-phone} nicht verfügbar und muss in diesem Fall durch das \lstinline|name|- oder \lstinline|nickname|-Feld ersetzt werden. %TODO Hier könnte man evtl. auch den direkten Link zu 'iOS Quirks' angeben. Aber wie in BiBTeX formatieren?
Ähnliches gilt auch für weitere Objekte oder deren Methoden und Felder, die unter bestimmten Plattformen nicht oder nur teilweise unter Berücksichtigung bestimmter Besonderheiten funktionieren \cite{Cordova_Plugin_Registry_Contacts}.

Während bei der nativen App-Entwicklung für \gls{ios} oder \gls{android} \gls{ui}-Komponenten für die Arbeit mit Kontakten bereitstehen, liefert die \gls{cordova}-\gls{api} hier lediglich Mechanismen für den Zugriff auf die Kontaktdaten, jedoch nicht die entsprechenden UI-Komponenten, da diese vom jeweiligen verwendeten \gls{gui}-Toolkit abhängen.
Die hier verwendete Oberflächen-Bibliothek \gls{jqm} beinhaltet lediglich allgemeine \glspl{widget} wie Listen, Buttons, Tabellen etc., sodass die Erstellung einer \gls{ui}-Komponente für die Auswahl von Kontakten dem Entwickler überlassen bleibt.

In Verbindung mit der Data-Binding-Bibliothek \gls{ko} kann eine solche Komponente jedoch relativ einfach erstellt werden, indem die Felder in der Oberfläche an Eigenschaften des \glspl{view-model} gebunden werden, welches durch Benachrichtigungen des Models, das den Zugriff auf die native Ebene des Betriebssystems abwickelt, die Daten der Anwendung und des Geräts anzeigen kann (\seeref{sec:ko}).

Grundsätzlich bietet die \gls{cordova}-Plugins-\gls{api} trotz einiger Einschränkungen (\so) ein nützliches Werkzeug mit einem breiten Funktionsumfang für den grundlegenden Zugriff auf die Kontaktverwaltung aller großen mobilen Betriebssysteme.


%			Explorationsergebnisse
%			Grenzen und Möglichkeiten bei der Umsetzung (Nach jeder Feature-Beschreibung)
%				Aha-Erlebnisse
%				Konkrete Quellcode-Beispiele
%				Ecken und Kanten, Spitzfindigkeiten zeigen 
%				Eigenschaften, Was fällt auf ? Am Beispiel

	\section{Fazit} % Zusammenfassung der Erfahrung Umsetzung
%		Benutzung (Technik, Methodik, Entwicklerfreundlichkeit, Kompatibilität mit anderen Systemen / Software)
%		Bewertung
%		Vor-/Nachteile (zueinander / gegenüber nativer App-Entw.), Risiken, Chancen


%Glossar ausgeben
\printglossary[style=altlist]
 
%Abkürzungen ausgeben
\deftranslation[to=German]{Acronyms}{Abkürzungsverzeichnis}
\printglossary[type=\acronymtype,style=long]
%TODO Abkürzungen statt in eigenes Abkürzungsverzeichnis lieber mit ins Glossar???
 
% Literaturverzeichnis ausgeben
\MakeBibliography

%===========================================================

\end{document}

