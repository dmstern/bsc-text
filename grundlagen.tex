\chapter{Theoretische Grundlagen}
%1. Theoretische Grundlagen → als Grundlage für implementierung.
% Was muss der leser wissen, um die realisierung zu verstehen? 
%1. Konkreter, technoloischer: Worum geht es bei plattformunabhängiglkeit, wie ist zu erreichen? → veschiedene grundsätzliche Konzepte
%2. übersicht / Eigenschaften & Einordnung der vorgestellten L\"osungen
%3. Verwendete Technologien → wie macht das phonegap?
%1. Knockout, jquerymobile, etc.

\section{Apps für mobile Geräte}

\subsection{Mobile (native) Apps} \label{native}
Unter mobilen \glspl{app} versteht man im Allgemeinen Anwendungssoftware für Tablet-Computer oder Smartphones. 
Im Laufe der letzten Jahre haben sich auf dem Markt für Mobilgeräte durch viele konkurierende Gerätehersteller eine Vielzahl von Smartphone- und Tablet-Betriebssystemen herausgebildet.
Im Entwicklungsbereich wird in dem Zusammenhang auch von \emph{Plattformen} gesprochen.

Zu den Plattformen mit dem höchsten Marktanteil zählen Googles Betriebssystem \gls{android}, \gls{ios} von Apple, \gls{win-phone} und \gls{blackberry-os} des gleichnamigen Smartphone-Herstellers Blackberry \cite{platforms-marketshare}.
% Nativ-Entwicklung für die jeweiligen Plattformen: Android / iOS
Die \gls{app}-Entwicklung für diese mobilen Betriebssysteme erfolgt mehr oder weniger ähnlich und soll im Folgenden, um auf die beiden größten Vertreter einzugehen, anhand von \gls{android} beziehungsweise \gls{ios} näher beschrieben werden.

%	-> \gls{sdk}s verwalten
Grundsätzlich müssen auf der \gls{ide} die entsprechenden \glspl{sdk} der Plattform, für die entwickelt wird, installiert sein. 
Diese enthalten Softwarekomponenten, die zur Entwicklung der \gls{app} notwendig sind, beispielsweise Klassen, die es einem erlauben, auf native Funktionalitäten des Betriebssystems wie zum Beispiel das Adressbuch, den Benachrichtigungsmechanismus oder auch auf Hardwarekomponenten wie die Kamera, den Bewegungssensor oder das \gls*{gps}-Modul zuzugreifen sowie die entsprechenden plattformspezifischen Oberflächenkomponenten des jeweiligen \gls*{gui}-Toolkits zu nutzen.

%	-> Code schreiben für die jeweilige Plattform
% Allgemein / Android
Als Programmiersprache für die \gls{android}-\gls{app}-Entwicklung wird \gls{java} verwendet. Das heißt, als Voraussetzung für die Entwicklung von \gls{android}-\glspl{app} ist lediglich eine geeignete \gls{ide} wie \gls{eclipse}, \gls{netbeans} oder \gls{intellij} sowie eine Installation des \gls{java}- und des \gls{android}-\gls{sdk} nötig. 
\todo{wie sieht es mit dem deployment, also der auslieferung in appstores etc aus?}
Seit 2013 bietet Google darüberhinaus die auf \gls{intellij} basierende und eigens für die \gls{android}-Entwicklung angepasste \gls{ide} \gls{android-studio} an \cite{android-studio}, die bereits alle notwendigen Toolkits enthält. 
Nachdem der Code geschrieben ist, kann er kompiliert und zu einem lauffähigen Programm gebaut werden (\seename\ \autoref{fig:hybrid-apps-schaubild}). Anschließend kann die \gls{app} in dem für die Zielplattform vorgesehenen Dateiformat ausgeliefert und auf dem Zielgerät installiert werden.

% iOS: 
Auch der Software- und Computer-Hersteller Apple bietet mit \gls{xcode} eine firmeneigene \gls{ide} zur \gls{app}-Entwicklung für sein mobiles Betriebssystem \gls{ios} an. Anders als Google geht der iPhone-Hersteller hier allerdings etwas restriktiver vor. So läuft die \gls{ide} \gls{xcode}, die man für die native \gls{ios}-Entwicklung benötigt, nur unter dem hauseigenen Betriebssystem \gls{osx} und das wiederum nur auf den firmeneigenen Mac-Rechnern. So sichert sich Apple auch durch jeden Entwickler einen neuen Kunden. \todo{darf man so was hier anmerken, oder lieber weg?}
Ansonsten verläuft der Entwicklungsprozess bei der \gls{ios}-Entwicklung im Prinzip ähnlich zur \gls{android}-Entwicklung (\seename\ \autoref{fig:hybrid-apps-schaubild}).
Als Programmiersprache wird \gls{obj-c} verwendet, einer um objektorientierte Elemente erweiterte Variante der Programmiersprache \gls{c}.

Möchte ein Auftraggeber einer Software also statt seinen Kunden nur eine \gls{app} für ein Betriebssystem anzubieten, einen größeren Nutzerkreis erschließen, muss die zu entwicklende \gls{app} für jede Zielplattform neu programmiert, getestet und gebaut werden, da jede mobile Plattform ihre eigenen Toolkits, Bibliotheken und Programmiersprachen verwendet, was die native \gls{app}-Entwicklung für potenzielle Auftraggeber zu einem sehr kostenaufwändigen Projekt werden lassen kann.
Andererseits bietet die native \gls{app}-Entwicklung vollständige Unterstützung der betriebssystemeigenen Funktionalitäten wie den Zugriff auf Kamera, Adressbuch, Bewegungssensoren etc. der jeweiligen Plattform, sodass ein Softwareprojekt mit solchen besonders hardware- oder betriebssystemnahnen Anforderungen die Entwicklung einer nativen (plattformspezifischen) \gls{app} notwendig erscheinen lassen kann.\footnote{Mehr dazu in \autoref{hybrid}}

\subsection{Web-Anwendungen}

% Zuerst gab es Websites, dann dynamische websites
Eine \gls{web-app} (oder dt. \emph{Web-Anwendung}) ist eine Andwendungssoftware, die auf einem Web-Server läuft und auf die der Nutzer mittels eines Browsers zugreifen kann; also eine dynamische Website, wie man sie auch schon vor dem Aufkommen von Smartphones und modernen Tablets kannte. 

Die Grundlage für die Enwicklung von Internetseiten bildet der langjährige Standard \gls{html}, mit dem deren Aussehen, Inhalt und Struktur textuell beschrieben werden kann. 
In Kombination mit \gls{css} für die modulare Gestaltung einer Website sowie \gls{js}, einer Skriptsprache zur \gls{dom}-Manipulation, bietet die HTML-Spezifikation in ihrer neusten Version (\gls{html}5) im Grunde alles, was für die Entwicklung einer modernen Benutzerschnittstelle am Computer notwendig ist. 
Die Fachlogik liegt, neben den Oberflächen-Komponenten in Form von \mbox{\gls{html}-,} \gls{css}- und Javascript-Dokumenten, auf einem Webserver und verarbeitet und reagiert auf Anfragen des Clients (hier also des Browsers).
Als Server-Technologie ist ein breites Spektrum an Programmiersprachen und Umgebungen einsetzbar (einige sind beispielsweise \gls{php}, \gls{java}, \gls{asp} u.\,v.\,a.\,m.).

Somit bietet die Entwicklung einer \gls{web-app} (abgesehen von einigen browser-spezifischen Eigenheiten) bereits eine gewisse Plattformunabhängigkeit, da jedes moderne (mobile) Betriebssystem über einen Webbrowser verfügt. 
Zwar müssen Entwickler in bestimmten Details bei der Erstellung des Codes auf die teilweise unterschiedliche Unterstützung (bspw. von \gls{html}-Elementen)\todo{genauer?} durch die verschiedenen Browser achten, aber darüberhinaus wird der Entwicklungsaufwand für eine \gls{web-app} nicht von der Anzahl der Zielplattformen bestimmt, da von Client-Seite aus verschiedene Browser durch die Verbreitung und Beachtung von Web-Standards weitgehend einheitliche \gls{html}-Dokumente lesen und interpretieren können und das Backend nicht auf Clients mit unterschiedlichen Plattformen, sondern auf Webservern liegt, deren Plattform bei der Entwicklung entweder schon bekannt oder nicht relevant ist (beispielsweise weil auch die Fachlogik plattformunabhängig mit \gls{php} oder \gls{java} realisiert wurde).

% Dann für sämtliche INet-Dienste auch noch eine \gls{app}
Obwohl es, durch damals eher im Business-Bereich verortete Internet-Handys und Palmtops, auch vor den heute üblichen mobilen Touch-Geräten bereits mobile Internetseiten gab, die speziell für die Darstellung auf kleinen Displays ausgerichtet waren, boten mit der massenhaften Verbreitung von mobilen, internetfähigen Geräten und deren (im Folgenden erläuterten) stark anwendungsorientierten Bedien-Konzepten viele herkömmliche Internet-Dienste nun auch zusätzlich eine native \gls{app} für verschiedene mobile Plattformen an.
So sind beispielsweise auch E-Mail-Dienste wie \gls{gmx}, \gls{web-de} oder \gls{gmail} seit der Verbreitung von Smartphones und Tablets auch in Form einer eigenen \gls{app} für \gls{android} und \gls{ios} vertreten, sodass der Nutzer, statt, wie von der Desktop-Computer-Nutzung gewohnt, einen anbieterunabhängigen Mail-Client zu konfigurieren, über den er seine E-Mails abruft, unter Umständen gleich die jeweilige \gls{app} des E-Mail-Anbieters startet \cite{gmx, web.de, gmail}.
Das heißt, der Nutzer folgt einem geänderten Bedienungsmuster seines Mobilgeräts gegenüber der herkömmlichen Computer-Nutzung: um zu einem bestimmten Ergebnis zu gelangen (bspw. \emph{Nachrichten lesen}) also die Frage zu beantworten, \emph{wie} er dahin gelangt (Einen Browser öffnen und zur gewünschten Seite navigieren: www.tagesschau.de), ist es für Anwender heutiger Mobilsysteme naheliegend, gleich die passende \gls{app} zu starten (hier bspw. die Tagesschau-\gls{app}).

% Gründe für App statt Web-Anwendung
Dafür gibt es verschiedene mögliche Gründe. Zum Einen muss im Gegensatz zu einer Website bei der mobilen \gls{app} nicht die komplette Oberfläche (\gls{html}-, \gls{css}- und JavaScript-Dokumente sowie Grafiken) übertragen werden, sondern lediglich die Nutzdaten (also beispielsweise, um beim obigen Beispiel zu bleiben, die Nachrichten in Textform), was dem Nutzer ein höheres Maß an Performanz einbringt.
Zum Anderen können trotz Vollbildmodus in bestimmten Fällen \gls{gui}-Elemente des Webbrowsers bei der Benutzung einer Web-Anwendung störend sein, so ist beispielsweise die Adresszeile am Rand nicht unbedingt erwünscht, wenn der Nutzer statt im Internet zu surfen dort eigentlich eine bestimmte Anwendung nutzen möchte. 
Ein anderes Beispiel für ein eventuell unerwünschtes Verhalten der Benutzerschnittstelle ist das der \emph{Menü}-Taste bei \gls{android}-Geräten, die im Falle der Nutzung einer Web-Anwendung über den Browser nicht den Kontext der eigentlich benutzten Anwendung (hier also der Website) anzeigt, sondern lediglich den des Browsers.

%TODO Stimmt so nicht ganz: offline websites gehen schon, aber logischerweise nur, wenn noch im browser-tab offen oder irgendwie runtergeladen und zu app-verknüpfung gepackt oder so...
In bestimmten Fällen kann eine nützliche Funktion einer \gls{app} die Offline-Nutzung sein, wenn beispielsweise durch die abgedeckten Anwendungsfälle keine Verbindung oder Synchronisation mit einem Server nötig ist. Beispiele hierfür könnten, um nur einige zu nennen, ein Taschenrechner, kleine Spiele, oder eine Bildverarbeitungs-\gls{app} sein. 
Für diese Offline-Nutzung einer \gls{app} zeichnet die Web-Anwendung ein geteiltes Bild: Zwar wurden in den letzten Jahren mehrere Methoden entwickelt, eine Web-Anwendung auch offline nutzen zu können, doch durch ihre (bereits am Namen erkennbare) Ausrichtung auf die Nutzung via Internet stellt die Implementierung dieser Funktionalität für Entwickler einen Zusatzaufwand dar. 
%TODO Wie offline: welche Thechnologien (nur erwähnen)
%TODO Quelle
Einige Möglichkeiten, eine Web-Anwendung offline-fähig zu machen, sind beispielsweise die aus der \gls{html}5-Spezifikation hervorgehenden Technologien \enquote{\gls{webstorage}}, ein Mechanismus zum lokalen Speichern von größeren Datenmengen in Form von Schlüssel-Wert-Paaren \cite{w3c_webstorage} sowie \gls{websql} bzw. \gls{indexed-db}, beides auf Web-Anwendungen otimierte Datenbanken-Spezifikationen, die vom \gls{w3c} herausgegeben werden \cite{w3c_websql, w3c_indexedDB}.
%TODO Auch hier: Abhängig von Browser (Hersteller + Version)

%TODO Formatierung / Befehle: Semantisch statt explizite Formatierung!

Allgemein kann man sagen, dass der Zugriff auf native Funktionalitäten des Geräts respektive des Betriebssystems nicht oder nur gering unterstützt wird, sodass der geringere Enwicklungsaufwand einer solchen \gls{web-app} (\seename\  \autoref{fig:hybrid-apps-schaubild}) unter Umständen zu Lasten des Funktionsumfangs und der Usibility der Anwendung geht.
\todo{recherchieren: wird was unterstützt? gibt es möglichkeiten, per javascript etc.? Was wird \enquote{gering} unterstützt? sollte man das noch weiter ausführen? touch-gesten etc.?}

\subsection{Hybride Apps}
%TODO Hybride-App: Begriff erläutern

%TODO Vor-/Nachteile der hybriden App-Entwicklung


%TODO Gibts hierfür auch so ne Art Grafik-Datenbank wie bei Bib? Dann müsste man den schmonsens nicht hier im Text verteilen.
%TODO Wie gibt man hier ne Bild-Quelle an?

\begin{figure}
\centering
\includegraphics[width=1\textwidth,natwidth=1,natheight=1]{hybrid-apps-schaubild.jpg}
\caption[Schaubild Hybrid Apps]{Entwicklungsstufen der verschiedenen Arten von \glspl{app}. Während bei der nativen \gls{app} der gesamte Entwicklungszyklus einmal pro Plattform durchlaufen werden muss, verringert sich der Aufwand für die \gls{web-app} erheblich. Bei der \gls{hybrid-app} muss die Anwendung zwar einmal für jede Plattform gebaut und ausgeliefert werden um die Schnittstellen für die nativen Plattformen zu implementieren, aber der hauptsächliche Entwicklungsaufwand des Programmierens und testens fällt aufgrund des generischen Charakters nur einmal an.}
\label{fig:hybrid-apps-schaubild}
\end{figure}

\section{Plattformunabhängige App-Entwicklung}

\subsection{Möglichkeiten des Erreichens von Plattformunabhängigkeit}
\subsection{Lösungen für die hybride App-Entwicklung}

\subsection{Entwicklung von hybriden Apps}\label{hybrid}

\begin{comment}
\subsection{Grundlegende Technologien}
Konkret heißt das, die Software wird als Webanwendung mit den zugehörigen Technologien (\gls{html}}, Javascript, \gls{css}) entwickelt und für die jeweiligen benötigten Plattformen in eine native \gls{app} eingebettet, welche hier allerdings hauptsächlich aus einer Web-View, also einer abgespeckten Variante eines Web-Browsers zur Anzeige der Webanwendung, besteht. Da dieser Ansatz der am häufigsten von Cross-Platform-Frameworks genutzte ist, \todo{nachweis!} stellt dieser auch den Schwerpunkt der in dieser Arbeit explorierten Technologien dar und soll hier näher erläutert werden. 
\end{comment}

\subsubsection{Phonegap}
Die entscheidende Lücke auf dem Weg zur \gls{hybrid-app} zwischen Webanwendung und der nativen \gls{app} schließt das Framework \gls{phonegap} von Adobe. 
Bei der Webanwendung taten sich in der Herleitung zwei grundlegende Problemfelder gegenüber der nativen \gls{app}-Entwicklung auf. 
Zum Einen besteht eine Webanwendung aus mehreren Dokumenten, die auf einem Server liegen und nur über den Browser des Betriebssystems abgerufen werden, die Benutzung fühlt sich also für den Nutzer durch die ausbleibende Installation, die benötigte Internetverbindung und die optische Präsenz der Browser-Oberfläche nicht wie eine mobile \gls{app} an. 
Zum Anderen können die Anforderungen an die Anwendung einen Zugriff auf native Features des Geräts oder seines Betriebssystems erfordern, der mittels herkömmlicher Webtechnologien nicht oder nur sehr eingeschränkt möglich sind. 

Für letzteres bietet PhoneGap eine Javascript-Bibliothek, die auf das Cross-Platform-Framework \gls{cordova} von Apache aufbaut und den Zugriff auf native Features des jeweiligen Betriebssystems ermöglicht. \todo{wie genau, weiß ich noch nicht, noch rausfinden!}
Bei der konkreten Verwendung von nativen Features muss jedoch teilweise wieder auf die Unterstützung durch die jeweiligen Plattformen geachtet werden. \todo{oder zumindest muss deklariert werden, was verwendet werden soll. -> checken!}

Darüber hinaus übernimmt das Online-Portal \gls{phonegap-build} den Bauprozess der \gls{app}, also das Überführen in ein installierfähiges \gls{app}-Format. 
Während der Entwickler ohne diesen Build-Service die verschiedenen Toolkits aller Zielplattformen lokal verwalten müsste, um Cordova zu verwenden,\cite{phonegap-doc-cordova} reicht es hier aus, die Web-Anwendung (beispielsweise per öffentlichem Git-Repository) auf das Portal hoch zu laden, den Build-Prozess im Browser abzuwarten und die fertigen \glspl{app} auf das gewünschten Zielgerät herunterzuladen und zu installieren. 

Für Auftraggeber oder Entwickler, die ihren Code als sehr sensibel und vertraulich ansehen, könnte diese Variante allerdings ein Problem darstellen sein, da der Quellcode dann stets auch auf den Servern von Adobe liegt.

\subsubsection{JQueryMobile}
Die Javascript-Bibliothek JQueryMobile baut auf JQuery auf und bietet ein Oberflächen-Toolkit für mobile Webseiten. Sie besteht zum Einen aus einer Javascript-Datei, die in die \gls{html}-Seite eingebunden wird und zum Anderen aus einem \gls{css}, das für das an mobile Touch-Geräte angepasste aussehen verantwortlich ist. 
So können einerseits UI-Elementen explizit Style-Klassen aus dem JQueryMobile-\gls{css} zugewiesen werden, andererseits sorgt das JQuery-Javascript ohnehin bei allen verwendeten \gls{html}-Elementen dafür, dass die Style-Klassen dynamisch vergeben werden und die UI-Komponenten somit ihr \gls{app}-typisches Aussehen erhalten.
Lediglich mit dem zusätzlichen Attribut \enquote{data-role} werden den UI-Elementen Eigenschaftentypen zugewiesen, über die die JQueryMobile-Bibliothek erkennt, wie dieses dargestellt werden soll.
\todo{Konkreter, Beispiel!}

Dieser Mechanismus erleichtert es dem Frontend-Entwickler erheblich, eine auf Touchscreens ausgerichtete Oberfläche zu entwerfen, da er im Grunde neben einigen zusätzlichen Attributen sämtliche herkömmlichen \gls{html}-Elemente verwenden kann. 

\subsubsection{KnockoutJS}
Auch bei \emph{Knockout} handelt es sich um eine Javascript-Bibliothek, die per \mbox{\texttt{<script>}-Tag} der \gls{html}-Seite hinzugefügt wird. Knockout übernimmt mit einem MVVM (Model-View-ViewModel) das \emph{Data-Binding}, also die Verknüpfung zwischen Daten-Feldern im Programm und UI-Elementen.
So lässt sich die UI (View) sehr klar von der Programmlogik (Model) trennen, was der Lesbarkeit, Erweiterbarkeit und Wartbarkeit der Software zugute kommt.

Statt also Javascript mit \gls{html}-String-Schnipseln zu vermischen, indem das \gls{dom} direkt manipuliert und UI-Elemente zur Laufzeit
\todo{gibt es bei javascript überhaupt eine laufzeit?} 
programmatisch erweitert werden, definiert der Entwickler in einem separaten Javascript ein View-Model und bindet mit dem \gls{html}-Attribut \enquote{data-bind} eine bestimmte View-Eigenschaft an ein Datenfeld aus dem View-Model.
\todo{Konkreter, Beispiel, Grafik.}

