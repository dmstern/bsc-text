
\unnumberedPart{Resümee und Ausblick}	\label{sec:schluss}
%Genereller Ausblick / Resümee (der ganzen Arbeit)

\unnumberedChapter{Resümee}\todo{Hier evtl. noch geeigneteren, aussagekräftigeren titel finden} %TODO Einen Sinnvollen Namen geben.
%TODO Zusammenfassung (Umfang: ca. 1-2 Seiten)
\todo{Kurzen Abriss über das, was erarbeitet wurde, einbauen. Was ist das Ergebnis der Arbeit?}

% Ergebnis: (Wo sind denn jetzt die Grenzen und Möglichkeiten?)
Durch den wachsenden Markt und die breite Vielfalt an angebotenen Lösungen sowie deren diversen Erweiterungsmöglichkeiten und meist guter Dokumentation bietet die plattformunabhängige App-Entwicklung zwar technisch einen deutlich anderen Entwicklungsansatz als die herkömmliche Herangehensweise für die plattformspezifische Realisierung nativer Apps, der jedoch kaum Möglichkeiten offen zu lassen scheint.
Dieser unterschiedliche Ansatz bietet sowohl Chancen in der [...]\todo{Ausformulieren} als auch Risiken mit [...] und muss im Einzelfall spezifisch abgewogen werden. 

\unnumberedChapter{Ausblick} \todo{Hier evtl. noch geeigneteren, aussagekräftigreen titel finden} %TODO Einen Sinnvollen Namen geben.
%TODO Ausblick (Umfang: ca. 1 Seite)
% Wie ließe sich die Arbeit fortsetzen? Wo wären Weiterentwicklungen denkbar?
%TODO Ausformulieren!
\todo{Ausformulieren.}

Implementierung, Erprobung und Auswertung weiterer Features im Praxis-Teil

Exploration mehrerer Lösungen zur plattformunabhängigen App-Entwicklung verschiedener Ansätze, statt nur einer.

Genauere Untersuchung der dahinterliegenden technischen Funktionsweisen anstelle des starken Fokus auf die Entwickler-Sicht (bspw. um Grundlagen für die technische Weiterentwicklung verschiedener Lösungen aufzuzeigen).

Evaluation über die Relevanz und Brauchbarkeit der beschriebenen Technologien für verschiedene Entwickler

Um Grenzen noch genauer auszuloten: Eigene Implementierung eines neuen Cordova-Plugins, das eine spezifische Anforderung erfüllt, die bis dato nicht mit den zur Verfügung stehenden Lösungen realisierbar ist.
Dabei könnte beispielhaft gezeigt werden, ob durch die modulare Erweiterbarkeit eine Frameworks wie Cordova der plattformunabhängigen App-Entwicklung tatsächlich keine Grenzen in der Machbarkeit, sondern eher in der Abwägung, ob sich der Implementierungsaufwand lohnt, gesetzt sind.

Für den Produktiv-Betrieb, bspw. in größeren Unternehmen oder kritischen Anwendungsbereichen, könnten Sicherheitsaspekte, wie in \fullref{sec:praxis} beschrieben, genauer beleuchtet werden. 

% Welche Auswirkungen können sich aus der Arbeit ggf. ergeben?
Der Überblick an Möglichkeiten und Lösungen sowie die Erläuterungen der Technologien und der Verwendung des Cordova-Frameworks kann unter bestimmten Voraussetzungen für Entwickler eine Grundlage bilden, um in der Konzeption und Auswahl von Technologien für die Entwicklung von mobilen 
