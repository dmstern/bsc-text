% Style-Definitions =======================================

% Zeilenabstand:
\renewcommand{\baselinestretch}{1.2} 
%TODO soll nur in Textteil gelten, nicht in Bildunterschriften etc

% Farben für Hyperlinks:
\definecolor{@darkblue}{RGB}{0,0,0}
\definecolor{@darkgreen}{RGB}{0,0,0}

% Style für erste Terminus-Nennung:
\defglsdisplayfirst[\glsdefaulttype]{\textit{#1}}
\defglsdisplayfirst[technologies]{\textit{#1}}
%\defglsdisplayfirst[ignored]{#1}
%\defglsdisplay[ignored]{{#1}}

% Hurenkinder und Schusterjungen verhindern:
\clubpenalty10000
\widowpenalty10000
\displaywidowpenalty=10000
\interfootnotelinepenalty10000

% Einstellungen für Code-Listings:
\definecolor{mygreen}{rgb}{0,0.6,0}
\definecolor{mygray}{rgb}{0.5,0.5,0.5}
%\definecolor{lightgray}{rgb}{0.1,0.1,0.1}
\definecolor{mymauve}{rgb}{0.58,0,0.82}
\definecolor{lila}{HTML}{891A44}
\definecolor{xlightgray}{HTML}{FAFAFA}

\lstset{ %
  backgroundcolor=\color{xlightgray},	% choose the background color
%  basicstyle=\footnotesize,		% the size of the fonts that are used for the code
%  breakatwhitespace=false,			% sets if automatic breaks should only happen at whitespace
%  breaklines=true,					% sets automatic line breaking
  captionpos=b,						% sets the caption-position to bottom
  commentstyle=\color{mygreen},		% comment style
  deletekeywords={...},				% if you want to delete keywords from the given language
  escapeinside={\%*}{*)},			% if you want to add LaTeX within your code
%  extendedchars=true,				% lets you use non-ASCII characters; for 8-bits encodings only, does not work with UTF-8
  frame=single,						% adds a frame around the code
  keepspaces=true,					% keeps spaces in text, useful for keeping indentation of code (possibly needs columns=flexible)
  keywordstyle=\color{lila},		% keyword style
%  language=HTML,					% the language of the code
  morekeywords={*,...},				% if you want to add more keywords to the set
  numbers=left,						% where to put the line-numbers; possible values are (none, left, right)
%  numbersep=5pt,					% how far the line-numbers are from the code
%  numberstyle=\tiny\color{mygray}, % the style that is used for the line-numbers
  rulecolor=\color{lightgray},			% if not set, the frame-color may be changed on line-breaks within not-black text (e.g. comments (green here))
  showspaces=false,                % show spaces everywhere adding particular underscores; it overrides 'showstringspaces'
%  showstringspaces=false,			% underline spaces within strings only
  showtabs=false,					% show tabs within strings adding particular underscores
  stringstyle=\color{mymauve},		% string literal style
%  tabsize=2,						% sets default tabsize to 2 spaces
  title=\lstname					% show the filename of files included with \lstinputlisting; also try caption instead of title
}

