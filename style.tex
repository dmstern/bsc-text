% COMMANDS =================================================

\renewcommand{\appendixpagename}{\appendixname}

% Querverweise:
\newcommand{\linename}{Zeile~}
\newcommand{\linenamepl}{Zeilen~}
%TODO Als Option für neues Kommando "seelst" oder so. -> Siehe Seiten-Option bei Zitationen.
\newcommand{\seeref}[1]{\seename\ \autoref{#1}}

\renewcommand*{\fullref}[1]{\hyperref[{#1}]{\autoref*{#1} \textit{\nameref*{#1}}}}
\newcommand{\seefullref}[1]{\seename\ \fullref{#1}}
\newcommand{\seefullpageref}[1]{\seefullref{#1}\ auf \pagename\ \pageref{#1}}

% Style für Dateinamen:
\newcommand{\filename}[1]{\textsf{#1}}

% Variable für Pfad zu Code-Listings:
\newcommand{\pathtolistings}{codelistings}
\newcommand{\listingpath}[1]{
	\renewcommand{\pathtolistings}{#1}
}

% Variable für Pfad zu Bsp-App-Code:
\newcommand{\pathtoappcode}{app}
\newcommand{\apppath}[1]{
	\renewcommand{\pathtoappcode}{#1}
}

% Seitenränder ---------------------------------------------

% Default Margin: siehe Master-Datei.
\newcommand{\bigcodemargin}{3cm}
\newcommand{\textbodymargin}{4cm}

% Image-Layout ---------------------------------------------

\newcommand{\fullimagesize}{1\linewidth}
\newcommand{\screenshotRes}{140}

% Bildunterschrift für Quellenangabe:
\newcommand{\imagesourcespace}{-1.5em}
\newcommand{\imagesourcefont}{\captionsetup{font={footnotesize,it}}}
\newcommand{\imagesourcename}{Quelle}
\newcommand{\imagesourcelabel}{\imagesourcename:\ }

% Neuer Befehl für Bilder:
\newcommand{\image}[5]{
	\begin{figure}[]
	\centering
	\includegraphics[
		#2,								% 	#4:	Größe: entweder 
										% 			'width=1\textwidth,' = gesamte Breite 
										% 			'resolution=\screenshotRes' = Orig.-gr.
		natwidth=1,						% 		Parameter für latexmk
		natheight=1						% 		Parameter für latexmk
		]{#1}							%	#1: Label und Dateiname
		\caption
			[#3]						%	#2: Kurzbeschriftung (im Verzeichnis)
			{#4}						%	#3: Beschriftung (am Bild)
		\label{fig:#1}					% 		Label: 'fig:' + 'DATEINAME'
		\vspace{\imagesourcespace}		% 		Abstand zur Quellenangabe
		\imagesourcefont{}				% 		Schrift f. Quellenangabe
		\caption*{\imagesourcelabel #5}	%	#5: Quellenangabe
	\end{figure}
}

% Layout ===================================================

% Layout für Code-Listings ---------------------------------
% (Wird weiter unten referenziert und weiter verfeinert)
\newenvironment{code}{
	\begin{singlespace}							% Einfacher Zeilenabstand
	}{
	\end{singlespace}
}

% Layout für kleine Code-Listings auf einer Seite ----------
% (Wird weiter unten referenziert und weiter verfeinert)
\newenvironment{minicode}{
	\par\noindent\begin{minipage}{\linewidth}	% Code auf einer Seite zusammenhalten
	\begin{code}
	}{
	\end{code}
	\end{minipage}\par\addvspace{\topskip}
}

% Neuer Befehl für include-Code-Listings:
% (Wird weiter unten referenziert und weiter verfeinert)
\newcommand{\includecode}[3]{
	\begin{code}
		\lstinputlisting[#1,#2]{\pathtolistings#3} % #1: Weitere Optionen / #2: Code-Datei
	\end{code}
}

% Neuer Befehl für include-Code-Listings:
% (Wird weiter unten referenziert und weiter verfeinert)
\newcommand{\includeminicode}[3]{
	\begin{minicode}
		\lstinputlisting[#1,#2]{\pathtolistings#3} % #1: Weitere Optionen / #2: Code-Datei
	\end{minicode}
}

% Neuer Befehl für HTML-Inludes:
\newcommand{\includehtml}[2]{
	\includeminicode{style=htmlcssjs}{#2}{#1}
}

% Neuer Befehl für HTML-Inludes:
\newcommand{\includebightml}[2]{
	\includecode{style=htmlcssjs}{#2}{#1}
}

% Neuer Befehl für Bash-Inlcudes:
\newcommand{\includebash}[2]{
	\includeminicode{
		style=shellscript,
		label=lst:#1
	}{#2}{#1}
}

%-----------------------------------------------------------

% Hurenkinder- und Schusterjungen-Regel:
\widowpenalties 3 10000 10000 100
\clubpenalties 3 10000 10000 100
\displaywidowpenalty=10000

% Fußnoten auf einer Seite behalten:
\interfootnotelinepenalty10000

% Abstand von Ziffer zu Fußnotentext:
\setlength{\footnotemargin}{1.2em}

%-----------------------------------------------------------

% Style für Glossar-Begriffe:
\defglsdisplayfirst[\glsdefaulttype]{\textit{#1}}
\defglsdisplayfirst[technologies]{\textit{#1}}
\defglsdisplayfirst[ignored]{\textit{#1}}
%\defglsdisplay[ignored]{{#1}}

%TODO Zusammengesetzte Begriffe wie GUI-Toolkit oder UI-Logik etc. NICHT Kursiv!

% COLORS ===================================================

% Farben für Hyperlinks ------------------------------------
\definecolor{@darkblue}{RGB}{0,0,0}
\definecolor{@darkgreen}{RGB}{0,0,0}

% Code-Farben ----------------------------------------------
% EIGENE
\definecolor{mygreen}{HTML}{93A83C}
\definecolor{xdarkgray}{HTML}{202020}
\definecolor{lila}{HTML}{891A44}
\definecolor{xlightgray}{HTML}{FCFCFC}
\definecolor{mycyan}{HTML}{056A70}
\definecolor{darkcyan}{HTML}{1B353B}
\definecolor{cyan-black}{HTML}{0D2025}

% AUS VORLAGE
\definecolor{lightgray}{rgb}{0.9, 0.9, 0.9}
\definecolor{darkgray}{rgb}{0.4, 0.4, 0.4}
%\definecolor{purple}{rgb}{0.65, 0.12, 0.82}
\definecolor{editorGray}{rgb}{0.95, 0.95, 0.95}
\definecolor{editorOcher}{rgb}{1, 0.5, 0} % #FF7F00 -> rgb(239, 169, 0)
\definecolor{editorGreen}{HTML}{619A0E} % #007C00 -> rgb(0, 124, 0)
\definecolor{orange}{rgb}{1,0.45,0.13}		
\definecolor{olive}{rgb}{0.17,0.59,0.20}
\definecolor{brown}{rgb}{0.69,0.31,0.31}
\definecolor{purple}{rgb}{0.38,0.18,0.81}
\definecolor{lightblue}{rgb}{0.1,0.57,0.7}
\definecolor{lightred}{rgb}{1,0.4,0.5}

\lstdefinestyle{htmlcssjs} {%
	% Code
	language=HTML5,
	alsolanguage=JavaScript,
	alsodigit={.:;},	
	tabsize=2,
	showtabs=false,
	showspaces=false,
	showstringspaces=false,
	extendedchars=true,
	breaklines=true,
	%
	% Code design
	identifierstyle=\color{xdarkgray},
	keywordstyle=\color{lila}\bfseries,
	ndkeywordstyle=\color{editorGreen}\bfseries,
	stringstyle=\color{mycyan}\ttfamily,
	commentstyle=\color{mygreen}\ttfamily,
	%
	% German umlauts
	literate=%
	{Ö}{{\"O}}1
	{Ä}{{\"A}}1
	{Ü}{{\"U}}1
	{ß}{{\ss}}1
	{ü}{{\"u}}1
	{ä}{{\"a}}1
	{ö}{{\"o}}1
}

\lstdefinestyle{shellscript} {%
	% Code
	language=bash,
%	alsolanguage=,
	alsodigit={.:;},	
%	otherkeywords={"cordova ", create, add, remove, ls, list, search, rm, run, build},
%	ndkeywords={platform, platforms, plugin, plugins},
	tabsize=2,
	showtabs=false,
	showspaces=false,
	showstringspaces=false,
	extendedchars=true,
	breaklines=true,
	%
	% Code design
	identifierstyle=\color{xdarkgray},
	keywordstyle=\color{lila}\bfseries,
	ndkeywordstyle=\color{editorGreen}\bfseries,
	stringstyle=\color{mycyan}\ttfamily,
	commentstyle=\color{mygreen}\ttfamily,
	%
	% German umlauts
	literate=%
	{Ö}{{\"O}}1
	{Ä}{{\"A}}1
	{Ü}{{\"U}}1
	{ß}{{\ss}}1
	{ü}{{\"u}}1
	{ä}{{\"a}}1
	{ö}{{\"o}}1
}

% Listing Settings =========================================
%
\lstset{
 	caption=\lstname,
	escapeinside={\%*}{*\%},
%	captionpos=b,
	%
	% General Listings-Design
	backgroundcolor=\color{xlightgray},
	basicstyle={\footnotesize\ttfamily},   
	numberstyle=\tiny\color{darkgray},
	frame=single,
	rulecolor=\color{lightgray},
	%
	% Line Numbers
	xleftmargin={0.75cm},
	numbers=left,
	stepnumber=1,
	firstnumber=1,
	numberfirstline=true,	
}

