
\chapter{Anwendungsfälle der Beispiel-Anwendung}

%TODO Kurz ein paar erläuternde Sätze zum Stand dieser Beschreibung und dem Stand der implementierung.

\pagebreak
\newgeometry{margin=\bigcodemargin}
\section{Anwendungsfalldiagramm}

\image{af-diagramm}{width=\fullimagesize}{Anwendungsfalldiagramm der Beispielanwendung}{Anwendungsfalldiagramm der Beispielanwendung. }{\ownGraphic}

\restoregeometry
\pagebreak

\section{Anwendungsfallbeschreibung}

\usecase{Listen verwalten}	\label{manageLists}

	\begin{description}
	
		\actors{initiiert vom Listenmitglied}
		\initcondition{Es existieren Listen für den angemeldeten Benutzer. (Das System legt für jeden Benutzer mindestens eine Standardliste an.)}
		
		\begin{verlauf}
		
		\item Das System zeigt eine Übersicht der gespeicherten Listen, derer der aktive Benutzer Mitglied ist.
		\item Für alle Listen hat er hier die Möglichkeit,
			\begin{enumerate}
			\item sich den Listeninhalt anzeigen zu lassen \\ (\extend{showList}) und 
			\item Listen zu teilen (\extend{shareList}). 
			\end{enumerate}
		\item Für alle Listen, deren er darüber hinaus Eigentümer ist, hat er die Möglichkeit, diese 
			\begin{enumerate}
			\item zu löschen (\extend{deleteList}), 
			\item umzubenennen (\extend{renameList}) und
			\item zu erstellen (\extend{createList}).
			\end{enumerate}
		\item Das System prüft den verfügbaren Bildschirmplatz und initiiert, sofern der Platz dafür ausreicht, automatisch den Anwendungsfall \namerefH{showList} anzeigen für die erste Standardliste.
		\end{verlauf}
	
	\end{description}

\usecase{Liste erstellen}	\label{createList}
	\begin{description}
	
		\actors{Initiiert vom Listeneigentümer}
		\begin{verlauf}
		
		
		\item Der Nutzer signalisiert dem System, dass er eine neue Liste erstellen möchte.
		\item Weiter mit \namerefH{nameList}.
		\end{verlauf}
	\end{description}

\usecase{Liste umbenennen}	\label{renameList}
	\begin{description}
		
		\actors{Initiiert vom Listeneigentümer}
		\begin{verlauf}
		
			\item Der Nutzer signalisiert dem System, dass er eine Liste umbenennen möchte.
			\item Weiter mit \namerefH{nameList}.
		\end{verlauf}
	\end{description}

\usecase{Liste benennen}	\label{nameList}
	\begin{description}
	
		\actors{Initiiert vom Listeneigentümer}
		\begin{verlauf}
			\item Das System zeigt ein Eingabefeld, in dem der neue Listenname eingegeben werden kann.
			\item Der Nutzer gibt den Namen für die neue Liste ein und bestätigt die Eingabe.
			\item Das System speichert die Liste mit dem eingegebenen Namen ab (je nach Kontext eine neu angelegte oder die alte mit neuem Namen).
			\item Die Ansicht wechselt wieder zur Listenansicht, in der jetzt auch die neu angelegte Liste erscheint.
		
		\end{verlauf}
	\end{description}
	
\usecase{Liste löschen}		\label{deleteList}
	\begin{description}
		\actors{Initiiert vom Listeneigentümer}
		\begin{verlauf}
		
		\item Der Nutzer signalisiert dem System, dass er eine Liste löschen möchte.
		\item Weiter mit \namerefH{deleteObject}.
		
		\end{verlauf}
	\end{description}

\usecase{Objekt löschen}	\label{deleteObject}
	Teil-Anwendungsfall. 
	\begin{description}
		\begin{verlauf}
		\item Das System öffnet einen Dialog, in dem der Nutzer gefragt wird, ob er sicher ist, dass er das ausgewählte Objekt löschen möchte.
		\item Der Nutzer bestätigt den Dialog zum Löschen.
		\item Das System löscht das gewählte Objekt aus dem System und wechselt wieder zur vorherigen Ansicht.
		\end{verlauf}
		\begin{altVerlauf}
			\setcounter{enumi}{1}\item Der Nutzer bricht den Vorgang ab.
		\end{altVerlauf}

	\end{description}

\usecase{Liste teilen}		\label{shareList}
	\begin{description}
		\actors{Initiiert vom Listenmitglied}
		\begin{verlauf}
		
			\item Der Nutzer signalisiert dem System, dass er andere Nutzer zu einer Liste einladen möchte.
			\item Das System wechselt zum Adressbuch des Betriebssystems.
			\item Der Nutzer wählt einen oder mehrere Kontakte aus seinem Adressbuch aus und bestätigt die Eingabe mit \enquote{Fertig}.
			\item Das System wechselt wieder zur Besorgungen-App und zeigt einen Dialog, in dem die ausgewählten Kontakte noch einmal aufgelistet werden.
			\item Der Nutzer bestätigt die Eingabe mit \enquote{Liste teilen}.
			\item Das System speichert die Information der neuen Listenmitglieder ab.
			\item Das System versendet Benachrichtigungen an die eingeladenen Nutzer, dass eine Liste mit ihnen geteilt wurde.
		
		\end{verlauf}
	\end{description}
	
\usecase{Tasks verwalten\,/\,Listeninhalt anzeigen} \label{showList}
	\begin{description}
		\actors{initiiert vom Listenmitglied}
		\begin{verlauf}
		
		\item Der Nutzer signalisiert dem System mit der Auswahl einer Liste, dass er deren Inhalt anzeigen oder verwalten möchte.
		\item Das System zeigt neben den vorher schon angezeigten Listen nun auch den Inhalt der selektierten Liste, also die darin enthaltenen Tasks an. 
		\item Von hier aus hat der Nutzer folgende Interaktionsmöglichkeiten (extend):
			\begin{enumerate}
				\item \namerefH{shareList}
				\item \namerefH{createTask}
				\item \namerefH{delegateTask}
				\item \namerefH{editTask} \\
				Außerdem werden dem Nutzer je nach verfügbarem Bildschirmplatz auch für alle Tasks Shortcuts zu allen Extend-Anwendungsfälle von \namerefH{editTask} angezeigt.
			\end{enumerate}
		
		\end{verlauf}
	\end{description}
	
\usecase{Task löschen}		\label{deleteTask}
	\begin{description}
		\actors{Initiiert vom Listenmitglied}
		\begin{verlauf}
		
			\item Der Nutzer signalisiert dem System, dass er einen Task löschen möchte.
			\item Weiter mit \namerefH{deleteObject}.
			
		\end{verlauf}
	\end{description}
	
\usecase{Task erstellen}	\label{createTask}
	\begin{description}
		\actors{Initiiert vom Listenmitglied}
		\begin{verlauf}
		
			\item Der Nutzer signalisiert dem System, dass er einen Task erstellen möchte.
			\item Das System öffnet ein Eingabefeld, in das der Name für den neuen Task eingetragen werden kann.
			\item Der Nutzer trägt den Namen für den Task ein und bestätigt die Eingabe.
			\item Das System speichert den neuen Task in die ausgewählte Liste ab.
		
		\end{verlauf}
		
	\end{description}
\usecase{Task umbenennen}	\label{renameTask}
	\begin{description}
		\actors{Initiiert vom Listenmitglied}
		\begin{verlauf}
		
			\item Der Nutzer signalisiert dem System, dass er einen Task umbenennen möchte.
			\item Das System öffnet ein Eingabefeld, in das der neue Name für den Task eingetragen werden kann.
			\item Der Nutzer trägt den Namen für den Task ein und bestätigt die Eingabe.
			\item Das System ändert den Namen des Tasks und speichert ihn ab.
		
		\end{verlauf}
	\end{description}
\usecase{Task bearbeiten}	\label{editTask}
	\begin{description}
		\actors{initiiert vom Listenmitglied}
		\begin{verlauf}
		
		\item Der Nutzer signalisiert dem System, dass er einen Task bearbeiten möchte.
		\item Das System öffnet den Bearbeiten-Dialog für den ausgewählten Task.\\
		Der Nutzer hat folgende Interaktionsmöglichkeiten:
			\begin{enumerate}
				\item \namerefH{addImage}
				\item \namerefH{terminateTask}
				\item \namerefH{locateTask}
				\item \namerefH{delegateTask}
				\item \namerefH{finishTask}
				\item \namerefH{abortTask}
				\item \namerefH{renameTask}
			\end{enumerate}
		
		\end{verlauf}
	\end{description}
	
\usecase{Bild an Task hängen}	\label{addImage}
	\begin{description}
		\actors{Initiiert vom Listenmitglied}
		\begin{verlauf}
		
			\item Der Nutzer signalisiert dem System, dass er ein Bild an einen Task hängen möchte.
			\item (In der Prototyp-Version deaktiviert) Das System öffnet einen Dialog, in dem gefragt wird, ob ein Bild mit der Kamera aufgenommen oder ein Bild aus der Galerie ausgewählt werden soll.
			\item Der Nutzer wählt die Option \enquote{Von der Kamera}.
			\item Das System wechselt zur Kamera-Anwendung des Betriebssystems.
			\item Der Nutzer nimmt ein Foto auf und bestätigt die Eingabe mit \enquote{Fertig}.
			\item Das System wechselt wieder zur Besorgungen-App und zeigt in einem Dialog das aufgenommene Foto.
			\item Der Nutzer bestätigt die Eingabe mit \enquote{Anhängen}.
			\item Das System speichert das Foto zum ausgewählten Task ab.
				
		\end{verlauf}
		
	\end{description}

\usecase{Task terminieren}	\label{terminateTask}
	\begin{description}
		\actors{Initiiert vom Listenmitglied}
		\begin{verlauf}
		
			\item Der Nutzer signalisiert dem System, dass ein Task an einem bestimmten Datum erledigt werden soll.
			\item Das System zeigt ein Datumsauswahlfeld an.
			\item Der Nutzer gibt das entsprechende Darum in das Feld ein.
			\item Das System fragt in einem Dialog, ob der Task auch in den Kalender des Betriebssystems eingetragen werden soll.
			\item Der Nutzer bestätigt den Dialog.
			\item Das System speichert den Task als ganztägigen Termin in den Kalender des Betriebssystems ab und gibt eine entsprechende Meldung über das erfolgreiche Speichern zurück.
				
		\end{verlauf}
		
	\end{description}

\usecase{Task verorten}		\label{locateTask}
	\begin{description}
		\actors{Initiiert vom Listenmitglied }
		\begin{verlauf}
		
			\item Der Nutzer signalisiert dem System, dass ein Task an einem bestimmten Ort erledigt werden soll.
			\item Das System zeigt ein Eingabefeld, um den Ort einzugeben.
			\item Der Nutzer trägt den Ort in das Eingabefeld ein (entweder gleich als Freitextsuche oder vorerst nur als reine GPS-Koordinaten).
			\item Das System speichert den angegebenen Ort zum betreffenden Task ab.
				
		\end{verlauf}
		
	\end{description}

\usecase{Task zuweisen}		\label{delegateTask}
	\begin{description}
		\actors{Initiiert vom Listenmitglied}
		\begin{verlauf}
		
			\item Der Nutzer signalisiert dem System, dass er einen Benutzer für einen Task zuständig zuweisen möchte.
			\item Das System zeigt eine Liste der Listenmitglieder der aktuellen Besorgungsliste. 
			\item Der Nutzer wählt einen Nutzer aus dieser Liste aus.
			\item Das System speichert den ausgewählten Nutzer als neuen Task-Inhaber ab und zeigt mit einem kleinen Icon am Task an, wem dieser Task zugewiesen ist.
		
		\end{verlauf}
		
	\end{description}

\usecase{Erledigung abbrechen (Task zurückgeben)}	\label{abortTask}
	\begin{description}
		\actors{Initiiert vom Task-Inhaber}
		\initcondition{Der Task wurde einem Nutzer zugewiesen, der dadurch zum Task-Inhaber wird.}
		\begin{verlauf}
		
			\item Verlauf wie bei \namerefH{delegateTask}.
			\item Statt 3.: Der Nutzer wählt aus der Liste den Default-Wert \enquote{(kein)} aus.
				
		\end{verlauf}
	\end{description}

\usecase{Task übernehmen}	\label{takeTask}
	\begin{description}
		\actors{Initiiert vom  Listenmitglied}
		\begin{verlauf}
		
			\item Verlauf wie bei \namerefH{delegateTask}.
			\item Statt 3.: Der Nutzer wählt aus der Liste sich selbst aus.
		
		\end{verlauf}
		
	\end{description}

\usecase{Task abschließen}	\label{finishTask}
	\begin{description}
		\actors{Initiiert vom Listenmitglied}
		\begin{verlauf}
		
			\item Der Nutzer signalisiert dem System, dass ein Task erledigt ist.
			\item Das System speichert den Task als erledigt ab und stellt dieses visuell dar (Bspw. durchgestrichen, auf Erledigt-Liste verschoben).
				
		\end{verlauf}
	\end{description}

\usecase{Erinnern}			\label{notify}
	\begin{description}
		\actors{Initiiert vom Controller; Listenmitglied ist beteiligt.}
		\begin{verlauf}
		
			\item Das System erkennt durch den permanenten GPS-Abgleich der zugeordneten Listenmitglieder mit den verorteten Tasks, dass sich ein Nutzer gerade in der Nähe des Ortes eines Tasks befindet.
			\item Das System sendet eine Benachrichtigung an den nativen Benachrichtigungsmechanismus des Betriebssystems, aus der hervorgeht, dass der Nutzer sich gerade in der Nähe eines Ortes eines zu erledigenden Tasks befindet.
		\end{verlauf}
%		\begin{altVerlauf}
%
%
%			\item Der Nutzer wählt das Benachrichtigungselement aus.
%			\item [Alternativ:]
%				\begin{itemize}
%					\item Weiter mit Karte anzeigen.		
%					\item Das System wechselt zur Detailansicht (Task bearbeiten) des Tasks.
%				\end{itemize}
%
%		\end{altVerlauf}
	\end{description}

\usecase{Karte anzeigen}	\label{showMap}
	\begin{description}
		\actors{Initiiert vom Listenmitglied}
		\initcondition{Es wurde ein Ort für den betreffenden Task hinterlegt (\namerefH{locateTask}).}
		\begin{verlauf}
		
			\item Der Nutzer signalisiert dem System, dass er den Ort eines bestimmten Tasks auf der Karte ansehen möchte.
			\item Das System wechselt zur nativen Kartenanwendung des Betriebssystems und zeigt (neben dem aktuellen Standort) den Ort des Tasks mit einer Markierung auf der Karte an.
		
		\end{verlauf}
	\end{description}
	